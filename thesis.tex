\documentclass[twoside,11pt,openright,a4paper]{report}

\usepackage{cstitle}
\usepackage[utf8]{inputenc}
\usepackage[american]{babel}
\usepackage{latexsym}
\usepackage{amssymb}
\usepackage{amsmath}
\usepackage{epsfig}
\usepackage[T1]{fontenc}
\usepackage{lmodern}
\usepackage{color}
\usepackage{datetime}
\usepackage{epstopdf}
\usepackage{hyperref}

\usepackage{csquotes}
\usepackage{listings}
\usepackage[backend=biber]{biblatex}

\usepackage{minted}
\usepackage{float}
\floatstyle{boxed}
\restylefloat{figure}

\addbibresource{refs.bib}

\hypersetup{
    pdfborder = {0 0 0}
}
\newcommand{\orangutan}{Orangutan}
\newcommand{\bibtex}{Bib{\TeX}}
\newcommand{\chapref}[1]{(Section~\ref{#1})}
\newcommand{\rfquote}[2]{
\begin{flushright}
\emph{#1}\\
-- #2
\end{flushright}}

\newcommand{\figref}[1]{Figure~\ref{#1}}
\newcommand{\ie}{i.e.}
\newcommand{\eg}{e.g.}
\newcommand{\newdef}[1]{`#1'}
\newcommand{\file}[1]{.#1 file}

\graphicspath{ {graphics/} {./} }
% see http://imf.au.dk/system/latex/bog/

\begin{document}
\pagestyle{empty}
\pagenumbering{Alph}
\begin{titlepage}
\author{Rohde Fischer - 20052356}
\csAdvisor{Olivier Danvy}
\title{An automated approach\\to organizing {\bibtex} files}
\maketitle

\end{titlepage}

%%%%%%%%%%%%%%%%%%%%%%%%%%%%%%%%%%%%%%%%%%%%%%%%%%%%%%%%%%%%%%%%%%%%%%%

\pagestyle{plain}
\setcounter{page}{1}


\chapter*{Abstract}
\addcontentsline{toc}{chapter}{Abstract}

We successively describe the bibliographic software tool {\bibtex}
(both how to use it in principle and how it is used in practice), list
a range of practical issues {\bibtex} users encounter, propose an
approach to hand\-ling these issues, and review how they are tackled
in related work.  We then present an analysis of {\bibtex} files that
detects these issues and we describe how to solve them by organizing
{\bibtex} files using the results of the analysis.  We implemented a
proof of concept, {\orangutan}, that analyzes existing {\bibtex} files
and emits diagnostics and suggestions.


\chapter*{Resumé}
\addcontentsline{toc}{chapter}{Resumé}

Vi beskriver successivt det bibliografiske software-værktøj {\bibtex}
(både, hvordan det bruges i princippet og hvordan det bruges i
praksis), lister en række praktiske problemer {\bibtex}-brugere møder,
foreslår en fremgangsmåde til håndtering af problemerne og en
gennemgang af, hvordan de takles i relateret arbejde.  Vi præsenterer
en analyse af {\bibtex}-filer, der påviser problemerne og vi
beskriver, hvordan disse kan løses ved at organisere {\bibtex}-filerne
ved hjælp af resultaterne fra analysen.  Vi har implementeret et proof
of concept, {\orangutan}, der analyserer eksisterende {\bibtex}-filer
og udskriver diagnoser og forslag.


\chapter*{Acknowledgments}
\addcontentsline{toc}{chapter}{Acknowledgments}

To my advisor, Olivier Danvy, goes a deep thank you.  During the work
of this dissertation his constant support and help has been of immense
value.

Thanks to René Rydhof Hansen 

At the office, provided by Århus University, my three office mates has
been helpful on giving critical feedback, correcting the dissertation
and moral support.  Gratitude goes to Anders Lindkvist, Troels
Fleischer Skov Jensen and Martin Eik Korsgaard Rasmussen.

Livia-Marina Gherghe and Vera Ohlsen has been very helpful on
providing feedback and corrections for the dissertation.

JabRef community for their helpful nature, when asking questions.

Gratitude goes to the staff at Århus University, for providing a good
study environment.  Their diligent work has ensured that the relevant
infrastructure, both system wise and the everyday practicalities, have
been in order.

For Bookstudenterkørsel.dk a huge thanks for providing a student job
with huge understanding and acceptance of my studies, allowing a
perfect unification of both studying and working.  Thanks allowing the
time and providing the space needed for my studies.

To my siblings, Karl and Anne, and to my sister in law, Julija a huge
thanks for the moral support and all the help and feedback provided.

For my good friends Peter Laursen, Iskra Dinkova, Livia-Marina Gherghe
and Nina Sprogøe another huge thanks for the moral support.



\tableofcontents
\newpage

\pagenumbering{arabic}
\setcounter{secnumdepth}{2}

%%%%%%%%%%%%%%%%%%%%%%%%%%%%%%%%%%%%%%%%%%%%%%%%%%%%%%%%%%%%%%%%%%%%%%%

\chapter{Introduction}
\label{ch:intro}
\section{BibTeX format}
BibTeX is a commonly used format for managing lists of references in
academic circles \remark{Now now, can we prove this?}.  The format is
designed for use with {\LaTeX}, but plugins for other formats do
exist. \autocite{bibtex_resource}

The format itself is fairly simple, at it's main level we got
$@STING$, $@PREAMPLE$, $@COMMENT$ and $entries$.  $@STRING$ is for
abbreviations that can be used later in the BibTeX, $@PREAMPLE$ is for
defining how to format the text and $@COMMENT$ is for comments and
$entries$ are the actual entries \autocite{bibtex_resource}.

Since $@PREAMPLE$ and $@COMMENT$ doesn't affect the contents they
won't be considered further in this document. \remark{Double check
  preample, recall reading somewhere that it's just LaTeX macros.}

Tags is case insensitive. If the contents needs to be enclosed in
either \{ and \} or quotes (depending on the persons taste), numbers
can be written without them and @sting abbreviations has to be without
them. Concatenation is done using \#.\autocite{bibtex_resource}

Designed by Oren Patashnik and Leslie Lamport in 1985

(Note to self:) should probably have half an eye on things like:  `G{\"o}del', when working with it
(Note to self:) people may use booktitle where title is appropriate
edition: The edition of a book---for example, ``Second''. This should be an ordinal, and should have the first letter capitalized, as shown here; the standard styles convert to lower case when necessary. \autocite{bibtex_description}

%%% Local Variables:
%%% mode: latex
%%% TeX-master: "thesis"
%%% End:


\chapter{{\bibtex}}
\label{ch:about}
What is Bib{\TeX}

The introduction of {\TeX} has been a huge game changer for scientific
writings, which can be seen by the amount of articles written in
{\TeX} (or derivatives such as {\LaTeX}), for instance almost all
arXiv.org articles have {\LaTeX} sources.  Apart from being grateful
to Donald Knuth for providing a robust system for articles, {\TeX} has
also given the base for Oren Patashnik and Leslie Lamport to create
Bib{\TeX} for bibliography managements.

Before the wake of Bib{\TeX} references was managed entirely by hand.


What is a Bib{\TeX} file

Look, it's nice because

(The problem with Bib{\TeX})

\chapter{The challenges\\in using {\bibtex}}
\label{ch:problem-description}
Inspired by economics, the challenges in Bib{\TeX} can be divided into
structural and conjunctural issues.  The structural issues are the
ones intrinsic to Bib{\TeX}, thus it is caused by the design of and
the standard tools for Bib{\TeX} and information that can not be
found.  The conjunctural issues are the combination of circumstances,
for instance if the source used not having the complete information
(eg. extracting a references from an article where the author only
have initials even though the name is known) or the users having other
priorities than be bibliographies.

Whether an issue is seen as conjectural or structural is in part a
matter of opinion, as the difference can be stretched in either
direction - an issue also faced in economics.  Most of the issues are
arguably a combination the two as they could be fixed by careful labor
or by having the right tools available, eg. most bibliography managers
have tools to switch between abbreviated and full journal names.  The
ultimate goal of {\Orangutan} is essentially to move the issues to the
point where they are purely conjectural as it would be a matter of
using the tool to ensure a well structured Bib{\TeX} document.

An interesting point is that not all of the structural issues are
necessarily bad.  As pointed out earlier there are a lot of ways to
make use of the relaxed properties of Bib{\TeX}, for instance that it
ignores unknown tags is useful in de facto standards such as
commenting entries out by prefixing with $OPT$ or adding information
that are not a part of the Bib{\TeX} specification, such as ISSN, DOI
or the crossref, which is technically not specified by Bib{\TeX} but
still is part of the tool.  In figure-\ref{fig:mendeley_output}
an example is provided by a PhD student from the Chemistry Department
at Århus University, this example is created from Mendeley and shows a
lot of additional information about the article.

\begin{figure}[ht]
  \centering
\begin{small}
\begin{verbatim}
@article{Acatrinei2003,
author = {Acatrinei, Alice I and Browne, D and Losovyj, Y B 
          and Young, D P and Moldovan, M and Chan, Julia Y
          and Sprunger, P T and Kurtz, Richard L},
doi = {10.1088/0953-8984/15/33/101},
file = {:C$\backslash$:/Users/[...]pdf},
issn = {0953-8984},
journal = {Journal of Physics: Condensed Matter},
month = {aug},
number = {33},
pages = {L511--L517},
title = {{Angle-resolved photoemission study 
          and first-principles calculation 
          of the electronic structure of LaSb 2}},
url = {http://iopscience.iop.org/[...]},
volume = {15},
year = {2003}
}
\end{verbatim}
\end{small}
  \caption{Output from Mendeley containing additional information}
  \label{fig:mendeley_output}
\end{figure}

This is an issue, if strict conformity to the specification is desired,
however as this is a very practical and widely used feature a strict
validation would be counterproductive.  Also some formatting styles
actually make use of some of the unspecified tags as can be seen in
figure-\ref{fig:entry_with_issn}. It is still interesting to find tags
that does not conform to the specification and the de facto standards
in use, thus treating de facto standards as a feature and only treating
further deviations as actual issues seems to make most sense. 

% Some issues such as spelling errors are by nature conjunctural, these
% errors could be prevented by having a watchful eye and doing a spell
% check of a bib file.

%%
%% Internal ref 4
%%

Most, if not all, journals require that journal names should be
abbreviated when publishing.  However internally in the Bib{\TeX} file
the owner's personal priorities are consistent and correct naming.  As
Bib{\TeX} can be seen as a database of references, it would make most
sense to consider full names as the most correct and the abbreviations
to be a matter of formatting.  Unfortunately Bib{\TeX} does not handle
abbreviations at all, which for instance is apparent in articles from
arXiv.org, as can be seen in the bbl output in
figure-\ref{fig:inconsistent_naming}.

\begin{figure}[ht]
  \centering
\begin{small}
\begin{verbatim}
\bibitem[\protect\citename{Baroni \bgroup et al.\egroup }2014b]
          {baroni2014don}
          Marco Baroni, Georgiana Dinu, and Germ{\'a}n Kruszewski.
\newblock 2014b.
\newblock Don't count, predict!
          a systematic comparison of context-counting vs.
          context-predicting semantic vectors.
\newblock In {\em Proceedings of the 52nd Annual Meeting of
          the Association for Computational Linguistics},
          volume~1, pages 238--247.

\bibitem[\protect\citename{Bruni \bgroup et al.\egroup}2014]
          {bruni2014multimodal}
          Elia Bruni, Nam-Khanh Tran, and Marco Baroni.
\newblock 2014.
\newblock Multimodal distributional semantics.
\newblock {\em J. Artif. Intell. Res. (JAIR)}, 49:1--47.

[...]

\bibitem[\protect\citename{Collobert \bgroup et al.\egroup}2011]
          {collobert2011natural}
          Ronan Collobert, Jason Weston, L{\'e}on Bottou,
          Michael Karlen, Koray Kavukcuoglu, and Pavel Kuksa.
\newblock 2011.
\newblock Natural language processing (almost) from scratch.
\newblock {\em The Journal of Machine Learning Research},
          12:2493--2537.

[...]

\bibitem[\protect\citename{Kalchbrenner \bgroup et al.\egroup}2014]
          {kalchbrenner2014convolutional}
          Nal Kalchbrenner, Edward Grefenstette, and Phil Blunsom.
\newblock 2014.
\newblock A convolutional neural network for modelling sentences.
\newblock In {\em Proceedings of EMNLP}.
\end{verbatim}
\end{small}
  \caption{Inconsistent naming of journal and conference names}
\label{fig:inconsistent_naming}
\end{figure}

From the point of view that the style of Bib{\TeX} should format
abbreviations properly the issue seen here is structural, as the long
names should have been formatted by the tool.  In cases where the
abbreviation is wrong (eg. due to a typo) the issue would move towards
being conjectural again, unless some kind of abbreviation specific
spell checker is being used.  Using full names and then formatting
them accordingly seems like the most sensible idea, since the style of
abbreviation could be interchanged, should the need arise, it is more
readable and it would create better conditions for output tools to
provide consistent formatting.

Currently there are multiple strategies for ensuring consistency in
abbreviations, some do a complete search and replace on the bib-file
depending on the formatting they need, a bit more structured approach
is the use of strings in Bib{\TeX} to ensure a consistent naming of a
journal which can further be combined with the use of crossref.
Another approach is the use of Bib{\LaTeX} and biber, which provides
the solution in the formatting
options\cite{koppensteiner2011abbreviate}, provided that the
abbreviation handing of the style is correct.  This causes it to
become a more conjectural issue, as the issue now is if the right
names are written, however how conjectural depends on the correctness
of the tools.  Reference mangers such as JabRef, Mendeley etc. tend to
go with the strategy that it stores the references using full names
and then export it to a Bib{\TeX}-file (or whatever format you decide
to export to) having the desired abbreviation style applied to the
export, this again moves the issue towards being conjectural, for the
same reasons as the Bib{\LaTeX} and biber solution.

As the purpose of {\Orangutan} is to work on the Bib{\TeX} file
itself, the formatting in the end is technically not the primary
concern.  The concern would optimally just be to ensure a consistent
document, so the style can do its work.  Since Bib{\TeX} styles
currently does not take care of abbreviations directly, there might be
the need for considerations on how to easily and consistently ensuring
abbreviations according to a desired format.  Having a solution that
ensures a consistent structure, which is easy to modify to a desired
style of abbreviations, would move the issue towards the conjectural.
Having the styles actually handling the abbreviations (such as the
Bib{\LaTeX} and biber combination does) would make the issue entirely
conjectural if there is a an easy way to ensure that all journal
names, conference names etc. are always in correct full names.

%%
%% Internal ref 1
%% journal unknown oO
%%

Take the inconsistency in figure-\ref{fig:entry_with_issn}, found in a
article on arXiv.org, have two references from the same conference,
but different years.  The inconsistency is easy to identify due to the
consistent naming, which in turn affects how the mistake is percieved.
Correct and consistent naming will make it possible for tools to
detect inconsistencies when some fields are missing.  This exposes a
structural part of the issue, as no such tools exists (to the authors
knowledge).

\begin{figure}[ht]
  \centering
  \begin{small}
\begin{verbatim}
\bibitem[Bernardy and Claessen(2015)]{bernardy_efficient_2015}
J.-P. Bernardy and K.~Claessen.
\newblock Efficient parallel and incremental parsing
          of practical context-free languages.
\newblock \emph{J. of Funct. Prog.}, 25, 2015.
\newblock ISSN 1469-7653.
\newblock \doi{10.1017/S0956796815000131}.

[...]

\bibitem[Mu et~al.(2009)Mu, Ko, and Jansson]{MuKoJansson2009AoPA}
S.-C. Mu, H.-S. Ko, and P.~Jansson.
\newblock Algebra of programming in {Agda}:
          dependent types for relational program derivation.
\newblock \emph{J. Funct. Program.}, 19:\penalty0 545--579, 2009.
\newblock \doi{10.1017/S0956796809007345}.
\end{verbatim}
  \end{small}
  \caption{Additional tag ISSN is provided in one of the entries}
\label{fig:entry_with_issn}
% consider the pages
\end{figure}

It might be the case that an ISSN simply does not exist for the
$MuKoJansson2009AoPA$-entry, in this case a structural detection
system might be the cause of new structural issue, either the removal
of relevant data or forcing entries for cases where it is not
relevant.  In this specific case the search result in
figure-\ref{fig:entry_issn_found} reveals that the missing ISSN do
exist and thus a tool pointing out the inconsistency would in thise
case make the issue conjunctional.

% Source http://journals.cambridge.org/action/displayAbstract?fromPage=online&aid=6171388&fileId=S0956796809007345#
\begin{figure}[ht]
  \centering
\begin{verbatim}
@article{Mu:2009:APA:1630623.1630627,
 author = {Mu, Shin-cheng and Ko, Hsiang-shang 
           and Jansson, Patrik},
 title = {Algebra of Programming in Agda: 
          Dependent Types for Relational Program Derivation},
 journal = {J. Funct. Program.},
 issue_date = {September 2009},
 volume = {19},
 number = {5},
 month = sep,
 year = {2009},
 issn = {0956-7968},
 pages = {545--579},
 numpages = {35},
 url = {http://dx.doi.org/10.1017/S0956796809007345},
 doi = {10.1017/S0956796809007345},
 acmid = {1630627},
 publisher = {Cambridge University Press},
 address = {New York, NY, USA},
 }
\end{verbatim}
  \caption{Search revealing the ISSN}
\label{fig:entry_issn_found}
\end{figure}

%%
%% Internal ref 2
%% Source Electronic Proceedings in Theoretical Computer Science
%%

Provided a reliable way to look up correct entries a tool could move
the issues towards being conjuntural, since it would then just being a
matter of doing the lookup on all entries and then be done (apart from
the abbreviation issues above).  In the utopic case all entries could
be looked up at all times, but this is not likely as unpublished works
would need a reliable way to be registered.  Also even though the
databases out there are really good, most of the time, erronous
results can be found.  A lookup on Google Scholar in the beginning of
February 2016 for: ``Results and Analysis of SyGuS-Comp’15'' can be
seen in figure-\ref{fig:scholar_bad_result}.

\begin{figure}[ht]
  \centering
\begin{verbatim}
@article{alurresults,
  title={Results and Analysis of SyGuS-Comp’15},
  author={Alur, Rajeev and Fisman, Dana and Singh, Rishabh
          and Solar-Lezama, Armando}
}
\end{verbatim}
  \caption{Bad result from Google Scholar}
\label{fig:scholar_bad_result}
\end{figure}

Having found the article originally on arXiv.org the source of the
article is known to be EPTCS - Electronic Proceedings in Theoretical
Computer Science.  So not only does the Google Scholar result actually
not conform to the requirements of an article, the resource are in
fact not an article at all, but in the proceedings to a conference.
Finding the correct entry details at the EPTCS page reveals the entry
in figure-\ref{fig:eptcs_lookup}.  Relying on these being correct
would cause structural issues as it would the tools would now
automatically introduce new errors.

\begin{figure}[ht]
  \centering
\begin{small}
\begin{verbatim}
@Inproceedings{EPTCS202.3,
  author    = "Alur, Rajeev and Fisman, Dana  and Singh, Rishabh 
               and Solar-Lezama, Armando",
  year      = "2016",
  title     = "Results and Analysis of SyGuS-Comp'15",
  editor    = "\v{C}ern\'y, Pavol and Kuncak, Viktor 
               and Parthasarathy, Madhusudan"b,
  booktitle = "{\rm Proceedings Fourth Workshop on}
               Synthesis,
               {\rm San Francisco, CA, USA, 18th July 2015}",
  series    = "Electronic Proceedings in 
               Theoretical Computer Science",
  volume    = "202",
  publisher = "Open Publishing Association",
  pages     = "3-26",
  doi       = "10.4204/EPTCS.202.3",
}
\end{verbatim}
\end{small}
  \caption{Correct lookup on EPTCS, after failed lookup on Google Scholar}
\label{fig:eptcs_lookup}
\end{figure}

% mark

\begin{figure}[ht]
  \centering
  \begin{small}
\begin{verbatim}
\bibitem[Bernardy and Claessen(2013)]{bernardy_efficient_2013}
J.-P. Bernardy and K.~Claessen.
\newblock Efficient divide-and-conquer parsing 
          of practical context-free languages.
\newblock In \emph{Proc. of ICFP 2013}, pages 111--122, 2013.

[...]

\bibitem[Danielsson(2010)]{danielsson_total_2010}
N.~A. Danielsson.
\newblock Total parser combinators.
\newblock In \emph{Proc. of ICFP 2010}, ICFP '10,
          pages 285--296. ACM, 2010.
\end{verbatim}
  \end{small}
  \caption{Capt}
\label{fig:inconsistent_proceedings}
\end{figure}

A search for Bib{\TeX} entries give then entries in
figure-\ref{fig:missing_org_scholar_lookup} for the two articles, which
provides one possible option for a set of consistent entries.

\begin{figure}[ht]
  \centering
\begin{verbatim}
@inproceedings{bernardy2013efficient,
  title={Efficient divide-and-conquer parsing
         of practical context-free languages},
  author={Bernardy, Jean-Philippe and Claessen, Koen},
  booktitle={ACM SIGPLAN Notices},
  volume={48},
  number={9},
  pages={111--122},
  year={2013},
  organization={ACM}
}

@inproceedings{danielsson2010total,
  title={Total parser combinators},
  author={Danielsson, Nils Anders},
  booktitle={ACM Sigplan Notices},
  volume={45},
  number={9},
  pages={285--296},
  year={2010},
  organization={ACM}
}
\end{verbatim}
  \caption{Scholar lookup}
\label{fig:missing_org_scholar_lookup}
\end{figure}

As can be seen $ACM$ is the name of the organization and is probably
missing in the original Bib{\TeX} that produced the bbl file inspected
above.  The Denielsson has an additional tag with the content ``ICFP
'10'', which got added \remark{Olivier: do you happen to know what tag
  might have produced this?}

% ref 3

Following this example from arXiv.org we have entries with the same
conference where it can be seen that the conference name is different
in one of the entries in figure-\ref{fig:entry_journal_name_authors}
\remark{additional meta information perhaps}.  Also the list of author
names are in the second entry so heavily abbreviated that you cannot
realistically distinguish who the authors are, even if you know the
affiliation.

\begin{figure}[ht]
  \centering
\begin{small}
\begin{verbatim}
\bibitem{stanifordchen96grids}
S.~S.-C. \emph{et al}.
\newblock {GrIDS} -- {A} graph-based intrusion detection system 
          for large networks.
\newblock In {\em Proceedings of the 19th
          National Information Systems Security Conference},
          1996.

[...]

\bibitem{porras97emerald}
P.~A. Porras and P.~G. Neumann.
\newblock {EMERALD}: Event monitoring enabling responses 
          to anomalous live disturbances.
\newblock In {\em Proc. 20th {NIST}-{NCSC}
          National Information Systems Security Conference},
          pages 353--365, 1997.

\end{verbatim}
\end{small}
  \caption{Inconsistent reference to the conference and heavily abbreviated author names}
\label{fig:entry_journal_name_authors}
\end{figure}

% http://csrc.nist.gov/nissc/
% Different ways of writing it (Proc vs Proceedings, {NIST}--{NCSC} in one, pages omitted)
% even better, consider this entry:
Furthermore, consider the case where we use these proceedings to infer
some consistent way of writing the entries, based on the journal
names.  It turns out that ``National Information Systems Security
Conference'' used to be named ``National Computer Security
Conference'', which is probably the reason for the
$\{NIST\}--\{NCSC\}$ part of the first entry.  This results is that to
correctly identify potentiel areas of inconsistencies it should also
recognize entries such as the one in figure-\ref{fig:conference_name}.

\begin{figure}[ht]
  \centering
\begin{small}
\begin{verbatim}
\bibitem{snapp91dids}
S.~R.~S. \emph{et al}.
\newblock {DIDS} (distributed intrusion detection system) -
          motivation, architecture, and an early prototype.
\newblock In {\em Proceedings of the 14th
          National Computer Security Conference},
          pages 167--176, Washington, DC, 1991.
\end{verbatim}
\end{small}
  \caption{Name change of a conference}
\label{fig:conference_name}
\end{figure}


\remark{There should be some information about Bib{\TeX} creating a bbl
file, since me references come from there}


\chapter{Our approach to {\bibtex}}
\label{ch:approach}
\section{Introduction}
The goal of this chapter is to organize the issues people have with
{\bibtex}.  Answering what the issues in {\bibtex} are
\chapref{sec:intro_what_issues} and how to approach the {\bibtex}
issues \chapref{sec:intro_what_to_do}

\section{What are the issues in {\bibtex}}
\label{sec:intro_what_issues}

{\bibtex} has changed the landscape for scientific writing and eased a
lot of peoples lives, however not without any issues.  The challenges
range widely and trying to group similar looking issues we have:

\remark{We may want the same order as in the list of issues, probably
  reformulate this}

Conforming to the specification and de-facto standards is considered
as a correctness concern.  The use of abbreviations for journal names,
initials in author names, misspellings in general, misspellings in
names and {\bibtex} strings that end up being text will be considered
as lexical concerns.  All of these issues will be considered combined
as \newdef{correctness and lexical concerns}.

Inconsistent use of tags, inconsistent entry keys, forum names that
change and duplicate entries will be considered as \newdef{consistency
  concerns}.

The Utopian goal is to provide a structural solution for all these
issues so that if any further issues exist they would be purely
conjunctural.


\section{What can be done about the {\bibtex} issues}
\label{sec:intro_what_to_do}

As previously stated, the structural approach to the issues is desired.
This choice basically means ensuring that the tools handles the
issues, preferably to the level where all issues are solved.  As
touched upon shortly when inspecting the problems, it is not likely
that all issues can be solved perfectly.  For a structural solution
there are two approaches:


\subsection{Updating or replacing {\bibtex}}

One way of handling the issues structurally would be to change or
replace {\bibtex}, so it handles all lexical and consistency concerns.
This way would include changing the {\bibtex} specification to account
for relevant de facto standards, enforcing conformity, handling
abbreviations and controlling all data.  The updated version of
{\bibtex} could then either correct the issues when running into them
or fail building the \file{bbl} with appropriate error messages for
issues that the user need to take care of.

This approach would be probably be perceived as invasive as it would
cause existing {\bibtex} files not to work and it would impose the
tool on the users with requirements they may not desire.  The
perception would of course depend of perspective because the user who
wants structure and control might find it good that it is enforced.

As will be inspected in Chapter~\ref{ch:related} there are actually a
few attempts at both changing and replacing {\bibtex}.


\subsection{Augmenting {\bibtex}}

Instead of changing or replacing {\bibtex} an augmenting tool is
another option.  Such a tool be used together on {\bibtex} file
provide or suggest improvements, in stead of changing specifications.
An augmenting tool will be a supplement to current use of {\bibtex}
and be optional, rather than imposed on the users.


\section{How do we approach the {\bibtex} issues}
\subsection{Introduction}

The goal of this section is to introduce our choice of solutions for
the issues.


\subsection{Lexical and correctness concerns vs. consistency concerns}
\label{sec:approach_lexical_consistency}

The relation between the lexical and correctness concerns and the
consistency concerns reveals a dependency in the analysis.  Going
through the consistency concerns observing their relation gives:

\begin{itemize}
\item For inconsistent use of tags, a way to determine if entries are
  from the same forum is needed.  Such a way depends on consistent
  naming of the forum and a way to detect name changes.

\item For duplicate entries having unique identifiers such as arXiv
  numbers, ISSN or DOI will make the detection trivial.  Otherwise the
  detection has to be based on the similarity of the information, at
  best the information is identical, otherwise it has to be a similar
  as possible to improve the detection.  Thus solving the lexical
  concerns will be of use.

\item Inconsistent naming of entry keys can be handled by a naming
  scheme.  Such a naming scheme is usually based on the information in
  the entries.  So having the relevant tags and correct content in
  them will ensure a way to ensure consistent entry keys.

\item For name changes of forums we need to be able to recognize the
  names which is easier with correct and consistent names.
\end{itemize}

A common property about the consistency concerns is that they are
easier to handle, once the lexical and correctness concerns have been
handled.  This property indicates that a two phase solution may be
desired: first handling the lexical and correctness concerns, then
handling the consistency concerns.


\subsection{\nameref{sec:problems_duplicates}}
\label{sec:approach_duplicates}

Duplicate entries are fairly straight forward if the tags and the
contents are identical.  If the content and tags deviate a way of
detecting ``similarity'' will be needed, the easiest definition of
similarity is if the title and author is identical, however this
definition might need to take things like revisions and year into
account, as an author may decide to write a new version later or if
one for some reason desires to refer to different revisions.  Further
challenges may arise if there is lexical and correctness challenges as
per Section~\ref{sec:approach_lexical_consistency} fixing these first
is desired.


\subsection{\nameref{sec:problems_spelling}}

To detect misspellings a spell checker can be used.  Alternatively
checking the resources in online databases is an option.  If a spell
checker is used one should be aware false positives.  Domain specific
terms might not be present in the dictionary and if the original
source is misspelled, it would be a mistake to correct it (once
published the name published is the correct name of the reference!).
Provided a solution for the issues in online look ups the correct
spelling will be a matter of looking up, but references may not be in
the databases, and as stated in Section~\ref{sec:approach_look_ups}
there is no good way to ensure correct look ups.  So a spell checker
seems like a good way to get an indication of possible errors, but one
would still have to verify them.  As entries may be in different
languages a way of specifying the language should be considered.


\subsection{\nameref{sec:problems_spelling_names}}
\label{sec:approach_spelling_names}

Spell checking names with a normal spell checker will cause errors.  A
possible way to approach this would be to make online look ups in
databases with scientific authors, such as DBLP and Google Scholar.
This will still have the issue with name of authors who have not
published anything scientific, such the author of a book.  Extending
this solution to contain more databases such as book authors will
improve the solution, but will still be limited to known author names.


\subsection{\nameref{sec:problems_initials}}

Finding the initials is a matter of being able to detect single
letters with or without a period after it.  However if one for some
reason group initials together, \eg, making George R. R. Martin into
George RR Martin, then further detection will be needed.  Replacing
the initials with full names is appropriate whenever possible, but
since the full names may not be known some way of specifying that
initials is the only thing available is needed.  The best approach
will probably be the one described for spell checking names in
Section~\ref{sec:approach_spelling_names}.


\subsection{\nameref{sec:problems_look_ups}}
\label{sec:approach_look_ups}

Online database look ups can be a very useful tool for handling the
lexical and correctness concerns, but getting incorrect data can cause
problems.  The best approach to ensure correct look ups is if it is
possible to use services that are known to be correct.

A situation where relatively reliable look ups is possible, as in the
``EPTCS'' look up seen in \figref{fig:eptcs_lookup}, can be used to
improve the reliability of the results.  However there is still no
certain way to know if the database of the service is correct, so it
is still not certain.  Most likely the ID systems, such as arXiv
numbers, DOI and ISSN, will also provide a relatively reliable look up
mechanism, but it is still not guaranteed.

Another way to approach the bad look ups could be by doing the same
look up in multiple databases and then have some kind of voting system
that decides on which entry to trust.  This would however require
knowledge information sources each online service use, because their
source of information may be the same and then the same error could
get multiple votes.  The approach can be refined by having increased
trust in databases that are likely to be correct.

The most appropriate strategy is probably selecting the database most
likely to be correct and then have the user select if one agrees with
the result.  Doing the vote system would be overkill in most
situations, and the user still have to validate the result, since the
voting system will not provide a certain correct result.  Having the
user validate the result will make issues partly conjunctural, if one
just accept any result from the look up.


\subsection{\nameref{sec:problems_name_changes}}

Handling name changes of forums is supportive to ensuring consistency.
Since name changes cannot be derived automatically one approach would
be a database of known name changes, which has the disadvantage that
it needs to be maintained.  Adding a configuration to specify name
changes may also be appropriate.


\subsection{\nameref{sec:problems_de_facto}}

As stated in Section~\ref{sec:problems_de_facto} it is desired to be
able to validate if the file conforms to the specification and the
desired de-facto standards.  Validating conformity to the
specification is a simple task, as the specification is just a set of
rules.  A set of de-facto standards, likewise, is also a simple set of
rules.

De-facto standards however provide challenges, as they are the
standards currently in use.  This means that they both depends on who
the user is and the standards are subject to change.

A tool handling conformity to the specification and de-facto standards
should thus be configurable to account for changes in de-facto
standards.  For practicality the de-facto standards that are not
likely to change (such as ISSN and DOI) could be accounted for with a
default setting.


\subsection{\nameref{sec:problems_abbreviations}}

Ensuring a consistent use of either abbreviations or full names is
desired.  From the point of having the information in a complete
version converting full names, \newdef{de-abbreviating} is desired.
Using a database of standard abbreviations for forums will be useful
to de-abbreviate.  Taking care of a consistent way to switch between
full names and abbreviations is also desired.  Making use of strings
to handle the switching between full names and abbreviations is
probably the best approach since this will keep it clear which forum
is which.  This also allows the user to use string names that are:
full names, official names or their own style of abbreviations, to
their choice.


\subsection{\nameref{sec:problems_strings_as_text}}

A {\bibtex} string can end up being part of a text by mistake, in the
example used in Section~\ref{sec:problems_strings_as_text} where the
month ended up as a text rather than a string a simple check is
possible, because for a month we know what to expect.  Whenever
something in the middle of the text should have been a string, the
text would have to be checked for potential strings.  Automatically
correcting it would introduce a potential source of errors, because a
text being identical to a string name could just be a coincidence, so
it has to be the user's choice.


\subsection{\nameref{sec:problems_inconsistent_tags}}

Detecting inconsistent use of tags require a way of detecting when
entries are from the same forum.  When such a way is provided it is
possible to check if the set of fields are the same.  Having some kind
of statistics on the usage may further improve the feedback, since it
will be possible to suggest that the shortest path to consistency: if
it is by adding, or removing tags.

Since a lot of the forums are continuous, such as a conference being
held each year, the detail level of the information may change over
time.  Also in some cases a single item can have additional
information that are not general to the forum or for some reason not
have information according to the general standard.  Optimally there
should be a way to account for these cases, either by the user
enforcing conformity or having options for when a deviation occur.


\subsection{\nameref{sec:problems_inconsistent_keys}}

Provided that the lexical and correctness concerns has been solved,
handling inconsistent entry keys require very little effort.  Having a
rule for how the key names should be formatted is all that is needed.
Like in Section~\ref{sec:approach_duplicates} there is the issue of
similar, but different, entries.  Similar entries could result in the
same entry key, so there is the need for a way to disambiguate the key
names.  Since a lot of users already have databases in use, support
for one specifying a naming scheme would be appropriate.


%\subsection{Summary  and conclusions}
%To summarize:

\section{Summary and conclusions}

The issues in {\bibtex} files have been grouped in correctness and
lexical concerns and in consistency concerns.  Updating or replacing
{\bibtex} was compared to augmenting {\bibtex}.  It was observed that
the consistency concerns in general depends on the solution of the
correctness and lexical concerns.

\begin{itemize}
\item Duplicate entries can be found if there is a unique identifier
  or identical entries.  To detect deviating duplicates a definition
  of ``similarity'' will be needed.

\item Spelling errors in general can be solved by a spell checker and
  the usage of online look ups.  For a spell checker one must be aware
  of false positives and language.

\item Spelling errors in names will challenge normal spell checkers.
  Using online databases of authors will enable some checking but the
  solution will be limited to known author names.

\item Initials hiding peoples names can be handled by online
  resources, just as misspellings in names.

\item To get online look ups that contains the correct data is
  impossible, the results can be improved by selecting the most
  appropriate database for the look up and by introducing detection of
  erroneous look ups.

\item Name change of a forum can be handled by a database and by a way
  to specify name changes.

\item Conformity to de-facto standards and the {\bibtex} specification
  should be handled by checking the rules for the standards and being
  able to specify the desired de-facto standards.

\item Journal abbreviations should be moved into strings and be
  consistently de-abbreviated or abbreviated.

\item {\bibtex} strings that end up as part of the text should be
  detected by matching string names that appear in text.

\item Inconsistent tags should be detected based on when entries are
  from the same forum.  One should be able to specify deviations from
  the general set of information for the forum.

\item Inconsistent entry keys should be handled by a rule for the
  desired format for entry key names.
\end{itemize}


%\begin{figure}
%  \centering
%  \begin{verbatim}
%This is BibTeX, Version 0.99d (TeX Live 2015/Debian)
%The top-level auxiliary file: doc.aux
%The style file: plain.bst
%Database file #1: mybib.bib
%Warning--can't use both author and editor fields in book
%Warning--empty publisher in book
%Warning--empty year in book
%Warning--empty journal in Nobody06
%Warning--empty year in Nobody06
%\end{verbatim}
%  \caption{Example of error output from {\bibtex}}
%\label{fig:bibtex_out}
%\end{figure}


%\remark{People who request similar
%http://stackoverflow.com/questions/13630584/the-best-method-for-handling-bibtex-files
%http://stackoverflow.com/questions/32838020/awk-how-to-clean-bibtex-files
%http://www.latex-community.org/forum/viewtopic.php?f=50&t=12043
%http://tex.stackexchange.com/questions/76420/cleaning-up-a-bib-file
%http://tex.stackexchange.com/questions/174509/is-there-a-tool-service-that-can-enrich-a-bibtex-database
%<http://tex.stackexchange.com/questions/128989/how-can-one-validate-a-bib-file
%http://tex.stackexchange.com/questions/173621/how-to-validate-check-a-biblatex-bib-file
%http://stackoverflow.com/questions/13630584/the-best-method-for-handling-bibtex-files
%}


%%% Local Variables:
%%% mode: latex
%%% TeX-master: "thesis"
%%% End:


\chapter{Related work}
\label{ch:related}
\section{Mendeley}

%%% Local Variables:
%%% mode: latex
%%% TeX-master: "thesis"
%%% End:


\chapter{Analyzing {\bibtex} files}
\label{ch:analyzing}
\rfquote{Testing shows the presence,\newline
  not the absence of bugs.}{Edsger W. Dijkstra}


\section{Introduction}
The goal of this chapter is to show how {\bibtex} files are analyzed.


\section{What should be done about {\bibtex}}
\subsection{In principle}

Due to the practical issues in changing/replacing {\bibtex} and to
ensure separation of concerns an analyzing tool should be an
augmenting tool.

To analyze {\bibtex} one would need to parse a \file{bib} into a
suitable representation.  This representation would need to parse
{\bibtex} entries and strings at a minimum.  If one intents to pretty
print the result after resolving all issues, the parsing also needs
the preambles.  Comments are technically optional, but should be kept
too.

The easiest approach is a two step parse: first taking care of the
lexical and correctness concerns, then the consistency concerns.  By
taking care of the lexical and correctness concerns first the optimal
conditions for taking care of the consistency concerns is made.

The term database i here use for both local and online databases,
since it does not matter if one use a local copy, if it is up to date.

\subsubsection{First step: Lexical and correctness}
\begin{itemize}
\item Spelling errors in general, should be detected either by online
  lookups or a spell checker.  A combination of the two could be
  powerful, since it gives a potential indicator of false positives.
  Using a spell checker requires a way to configure the language for
  the individual entries.

\item Spelling errors in names should use databases with
  names for detection.

\item Initials can be detected by finding cases with a single letter,
  perhaps followed by a punctuation.  For suggestions databases with
  names will be useful.  Handling multiple letters combined can be
  done by looking for relatively few (probably up to 3) letters that
  are all capitalized.

\item Online lookups, will be impossible to detect for certain,
  because the intended search will be unknown.  The detection of the
  other issues can give an indicator of a bad lookup.  Since online
  lookups is a likely way to handle the databases needed for this
  tool, the quality of a lookup is a concern.  The most trusted online
  database available should be used whenever possible.  A system doing
  lookups in multiple databases will give further indications of
  potential issues.  The results should always be confirmed by the
  user, to prevent erroneous data.

\item Conformity to de-facto standards and the {\bibtex} specification
  requires a set of rules specifying: required tags, optional tags,
  exclusive tags, and inclusive tags.

\item Detecting invalid values should be done where there are clear
  rules for a correct value can be specified: such as ISSN, year, and
  month.  Furthermore, values that can be verified using a database
  should also be verified.

\item To detect journal names, in abbreviated form, or in full form,
  is done using a database of know journal names and their
  abbreviations.  It should use a database to de-abbreviate or
  abbreviate.  Furthermore this can be refined by detecting unknown
  abbreviations.

\item To handle {\bibtex} strings that end up as part of the text, the
  text should be compared with the strings in the \file{bib}.
\end{itemize}


\subsubsection{Second step: Consistency}
\begin{itemize}
\item To detect duplicate entries, using unique identifiers is the
  best way, then based on identical entries.  Otherwise potential
  duplicates is detected by a specification of similarity: having the
  title and authors as primary indicators.

\item To detect name changes of forums, a database of known changes is
  used, in conjunction with a way of specifying the changes.

\item Inconsistent tag usage should be handled by mapping entries from
  the same forums and comparing the tags in use for missing and
  additional tags.  Comparing forum entries over time will increase
  the usefulness, but adds the need to handle changes in the tags in
  use over time.

\item Inconsistent entry keys should be handled by having a naming
  scheme based on the data in the entries, with a way to disambiguate
  if two different entries would get the same name.
\end{itemize}


\subsubsection{Configuration}

Both of the steps above require ways to configure the behavior.  The
use of configurations should be as close to {\bibtex} as possible.

The preference to using the {\bibtex} format allows people to use what
they are familiar with.  Using a format that are readily supported in
programming frameworks, like JSON or XML might be considered easy by
the computer scientist, especially if it is one used to working with
those formats.  However, people outside computer science, such as a
physicist or the helpful chemist from earlier, will likely not be
familiar with such formats, nor should they.  A user of {\bibtex}
should at best only be concerned with {\bibtex}, when working with
{\bibtex}.

To use something in anger is an idiom for if something has been tested
in practice.  The idea of coding in anger has been expressed by Philip
Wadler~\cite{wadler1997_functional}.  For the {\bibtex} user, coding
in anger, could be the situation where the deadline is getting closer
and one just need things to work, at best ten minutes ago.  When faced
with the frustrations like that one tend not to care about beautiful
and elegant solutions, but rather wanting something that works with a
minimum of effort.  To accommodate the user writing {\bibtex} in
anger, is another reason to keep the specification as close to
{\bibtex} as possible, since it will minimize the effort.

Configuration should be done via de-facto standards inside the
\file{bib}, whenever possible.  For some configurations, de-facto
standards inside the \file{bib} is unreasonable.  These configurations
is better put into separate files.  However, the specification of
should still be designed to match {\bibtex} as closely as possible,
\ie, still using entries, tags and values to configure.


\subsection{In practice}

\subsubsection{De-facto configurations}

To account for the specifications about introducing two de-facto
standards will be appropriate: \texttt{OLDforum} to mark a previous
name of a forum, and \texttt{OPTanalyze} to configure tell the tool
about entry specific details.

The division into two de-facto standards is twofold: for
\texttt{OLDforum} the additional standard will allow bibliography
styles to make use of the additional information (one could easily
imaging a bibliographic style write ``\texttt{NISSC \textit{formerly
    known as} NCSC}''), and for \texttt{OPTanalyze} the content is
considered unlikely to be relevant to print in a bibliography.
Furthermore, the settings for \texttt{OPTanalyze} is kept in one tag
to prevent a multitude of new tags.

\remark{Yikes, any way to make the following clearer?!?!?}

For the \texttt{OLDforum} tag the value should be the string containing the
old name for the forum.  In some cases a forum can have changed its
name multiple times.  When multiple name changes has happened, and
referring to an old forum name, referring the most recent name
actively used in the \file{bib} is desired.  The design of the tag is
intended for disambiguation within a given file, not all files in
general.  If the tag is used for a bibliography style having a
multitude of names as the value will likely be confusing.  However,
only using names from the \file{bib} will cause the need for
re-detecting name changes, if an entry with a forum name between the
two entries are introduced.  Alternatively the format could be a list
of names (comma separated, since {\bibtex} use a comma as a separator),
this list would allow detailed backtracking.  A design with detailed
backtracking

Defining entry level deviations is done using \texttt{OPTanalyze},
using spaces to separate multiple settings.  The values for desired
deviations is as follows:

\begin{itemize}
\item \texttt{@DUPLICATEOK=X} to specify that an entry marked as a
  potential duplicate if deliberate, replacing \texttt{X} with the
  entry key of the potential duplicate.
\item \texttt{@LANG=XX} to specify the desired spell check language,
  replacing \texttt{XX} with the language code desired.
\item \texttt{@SPELLINGOK} to mark the spelling as correct.
\item \texttt{@NAMESOK} to mark that the names are correct.
\item \texttt{@INITIALSOK} to mark that the initials are correct.
\item \texttt{@NOLOOKUP} to mark that no look up should be done for
  the content of this entry.
\item \texttt{@CONFORMITYOK} to mark conformity to the specification
  and de-facto standards as correct.
\item \texttt{@ABBREVIATIONOK} to mark an abbreviated form as correct.
\item \texttt{@STRINGSOK} to mark that the texts has been checked for
  strings and that is is correct.
\item \texttt{@TAGSOK=forum} to mark that the usage of tags is correct and
  defines the standard for tag use for the entries from the \emph{same
  occurrence} of the forum.
\item \texttt{@TAGSOK=future} to mark that the usage of tags is
  correct and defines the standard for tag use for the entries from
  the \emph{same and future occurrences} of the forum.
\item \texttt{@TAGSOK=single} to mark that the usage of tags is
  correct for this single entry, not affecting other entries from the
  forum.
\item \texttt{@ENTRYKEYOK} to mark the entry key as correct.
\item \texttt{@LEXICALLYOK} to ignore all lexical checks for the
  entry, should be used with care.
\item \texttt{@CONSISTENCYOK} to ignore all consistency checks for the
  entry, should be used with care.
\item \texttt{@OK}, to mark an entry as fully correct, same as
  \texttt{@CONFORMITYOK @LEXICALLYOK @CONSISTENCYOK}, should be used
  with care.
\end{itemize}

The settings: \texttt{@ABBREVIATIONOK}, \texttt{@LEXICALLYOK},
\texttt{@CONSISTENCYOK}, and \texttt{@OK} may be a bit debatable on
necessity, however, their presence do ensure that the configuration is
complete and consistent.

For the conformity and de-facto check, it would be and option to have
settings for explicitly specifying the deviations in the use of tags.
However, explicitly allowing and denying tags will in most cases be
redundant, since once the deviation has been accepted it is accepted.

A similar set of tags could also be defined for the consistency check.
For the consistency check, it can be argued that an entry not
conforming to the standards of a forum may be updated to do so.  For
instance, if the forum standard have an ISSN on all entries and some
of the entries do not have an ISSN at the time.  Those entries might
get an ISSN later, which we want add, and in that case the
\texttt{@CONSISTENCYOK} become redundant, knowing this by explicitly stating
that the missing ISSN is the reason.  Another approach to
configurations becoming redundant would be to expand the tool to
detect configurations that serve no purpose.  This approach is
considered more appropriate, because it ensures more simplicity for
the user.

Another potential option would be to have a configuration for trusted
lookup services.  For example is one knows that an entry is correct in
a certain database then specifying that this database is to be
trusted.  This configuration might lead to a false sense of security,
since there is no way to guarantee that the data will not, some day,
be corrupted in the database of the lookup service.

An example of the configurations can be seen in
\figref{fig:analyzing_added_de_facto_standards}

\remark{Fill in the example}
\begin{figure}
  \centering
\begin{verbatim}
@{
  OLDforum = "",
  OPTanalyze = "@LANG=DA @NAMESOK @STRINGSOK"
}
\end{verbatim}
  \caption{An example using the de-facto standards for configuration}
  \label{fig:analyzing_added_de_facto_standards}
\end{figure}

To make the configurations even more intuitive for a {\bibtex} user,
an option is to add {\bibtex} strings with the relevant options in the
top of one's \file{bib}.  Then when configuring one can just use the
{\bibtex} strings and concatenate the relevant configurations.


\subsubsection{Other configurations}

Some of the configurations desired is not specific to an entry, but
rather the entire \file{bib}.  These configurations need to be
specified outside the specific entries.  Two options is: to put such
options inside one's \file{bib}, or put them in separate files.
Adding them to one's \file{bib} will introduce additional mess in the
file, and also the configurations may not correspond to proper
{\bibtex} formatting.  Having the configurations in separate files,
provides separation of concerns.  Furthermore, if the configuration is
in separate files they can easily be shared, such as a research
department having a standard, or a publisher who want authors to
follow their setup.

The format for such configurations should follow the format for a
{\bibtex} file.  Thus to the extend possible, it should consist of
entries with tags and values for configuration.  For other settings
defining {\bibtex} strings will be the favored choice.

For changes in names of forums a database of such changes is needed.
Currently no such database exist (to the authors knowledge), so the
user will need a way to specify his own.  Even if such a database did
exist, for the same reasons as lookups, the database might not be
complete.  A name change can be specified by having two entries for
the desired forum, adding the \texttt{OLDforum} tag to the entry
marking the name change.

\begin{figure}
  \centering
\begin{verbatim}
@PROCEEDINGS{forum_ncsc,
  title = "National Computer Security Conference"
}

@PROCEEDINGS{forum_nissc,
  title = "National Information Systems Security Conference",
  OLDforum = "ncsc_forum"
}
\end{verbatim}
  \caption{Configuring a name change of a forum}
  \label{fig:analyzing_configuration_name_change}
\end{figure}

The initial draft of the configuration is illustrated in
\figref{fig:analyzing_configuration_name_change}.  However, this
configuration ignores that the names in the actual \file{bib} may be
in their abbreviated form.  This way of configuring will work for most
journals, but proceedings is often, if not always, named according to
how many times a conference has been hold.  Thus in the example above
an entry would be named something like \texttt{Proceedings of the 20th
  national information systems security conference}.  Extracting the
names to strings is a part of solving the abbreviation of forum names.
Referencing the same strings instead will enable a parser to detect
the presence of the same string.  Thus if the \file{bib} looks like
\figref{fig:analyzing_configuration_name_change_bib_file_strings} then
the corresponding configuration will look like \figref{fig:analyzing_configuration_name_change_config_file_strings}.

\begin{figure}
  \centering
\begin{small}
\begin{verbatim}
% Re-usable strings
@STRING{PROCintro = "Proceedings of the"}
@STRING{nissc = "National Information Systems Security Conference"}

% Conferences
@STRING{nissc20 = PROCintro # "20th" # nissc}

% Proceedings
@INPROCEEDINGS{porras1997emerald,
  title = "EMERALD: Event monitoring enabling response to anomalous live disturbances",
  author = "Porras, Phillip A and Neumann, Peter G",
  booktitle = nissc20
}
\end{verbatim}
\end{small}
  \caption{\file{bib} using strings for conference names}
  \label{fig:analyzing_configuration_name_change_bib_file_strings}
\end{figure}

\begin{figure}
  \centering
\begin{verbatim}
@PROCEEDINGS{forum_nissc,
  title = nissc,
  OLDforum = "forum_ncsc"
}
\end{verbatim}
  \caption{\file{bib} using strings for conference names}
  \label{fig:analyzing_configuration_name_change_config_file_strings}
\end{figure}

The configuration of name changes should be using the most general
entry type available, such as \texttt{@ARTICLE}, \texttt{@PROCEEDINGS}
and \texttt{@BOOK}.  Either using rules in the name change checker to
correctly map the values from the general entry types to the specific
types, such as mapping \texttt{@PROCEEDINGS} to
\texttt{@INPROCEEDINGS} and detecting that the title tag in
\texttt{@PROCEEDINGS} correspond to booktitle in
\texttt{@INPROCEEDINGS}.

\remark{Oh my - that exploded rather quickly}

When specifying the de-facto standards, {\bibtex} entries should be
used.  When specifying a standard only deviations should be specified.
The configurations available are:

\begin{itemize}
\item \texttt{@required} for a tag we require to be present.
\item \texttt{@optional} for optional tags.
\item \texttt{@deny} for tags that are in the default configuration,
  that we want to reject
\item \texttt{@exludes=tag} for a tag that excludes the use of another
  tag, replacing \texttt{tag} with the name of another tag.  For
  example, if one allows both \texttt{ISSN} and \texttt{DOI} as tags,
  but want to ensure that only one of the tags is present, one would
  have the following: \texttt{ISSN = "@required @exludes=DOI"} and
  \texttt{DOI = "@required @excludes=ISSN"}.
\item \texttt{@includes=tag} for a tags where one of them is required
  and the other optional.  Usage is similar to \texttt{@excludes=tag}.
\end{itemize}

An example of a configuration of standards can be seen in
\figref{fig:analyzing_standards_config}.  This example sets article
entries to: reject \texttt{address} tags, that either \texttt{DOI} or
\texttt{ISSN} is present (but not both) and adds \texttt{url} as an
optional tag.  For books entries the example sets: that ISBN10 and/or
ISBN13 must be present.

\begin{figure}
  \centering
\begin{verbatim}
@article{standards_article,
  address = "@deny",
  DOI = "@required @excludes=ISSN",
  ISSN = "@required @excludes=DOI",
  url = "@optional"
}

@book{standards_book,
  ISBN10 = "@required @inclusive=ISBN13",
  ISBN13 = "@required @inclusive=ISBN10"
}
\end{verbatim}
  \caption{A snippet of the desired {\bibtex} based configuration for the correctness checker}
  \label{fig:analyzing_standards_config}
\end{figure}

Abbreviations of journal names can be configured by, introducing two
new tags: \texttt{abbreviated} and \texttt{fullname}, specifying the
abbreviated journal name and full journal name respectively.

\remark{Bah, proceedings are not as easy ><}

Inconsistent tags

For the specification of entry keys using a {\bibtex} string with the
predefined name \texttt{ENTRY\_KEY}.  Inside the string some way of
specifying the desired template for entry keys is needed.  Using a
template scheme, such as \texttt{\{tag\}} to match tags, is probably
the best solution.  This template system contradicts the desire to
keep the format close to {\bibtex}, but no better way has been found.
A template could look like: \texttt{\{author\}\{year\}\{title\}}.

However, this template is insufficient when considering how people
actually name their entries.  A lot of people use similar schemes, but
using the first author's last name, and a significant word from the
title.  Specifying these more detailed schemes will move the template
format further away from {\bibtex}, and unfortunately no better
approach has been for this either. \remark{some formatting needed here}

Fortunately, {\bibtex} strings comes to the rescue - at least
partially.  Having strings for the most common matches will ensure
that most users will never need see, nor even know about, the
underlying pattern matching system.  This will allow the user to use
concatenations to build the desired pattern.

\remark{Still need a good design for these strings}

\begin{figure}
  \centering
\begin{verbatim}
@STRING{ENTRY_KEY = author # year # "_" # title}
\end{verbatim}
  \caption{An example of a entry key pattern}
  \label{fig:analyzing_entry_key_pattern}
\end{figure}

\remark{Still need to specify consistency changes and if a tag is the
  desired consistency.}


\section{{\orangutan}}

\subsection{Introduction}

An attempt at implementing is {\orangutan}.


\subsection{Why {\orangutan} came to be}

The advent of {\bibtex} has been a game changer for bibliographic
references, which makes the issues in {\bibtex} undesirable.  There
are a lot of tools that provide partial solutions, so having an
augmenting tool would be an improvement.  The tool will be most
effective if it address first the correctness and lexical concerns
then the consistency concerns.  The attempt at making this tool has
been named \newdef{\orangutan}.


\subsection{What is {\orangutan}}

In the same spirit as {\bibtex}, {\orangutan} is designed to be a
simple software tool for improving bibliographic references.
{\orangutan} analyze and give suggestions for improvements to a
\file{bib}.


\subsection{How {\orangutan} is used in principle}

When analyzing {\bibtex} files using a two step solution is used
taking care of the correctness and lexical concerns first, then the
consistency concerns.  The two steps is used to ensure the best
possible conditions for handling consistency concerns.  The analyzing
tool then outputs the suggestions it has for improving the \file{bib}.

The tool use options, set by introducing a new de-facto standard.  The
options are for specifying if the language for the spell check and
that the entry is already considered to be correct.  To make the
options intuitive for the {\bibtex} user the options are designed to
match the {\bibtex} format.


\subsection{How {\orangutan} is used in practice}

The configuration format make use of that entry tags starting with
$OPT$ is a de-facto standard, for commenting a tag out.  This way of
configuring is chosen to prevent future clashes with standards.  A
simple way of specifying options is provided that feels natural for a
{\bibtex} user.  All the options are specified as an entry tag with
the name $OPTorangutan$.

In the current version there are three analyzing modules in use: a
spell checker, a correctness checker and an abbreviation checker.

The spell checker runs a spell check in the background using aspell.
Currently the spell checker is limited to only titles.  When spell
checking it use a configuration to specify the language, \eg,
$OPTorangutan = {@lang=DA}$.  Just as any other spell checker it marks
words that are misspelled and give a list of suggestions.

The correctness checker currently verifies the conformity with the
{\bibtex} specification and known de-facto standards.  Currently the
format for specifying entries is JSON, a {\bibtex} based format is
desired.  The format is relatively simple as seen in
\figref{fig:correctness_checker_json}.  The desired format for user
configuration is based on {\bibtex} such as can be seen in
\figref{fig:correctness_checker_bibtex}

\begin{figure}
  \centering
\begin{minted}{json}
{
  "book": {
    "author": {
      "required": true,
      "excludes": "editor"
    },
    "editor": {
      "required": true,
      "excludes": "author"
    },
[...]
\end{minted}
  \caption{A snippet of the JSON for configuring the correctness checker}
  \label{fig:correctness_checker_json}
\end{figure}

\begin{figure}
  \centering
\begin{verbatim}
@book{
  author = {@required @excludes=editor}
  editor = {@required @excludes=author}
[...]
\end{verbatim}
  \caption{A snippet of the desired {\bibtex} based configuration for the correctness checker}
  \label{fig:correctness_checker_bibtex}
\end{figure}

The abbreviation checker runs through journal names using a known list
of abbreviations.  The current version just suggest the full name
whenever an abbreviation is detected, these suggestions could be
heavily improved by suggesting strings.  The checking can be improved
by using know abbreviation standards or trying to detect abbreviations
that are unknown.


% Is there already fitting string for an journal name?
% Suggest strings for journal names, matching easier

\section{Summary and conclusions}

To summarize:

Having a tool that can analyze {\bibtex} files for issues is really
useful.  However, just like the lookup services can lure a user into
a false sense of security, so can \orangutan.  An analyzing tool will
not be a guarantee that there are no issues left.


\chapter{Organizing {\bibtex} files}
\label{ch:organizing}
\section{Introduction}

\section{Contents}

\section{Summary and conclusion}

\chapter{Conclusion and perspectives}
\label{ch:conclusion}
\rfquote{All that matters on the chessboard is good moves.}{Bobby
Fischer}

\noindent
Let us recapitulate: we have first described {\bibtex} -- both how to
use it in principle and how it is used in practice
(Chapter~\ref{ch:about}); we have then listed a range of practical
issues {\bibtex} users encounter
(Chapter~\ref{ch:problem-description}), we have proposed an approach
to hand\-ling them (Chapter~\ref{ch:approach}), and we have reviewed
how they are tackled in related work (Chapter~\ref{ch:related}); we
have then presented an analysis of {\bibtex} files that detects these
issues (Chapter~\ref{ch:analyzing}), and we have described how to
solve them by organizing {\bibtex} files using the results from this
analysis (Chapter~\ref{ch:organizing}).  We have implemented a part of
this analysis in a prototype, {\orangutan}.

{\bibtex}, despite its wide use, is far from a perfect tool, and
{\bibtex} user's face challenges that range far and wide.  Many tools
surround {\bibtex} and so do a lot of alternatives, some of which do
provide partial solutions to some of the issues one faces.  However,
as analyzed in Chapter~\ref{ch:related}, these tools are far from
sufficient.  For most of the challenges a {\bibtex} user faces, it is
possible to provide analysis tools that detect potential issues.

Such analyses cannot be perfect though, since any set of rules yields
false positives or relies on assumptions that are ill-founded since
{\bibtex} is not formally specified: while the soundness of the
specification of {\bibtex} is not questioned, its completeness is
unknown.  Analyses of {\bibtex} files therefore need to have
configurations, through de-facto standards, to detect and ignore false
positives.  In most cases, these analyses can also provide suggestions
for improvements, such as the suggestions from a spell checker.  These
suggestions can then be used to organize one's \file{bib}, either by
correcting the issues that have been detected or by adding de-facto
tags to prevent false positives.
%%%% Rewrite from here
% [and here report the status of orangutan at this time of writing,
% and its future as you see it today]

Just like chess players who strive to always improve the quality of
their moves, one can wish to improve the quality of one's {\bibtex}
files.  We have designed {\orangutan} as a proof of concept for
suggesting improvements.  It is our hope that this proof of concept
can contribute to improving the general quality of {\bibtex} files
and, consequently, can improve the precision of bibliographic
references in documents as well as save time for their authors and
their readers.

\printbibliography{}
\end{document}
