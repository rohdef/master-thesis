\rfquote{Now that we have found love\\what are we gonna do\\with
  it?}{Stevie Wonder}

\section{Introduction}

For organizing {\bibtex} files, the following is covered: how to
organize in principle \chapref{sec:organizing_principle}, how to
organize in practice \chapref{sec:organizing_practice}, what the
choices for the prototype {\orangutan} are
\chapref{sec:organizing_orangutan_what}, how {\orangutan} handles it
in principle \chapref{sec:organizing_orangutan_how_principle}, and how
{\orangutan} handles it in practice
\chapref{sec:organizing_orangutan_how_practice}


\section{Organizing in general}

\subsection{In principle}
\label{sec:organizing_principle}

Having found the issues that are possible, a way to react to them is
needed.  Correcting is either done by changing the entries to a state,
where the analysis cannot find any more issues, or by using the
configurations to tell the analysis that the issues are false
positives.

There are two basic ways of reacting to the issues, one automatically
correcting them and the other is to present the issues and have the
user decide what to do.  As pointed out in the problem descriptions,
it is hard, if not impossible, to automatically correct, since it can
lead to incorrect changes and thus introduce a new structural issue.
Thus, the organizing is done by presenting the issues, letting the user
decide on the action to take.

When presenting the user with the issues, there are again two
approaches: letting one do a range of choices to handle the issues,
finally outputting a corrected \file{bib}, or giving a list of issues
and suggestions for corrections.  The last one requires the user to
manually edit his \file{bib}, and if the issues are listed with line
numbers, they should be listed backwards, so the lines will be correct
throughout the correction.

After all the concerns have been addressed, a pretty printing utility
will be a nice final touch, ensuring a consistent structure inside one's
\file{bib} and having the same indentations and formatting.

\subsection{In practice}
\label{sec:organizing_practice}

When multiple suggestions are available, all the alternatives should
be given to the user.  All suggestions should include a way to find
the violating entry, \ie, either the entry key or a location in the
file.  For all the detected issues, it should suggest using the
relevant configurations from
Section~\ref{sec:analyzing_configuration}, with exception of the ones
to disable entire checks, \ie, \texttt{@LEXICALLYOK},
\texttt{@CONSISTENCYOK} and \texttt{@OK}.  The suggestions that the
tool gives should be as follows:

\begin{itemize}
\item For duplicate entries, it should suggest: merging the entries,
  updating the content of one of the entries, and adding a de-facto
  configuration to mark the duplication as deliberate.  In the case
  where a merge is not possible, or if the online lookup contains
  additional information, it should also suggest the result from an
  online lookup.  For duplicated content, it should suggest using
  \texttt{crossref} when relevant and otherwise changing the content
  to make use of strings.

\item For general spelling errors, it should inform about the
  misspelled word and show the suggestions from the spellchecker.
  Furthermore, if online lookups are used to prevent false positives,
  the result from those should also be presented.

\item For spelling errors in names, it should only show the names that
  cannot be verified using databases.  Then it should show the name or
  names that gives challenges, and if possible, suggestions from the
  name databases.

\item Initials should be presented to the user, and if possible,
  suggestions from the name databases.

\item Whenever online lookups are used, if there are suggestions, they
  should be presented to the user, along with the option to disable
  the online lookups.  This presentation should also be the case when
  the lookup is a part of another tool.

\item For name changes, it should suggest usage of a de-facto standard
  to highlight a previous name of the forum, following the rule of
  suggesting only names actively used in the document, and suggesting
  the latest name, prior to the current entry.  If the entry is from
  the earliest instance of the forum in the file, no suggestions
  should be made.

\item De-facto standards and specification conformance should result
  in information about the deviations from standards, \ie, if a field
  is missing, this should be marked with the suggestion of adding it,
  if a field is not specified or has been marked with deny, it should
  suggest removal.  In the case of exclusive and inclusive fields
  where both are missing, it should suggest an alternative action.
  For the case of exclusive fields where both are present, it should
  suggest removing one of the violating fields.  Furthermore, it
  should show if the validation of the values fail, and if possible,
  where in the check they fail.  It should also show the suggestions
  from any online lookups to provide possible corrections.

\item For abbreviations, it should suggest changing to usage of
  strings, if the values are written as text.  It should suggest
  changing the content of the strings to either full names or
  abbreviated forms depending on the desired format for the user.

\item When detecting possible strings that ended up in text it should
  suggest either changing the entire text to a string, if it is the
  entire text that matches the name of the string and otherwise
  splitting the text, concatenating the matched string with the rest
  of the text.  If a text has a string representation, it should
  suggest removing the text and replacing it with the string,
  concatenating it with the rest of the text, if relevant.

\item For inconsistent tags, it should suggest the shortest path to
  consistency, whether it is by removing tags or adding tags to the
  entries.

\item When entry keys are inconsistent with the specified pattern, it
  should suggest replacing the entry key with the value that matches
  the pattern.
\end{itemize}

By following these instructions, it should at all times be possible to
reach a point where the analysis does not show any more issues: a
fixed point.


\section{\orangutan}

\subsection{What}
\label{sec:organizing_orangutan_what}

{\orangutan} is designed as a framework rather than an end-user
application.  The output it gives is the suggestions coming from the
analysis.  For instance, a tool can list the suggestions from
{\orangutan} in a backward order, for manual editing.


\subsection{How in principle}
\label{sec:organizing_orangutan_how_principle}

The output is given in JSON, which most programming frameworks
support.  The output can be in a trimmed version showing only the
detected issues, or in a full version containing all the entries.
Thus, the output could be used for listing only the issues, or for
printing an entire corrected {\bibtex} file after choosing the desired
solutions.


\subsection{How in practice}
\label{sec:organizing_orangutan_how_practice}

{\orangutan} gives back a JSON string containing at least the entries
with detected issues.  If configured to do so, it keeps the entries
without issues.  For the entries in the output that have issues, an
object is attached with the name \texttt{orangutan}, detailing the
specific issues and suggestions.

All detected issues will be put in an object named $orangutan$ on the
internal object for the entry.  Having the corrections along the
object will allow printing the corrected entry, once an action has
been decided on.  Effectively, the $orangutan$ object functions as a
map or dictionary, having an item for each entry tag with detected
issues, and an object containing the details of the issues.

The spell checker will add an object named $spelling$ to the object
with the issues.  It contains the details from the spellcheck: how
many words it checked $wordCount$, how many misspellings it found
$misspellingCount$ and most importantly a list named $misspellings$
for the details from the spell checker.  Each item in $misspellings$
is an object consisting of: the misspelled word in $word$, the
position of the misspelled word in $position$ and a list of
suggestions inside $alternatives$.  An example of a spelling error can
be seen in \figref{fig:orgazing_misspelling_output}.

\begin{figure}
  \centering
\begin{minted}{json}
{
  "title": {
    "spelling": {
      "wordCount": 3,
      "misspellingCount": 1,
      "misspellings": [
        [{
          "type": "misspelling",
          "word": "Algoritm",
          "position": 10,
          "alternatives": [
            "Algorithm",
            "Alacrity",
            "Ageratum",
            "Alacrity's" ]
        }]
      ]
    }
  }
}
\end{minted}
\caption{An example of Orangutan output on a spelling error}
\label{fig:orgazing_misspelling_output}
\end{figure}

\remark{Now does it actually check for unknown entry types???}

When checking for correctness, it performs a conformance check that
marks entries according to the {\bibtex} specification and de-facto
standards.  This check is named conformance check.  The output
specifies when an entry type or a tag is in violation with the rules
specified.  The conformance checker adds an object for the tags with
conformance errors to the $orangutan$ object.  The object for the tag
will then have an object named $specificationConformance$ containing a
description of the issues and a corresponding code.
\figref{fig:orgazing_nonconformity} displays an example of a violation
of an exclusive rule.

\begin{figure}
  \centering
\begin{minted}{json}
{
  "author": {
    "specificationConformance": {
      "description": "[author] and [editor] " +
                     "cannot be in the same entry",
      "code": 3,
      "field": "editor"
    }
  },
  "editor": {
    "specificationConformance": {
      "description": "[editor] and [author] " +
                     "cannot be in the same entry",
      "code": 3,
      "field": "author"
    }
  }
}
\end{minted}
\caption{An example of Orangutan output on a conformity error where
  the exclusive use has been violated}
\label{fig:orgazing_nonconformity}
\end{figure}

The current version just suggests the full name whenever an
abbreviation is detected.  It builds a structure with suggestions for
full names when an abbreviation is
detected. \figref{fig:orgazing_abbreviation} displays an {\orangutan}
output for an abbreviation.

\begin{figure}
  \centering
\begin{minted}{json}
{
  "journal": {
    "abbreviations": {
      "abbreviation": [
        "am. j. potato res."
      ],
      "suggestions": {
        "am. j. potato res.":
            [ "American Journal of Potato Research" ]
      }
    }
  }
}
\end{minted}
\caption{An example of Orangutan output on an abbreviation.}
\label{fig:orgazing_abbreviation}
\end{figure}

A trimmed version of the output for an entry is displayed in
\figref{fig:orgazing_complete}, which contains enough information to
re-print the entry.  A tool using {\orangutan} can thus take this
output, update the representation of the entry according to one's
choices, and output a complete suggestion for a corrected entry.  It
should be noted that the tool does not suggest the configurations for
analysis.  The suggestion of configurations can arguably be the
responsibility of a framework, such as {\orangutan}, and of the
frontend.

\begin{figure}
  \centering
\begin{minted}{json}
{
  "type": "other",
  "citationKeyUnmodified": "jelly_baby",
  "citationKey": "jelly_baby",
  "entryType": "article",
  "entryTags": {
    "author": [{
      "type": "text",
      "delimiter": "{",
      "part": "Rincewind the Wizzard"
    }],
    "title": [{
      "type": "text",
      "delimiter": "{",
      "part": "Interesting Times with Potatoes and Jelly Beans"
    }],
    "journal": [{
      "type": "text",
      "delimiter": "{",
      "part": "Am. J. Potato Res."
    }],
    [...]
    "year": [{
      "type": "text",
      "delimiter": "{",
      "part": "1994"
    }]
  },
  "orangutan": {
    "journal": {
      "abbreviations": {
        "abbreviation": ["am. j. potato res."],
        "suggestions": {
          "am. j. potato res.":
              ["American Journal of Potato Research"]
        }
      }
    }
  }
}
\end{minted}
\caption{A trimmed example of an entry from {\orangutan} }
\label{fig:orgazing_complete}
\end{figure}

\section{Current status of the prototype}

At the time of writing, the prototype can do a spell check, detect
abbreviations and check the conformity to a given specification.

The language for the spell checker can be set using the de-facto
\texttt{OPTanalyze}.  Spell checking is only done for title tags.  The
spell checker used is $aspell$, which mush be in the path.  There is
no option to add new words to the spell checker via {\orangutan}
itself.  However, the spell checker is configured to use the folder
$aspell/$ for local word lists, which has to be relative to the
directory, from which the implementing tool is run.  The format for
the word lists is specified in the Aspell manual under \emph{Format of
  the Personal and Replacement Dictionaries}~\cite{format2016_aspell}.

For checking the specification and conformity a simple set of rules
are defined.  The default rules are the {\bibtex} specification
according to \emph{{\bibtex} Entry and Field
  Types}~\cite{bibtex_roberts}, and de-facto standards for
\texttt{crossref}, \texttt{OPT*}, \texttt{OLDforum}, \texttt{DOI}, and
\texttt{ISSN}.  Thus is allows tags commented out by prefixing the tag
name with \texttt{OPT}.  The specifications are currently set in the
directory of {\orangutan}.

Abbreviations are detected using a list of know abbreviations, taken
from JabRef.  It can only detect abbreviated forms and suggest full
form, not vice versa.  The abbreviations are stored in a JSON file,
mapping from the abbreviated form to the full form. The abbreviation
configuration is currently located in the directory of {\orangutan}.

Inside the {\orangutan} source, the work for consistency checks has
been started, but is far from a working state.  There is no analysis
for any of the other issues.

The {\orangutan} framework is located at
\url{https://github.com/rohdef/orangutan}, and a simple command line
implementation is located at
\url{https://github.com/rohdef/orangutan-demo}.  Note that nodejs and
npm is required to run the tools, and that running \texttt{npm
  install} is a prerequisite to running the demo or using the
framework.


\section{Summary and conclusion}

Using the results from the analysis of a \file{bib}, sensible
suggestions for a course of action can be provided.  For most results,
a straightforward solution can be provided, such as the correct
spelling of a word.  The output from the prototype {\orangutan} shows
a proof of concept for providing output that one can use to resolve
the issues from the analysis.
