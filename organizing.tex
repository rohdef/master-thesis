\section{Introduction}

\section{Organizing in general}

After all the concerns has been addressed a pretty printing utility
will a nice final touch, ensuring a consistent structure inside ones
\file{bib} and having the same indentations and formatting for text.

\subsubsection{First step: Lex}
\begin{itemize}
\item Abbrev: should be handled by refactoring journal names into
  strings.
\end{itemize}

\subsubsection{Sec step: Consist}
\begin{itemize}
\item duplicate.  The tool should suggest merging when duplicates are
  found.
\item name change. When a forum name has changed it should suggest
  usage of a de-facto standard to highlight this.
\end{itemize}

\subsubsection{Configuration}

If relevant, refer to analyzing configuration and say the same
argument goes for organizing.


Having found the issues that are possible, a way to react to them is
needed.  Correcting is either done by changing the entries to a state,
where the analysis cannot find any more issues, or by using the
configurations to tell the analysis that the issues are false
positives.

There are two basic ways of reacting to the issues, one
automatically correcting them and the other is to present the issues
and have one decide what to do.  As pointed out in the problem
descriptions, it is hard, if not impossible, to automatically correct,
since it can lead to incorrect changes and thus introduce a new
structural issue.  Thus the organizing is done by presenting the
issues, letting the user decide on the action to take.

When presenting the user with the issues there are again two
approaches: letting one do a range of choices to handle the issues
finally outputting a corrected \file{bib}, or giving a list of issues
and suggestions for corrections.  The last one requires the user to
manually edit his \file{bib}, and if the issues are listed with line
numbers, they should be listed backwards, so the lines will be correct
throughout the correction.


\section{\orangutan}

\subsection{What}
 and give suggestions for improvements to a

\subsection{How in principle}

The analyzing tool then outputs the suggestions it has for improving
the \file{bib}.

\subsection{How in practice}

Just as any other spell checker it marks words that are misspelled and
give a list of suggestions.

suggesting strings

The current version just suggest the full name whenever an
abbreviation is detected,

\section{Summary and conclusion}

To summarize:

An analyzing tool will not be a guarantee that there are no issues
left.
