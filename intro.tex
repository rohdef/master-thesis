\remark{A first try at capturing the thesis in a single sentence}

This dissertation shows a tool for automated detection lexical
and inconsistency concerns i Bib{\TeX} references.


\section{The issue}
\remark{I'm not overly satisfied with the formulation, but it's a
  start, then I'll have to tighten it and make it sharper later}

In academic and research there are a lot of documents being written,
when those documents are meant for others to read, such as in a
journal, the authors want that work to not only be correct but also to
look good.  Having inconsistencies, spellling errors and alike can
make otherwise correct scientific work look sloppy.  This includes the
references, which are almost certian to be evaluated by researchers
interested in the work.  However unlike the content of the document
itself references are often not read as thoroughly through and even
when it is mistakes may not even be seen since the current reference
style may hide them through abbreviations and styling.  These errors
can then be carried on to other documents as reference libraries are
often reused and inconsistencies get harder to spot by hand as the
library gets larger over the years.

When gathering references people will usually, extract it from a PDF
file (or similar), copy and paste it from an online resource or write
them by hand.  Thus these types of errors can arise from bad
extractions from the PDF, by them being present in the PDF, mistakes
in the online resources or simply mistyping.  This leads to two ways
of approaching the issues: either try to prevent them from ever
happening or handling them when they have been entered into a library.

Most of the sources of input are intangible to control, as it is hard
(if not impossible) to prevent spelling errors in PDF files you get,
it is impossible to prevent people from making typos or being
uncertain if they have been using abbreviations and even if the online
resources did never have those errors, not everyone would use them.
Thus it is more interesting to focus on handling these when they have
entered the library.

As a lot of \remark{can we find how many} scientific work are done in
{\LaTeX}, and thus Bib{\TeX} has become a de facto standard for
references.  Having some tool to detect these errors would be a help
to authors of scientific documents.  Such a tool:

\remark{Keep up to date with what could be done (if something turns
  up).  Remember to argue what has been done and use rest for further
  work}

\begin{itemize}
\item Lexical concerns
  \begin{itemize}
  \item Titles with misspellings
  \item Names of people with misspellings
  \item Names of locations with misspellings
  \end{itemize}
\item Inconsistent entries
  \begin{itemize}
  \item Names with different formatting: different
    abbreviations of the name and name in full
  \item Entries that are not uniform, for instance some mention the
    publisher others do not.
  \end{itemize}
\item In addition
  \begin{itemize}
  \item Detection of never revisions
  \item Journal version or reports of conference papers
  \end{itemize}
\end{itemize}

\remark{Should probably write a bit more about the parts, to highlight
  various aspects of them etc.}

\section{Bib{\TeX} format}
Bib{\TeX} is a commonly used format for managing lists of references in
academic circles \remark{Now now, can we prove this?}.  The format is
designed for use with {\LaTeX}, but plugins for other formats do
exist. \autocite{bibtex_resource}

The format itself is fairly simple, at it's main level we got
$@STING$, $@PREAMPLE$, $@COMMENT$ and $entries$.  $@STRING$ is for
abbreviations that can be used later in the Bib{\TeX}, $@PREAMPLE$ is for
defining how to format the text and $@COMMENT$ is for comments and
$entries$ are the actual entries \autocite{bibtex_resource}.

Since $@PREAMPLE$ and $@COMMENT$ doesn't affect the contents they
won't be considered further in this document. \remark{Double check
  preample, recall reading somewhere that it's just LaTeX macros.}

Tags is case insensitive. If the contents needs to be enclosed in
either \{ and \} or quotes (depending on the persons taste), numbers
can be written without them and @sting abbreviations has to be without
them. Concatenation is done using \#.\autocite{bibtex_resource}

Designed by Oren Patashnik and Leslie Lamport in 1985

\remark{(Note to self:) should probably have half an eye on things
  like: `G{\"o}del', when working with it}

\remark{(Note to self:) people may use booktitle where title is
  appropriate edition: The edition of a book---for example,
  ``Second''. This should be an ordinal, and should have the first
  letter capitalized, as shown here; the standard styles convert to
  lower case when necessary. \autocite{bibtex_description}}

\remark{http://tug2000.tug.org/TUGboat/Articles/tb24-1/patashnik.pdf}

\section{BibTeX variants and alternatives}
\remark{So far mostly notes to self}
There is a lot of both variants and alternatives to BibTeX (out there in the wild :P ).

\begin{itemize}
\item mlbibtex - multi language http://citeseerx.ist.psu.edu/viewdoc/summary?doi=10.1.1.107.3886
\item biblatex
\item biber
\item bibtexml - YUCK, what a horrible idea! http://www.xj2z.net/UpLoadFile/cms/Ebook/305/ts305045.PDF
\item application managed (Mendeley, EndNote, RefMan, RefWorks, Zotero)
\end{itemize}

\section{Lookup services}
\remark{So far mostly notes to self}
DOI

%%% Local Variables:
%%% mode: latex
%%% TeX-master: "thesis"
%%% End:
