\remark{Trying to capturing the thesis in a single sentence}

This dissertation shows a tool for automated detection lexical
and inconsistency concerns i Bib{\TeX} references.


\section{The issue}
\remark{I'm still not entirely satisfied, but not it at least have a
  flow from what the issue is to what the focus is etc.}

When writing research documents, for instance an article for a
journal, it is desired to have such documents to be properly
presentable to reflect the professionalism of the work.  Having
inconsistencies, spellling errors and alike can make otherwise correct
scientific work look sloppy.  This includes the references, which are
almost certain to be evaluated by researchers interested in the work.
However unlike the content of the document itself references are often
not proofread as thoroughly and the errors may not be visible since
the current reference style may hide them through abbreviations and
styling.  Then errors can be carried on to other documents, as
reference libraries are often reused, furthermore inconsistencies get
harder to spot by hand, as the library grows over the years.

When gathering references people will usually, extract them from PDF
files (or similar), get them from an online resource or write them by
hand.  Thus the errors can arise from erroneous extractions from the
files used, by them being present in the first place, mistakes in the
online resources or simply mistyping.  This leads to two ways of
approaching the issues: either try to prevent them from ever happening
or handling them when they have been entered into a library.

Most of the sources of input are intangible to control, as it is hard
(if not impossible) to prevent misspelling in external documents, it
is impossible to prevent people from making typos or being
inconsistent in their use of abbreviations.  For online resources the
errors can also be present and even if this could be corrected, it is
unlikely to get everyone to use them all the time and it would not
correct already existing libraries.  It is therefore more
interesting to focus on handling these when they have entered the
library.

As a lot of \remark{can we find how many, and is it actually the
  mostly used format outside natural sciences?} scientific work are
done in {\LaTeX}, Bib{\TeX} has become a de facto standard for
references.  Having some tool to detect these errors would help
authors of scientific documents to keep their documents presentable.
Some of the possible issues such tool can focus are:

\remark{This list is both for what is going to be done and for further
  work}
\begin{itemize}
\item Lexical concerns
  \begin{itemize}
  \item Titles with misspellings
  \item Names of people with misspellings
  \item Names of locations with misspellings
  \end{itemize}
\item Inconsistent entries
  \begin{itemize}
  \item Names with different formatting: different abbreviations of
    the name and name in full
  \item Entries that are not uniform, for instance some mention the
    publisher others do not.
  \end{itemize}
\item In addition
  \begin{itemize}
  \item Only putting the initials of authors increases ambiguity
    \remark{kind of want this at the ambiguity somehow, as it's a
      good point}
  \item Detection of never revisions
  \item Journal version or reports of conference papers
  \item Missing fields
  \end{itemize}
\end{itemize}

For this dissertation the goal is to review the field of bibliographic
references, observe the shortcomings in published references and to
design a tool to analyze bibliographies for occurrences of the
shortcomings identified.

%\remark{People who request similar
%http://stackoverflow.com/questions/13630584/the-best-method-for-handling-bibtex-files
%http://stackoverflow.com/questions/32838020/awk-how-to-clean-bibtex-files
%http://cgi.di.uoa.gr/~charnik/oss/bibtool/
%http://www.latex-community.org/forum/viewtopic.php?f=50&t=12043
%http://tex.stackexchange.com/questions/76420/cleaning-up-a-bib-file
%http://tex.stackexchange.com/questions/174509/is-there-a-tool-service-that-can-enrich-a-bibtex-database

%Tools, mayhaps relevant
%http://cgi.di.uoa.gr/~charnik/oss/bibtool/
%http://www.development.root-1.de/Bibcut.php
%https://code.google.com/p/bibtex-check/
%}

\section{Bib{\TeX} format}
Bib{\TeX} is a commonly used format for managing lists of references in
academic circles \remark{Now now, can we prove this?}.  The format is
designed for use with {\LaTeX}, but plugins for other formats do
exist. \autocite{bibtex_resource}

The format itself is fairly simple, at it's main level we got
$@STING$, $@PREAMPLE$, $@COMMENT$ and $entries$.  $@STRING$ is for
abbreviations that can be used later in the Bib{\TeX}, $@PREAMPLE$ is for
defining how to format the text and $@COMMENT$ is for comments and
$entries$ are the actual entries \autocite{bibtex_resource}.

Since $@PREAMPLE$ and $@COMMENT$ doesn't affect the contents they
won't be considered further in this document. \remark{Double check
  preample, recall reading somewhere that it's just {\LaTeX} macros.}

Tags is case insensitive. If the contents needs to be enclosed in
either \{ and \} or quotes (depending on the persons taste), numbers
can be written without them and @sting abbreviations has to be without
them. Concatenation is done using \#.\autocite{bibtex_resource}

Designed by Oren Patashnik and Leslie Lamport in 1985

\remark{(Note to self:) should probably have half an eye on things
  like: `G{\"o}del', when working with it}

\remark{(Note to self:) people may use booktitle where title is
  appropriate edition: The edition of a book---for example,
  ``Second''. This should be an ordinal, and should have the first
  letter capitalized, as shown here; the standard styles convert to
  lower case when necessary. \autocite{bibtex_description}}

\remark{http://tug2000.tug.org/TUGboat/Articles/tb24-1/patashnik.pdf}

\section{BibTeX variants and alternatives}
\remark{So far mostly notes to self}
There is a lot of both variants and alternatives to BibTeX (out there in the wild :P ).

\begin{itemize}
\item mlbibtex - multi language http://citeseerx.ist.psu.edu/viewdoc/summary?doi=10.1.1.107.3886
\item biblatex
\item biber
\item bibtexml - YUCK, what a horrible idea! http://www.xj2z.net/UpLoadFile/cms/Ebook/305/ts305045.PDF
\item application managed (Mendeley, EndNote, RefMan, RefWorks, Zotero)
\end{itemize}

\section{Lookup services}
\remark{So far mostly notes to self}
DOI

%%% Local Variables:
%%% mode: latex
%%% TeX-master: "thesis"
%%% End:
