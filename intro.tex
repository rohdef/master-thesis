\section{BibTeX format}
BibTeX is a commonly used format for managing lists of references in
academic circles \remark{Now now, can we prove this?}.  The format is
designed for use with {\LaTeX}, but plugins for other formats do
exist. \autocite{bibtex_resource}

The format itself is fairly simple, at it's main level we got
$@STING$, $@PREAMPLE$, $@COMMENT$ and $entries$.  $@STRING$ is for
abbreviations that can be used later in the BibTeX, $@PREAMPLE$ is for
defining how to format the text and $@COMMENT$ is for comments and
$entries$ are the actual entries \autocite{bibtex_resource}.

Since $@PREAMPLE$ and $@COMMENT$ doesn't affect the contents they
won't be considered further in this document. \remark{Double check
  preample, recall reading somewhere that it's just LaTeX macros.}

Tags is case insensitive. If the contents needs to be enclosed in
either \{ and \} or quotes (depending on the persons taste), numbers
can be written without them and @sting abbreviations has to be without
them. Concatenation is done using \#.\autocite{bibtex_resource}

Designed by Oren Patashnik and Leslie Lamport in 1985

\remark{(Note to self:) should probably have half an eye on things
  like: `G{\"o}del', when working with it}

\remark{(Note to self:) people may use booktitle where title is
  appropriate edition: The edition of a book---for example,
  ``Second''. This should be an ordinal, and should have the first
  letter capitalized, as shown here; the standard styles convert to
  lower case when necessary. \autocite{bibtex_description}}

\remark{http://tutex.tug.org/pracjourn/2006-4/fenn/fenn.pdf}

\remark{http://tug2000.tug.org/TUGboat/Articles/tb24-1/patashnik.pdf}

\section{BibTeX variants and alternatives}
\remark{So far mostly notes to self}
There is a lot of both variants and alternatives to BibTeX (out there in the wild :P ).

\begin{itemize}
\item mlbibtex - multi language http://citeseerx.ist.psu.edu/viewdoc/summary?doi=10.1.1.107.3886
\item biblatex
\item biber
\item bibtexml - YUCK, what a horrible idea! http://www.xj2z.net/UpLoadFile/cms/Ebook/305/ts305045.PDF
\item application managed (Mendeley, EndNote, RefMan, RefWorks, Zotero)
\end{itemize}

\section{Lookup services}
\remark{So far mostly notes to self}
DOI

%%% Local Variables:
%%% mode: latex
%%% TeX-master: "thesis"
%%% End:
