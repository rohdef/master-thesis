\remark{Trying to capturing the thesis in a single sentence}

This dissertation documents a method for automated detection of
lexical and inconsistency concerns in {\bibtex} references.

\subsection{What \remark{working title}}

{\bibtex} has changed the landscape for scientific writing and eased a
lot of peoples lives, however it is not without issues.  The
challenges range widely:

Having files that does not conform with specifications is a
double-edged sword.  On one hand it gives a powerful tool for useful
de facto standards.  On the other hand it leads to inconsistencies,
missing fields and fields that might not even make sense.

The use of abbreviations can lead to non-uniform entries.  Having
abbreviations can provide challenges for tools that use {\bibtex}
files, for instance when searching and doing cross referencing.

Misspellings and mistypings is a common issue in {\bibtex}, the
information about an entry should match the information from the given
source.  Having misspellings can be an issue when people want to look
up a references.


\subsection{Why/how \remark{working title}}

The sources of the issues can be considered from the perspective of
structural vs conjunctural, which shows that there is no uniform way
of considering the reason for the challenges.  Using this approach has
uncovered a lot of the underlying reasons.

By design {\bibtex} does not prevent using the format however the user
desires, for good or bad.  This results in both a lot of
inconsistencies and missing information, but also in useful de facto
standards.

When adding entries to {\bibtex} there are various sources of errors.
Manual entry easily leads to spelling errors and as the file grows
large inconsistencies.  Entering by hand using copy and paste can
cause existing issues to be repeated.  Using online resources, one may
expect the entries to be correct, but is not given that they are.  The
entries will be no better than the source used.  Using the automatic
tools for extraction of meta data, from articles, may be prone to
further errors.

Some tools for editing Bib{\TeX}-files do not necessarily follow the
specification, and does not assist in ensuring consistency.  A lot of
the tools can convert between abbreviated and full form in the files,
essentially motivating to store abbreviations rather than full names.
Some even suggest tags that does not belong in the entry type.



An example of this can be seen from the {\bibtex} output in figure
\ref{fig:bibtex_out}, this is from a file that does not conform to the
specifications and contains unspecified tags, {\bibtex} just warns
about the missing tags and that the book has both author and editor,
but completely ignores all the unspecified tags.  That {\bibtex}
ignores unspecified tags also enable useful de facto standards such as
prefixing a tag with OPT to comment it out and adding $crossref$ to
the files for easy referencing between entries, but in turn provides
no way of ensuring consistency or that the entries are even
correct. \remark{rewrite}

\begin{figure}[h]
  \centering
  \begin{verbatim}
This is BibTeX, Version 0.99d (TeX Live 2015/Debian)
The top-level auxiliary file: doc.aux
The style file: plain.bst
Database file #1: mybib.bib
Warning--can't use both author and editor fields in book
Warning--empty publisher in book
Warning--empty year in book
Warning--empty journal in Nobody06
Warning--empty year in Nobody06
\end{verbatim}
  \caption{Example of error output from Bib{\TeX}}
\label{fig:bibtex_out}
\end{figure}

\subsection{What can be done - general \remark{working title}}
Addressing the human factor in it is one possibile way of solving
these issues.  ``Simply'' motivating people to do things right.  Alas
people are not machines and thus this approach will be impossible, for
most people the interest is not the tools they use but what they use
them for, so the interest in Bib{\TeX} will for most people being in
ensuring that their articles contain the relevant references.

Using the digital approach is another option.  This can be done in
various ways.

One option would be to change Bib{\TeX}, so it enforce the
specifications.  This would take care of some of the inconsistencies,
ensure that the specification is followed.  However this does not
address abbreviations, misspellings or the overall inconsistency.
Also this would require changes in the specifications to accommodate
the de facto standards that have arisen over time.  Furthermore this
would be immensely invasive as it would cause existing Bib{\TeX} files
not to work and it would impose the tool on the users with
requirements they may not desire nor care about.

Along the lines of changing Bib{\TeX} there are also a few attempts to
make a new and improved bibliography systems.  These mostly address
either annoyance at the format (eg. preferring XML) or to provide more
modern options (support for URLs and UTF-8).  This could be taken one
step further introducing the improved checks.  The disadvantage of
this is that it requires people to change the format, some tools like
Bib{\LaTeX} is designed, so it's compatible with Bib{\TeX} allowing a
smooth transition, this compatibility would however be lost by that
design.  Also like the previous suggestion it would only take care of
a limited amount of the issues.

In stead of changing or replacing Bib{\TeX} a supportive tool is
another option.  Such a tool can relatively easy check if the
Bib{\TeX} specification is followed, and supportive rules for de facto
standards be provided.  This would allow analyzing the provided
Bib{\TeX}-files, trying to detect inconsistencies, doing online
lookups etc.

\subsection{What can be done - analyzing tool \remark{working title}}

\remark{remember the new revisions desire}

%\remark{People who request similar
%http://stackoverflow.com/questions/13630584/the-best-method-for-handling-bibtex-files
%http://stackoverflow.com/questions/32838020/awk-how-to-clean-bibtex-files
%http://www.latex-community.org/forum/viewtopic.php?f=50&t=12043
%http://tex.stackexchange.com/questions/76420/cleaning-up-a-bib-file
%http://tex.stackexchange.com/questions/174509/is-there-a-tool-service-that-can-enrich-a-bibtex-database
%<http://tex.stackexchange.com/questions/128989/how-can-one-validate-a-bib-file
%http://tex.stackexchange.com/questions/173621/how-to-validate-check-a-biblatex-bib-file
%http://stackoverflow.com/questions/13630584/the-best-method-for-handling-bibtex-files
%}

\section{Lookup services}
\remark{So far mostly notes to self}
DOI

%%% Local Variables:
%%% mode: latex
%%% TeX-master: "thesis"
%%% End:
