\remark{Trying to capturing the thesis in a single sentence}

This dissertation shows a tool for automated detection of lexical and
inconsistency concerns in Bib{\TeX} references.


\section{The issue}
\remark{I'm still not entirely satisfied, but not it at least have a
  flow from what the issue is to what the focus is etc.}

When writing research documents, for instance an article for a
journal, it is desired to have such documents to be presented
properly.  This is out of respect for the work referred to, for the
ones reading the work and it is a requirement from journals.  Having
inconsistencies, spellling errors and alike can make otherwise correct
scientific work look sloppy.  This includes the references, which are
almost certain to be evaluated by researchers interested in the work.

% However unlike the content of the document itself references are often
% not proofread as thoroughly and the errors may not be visible since
% the current reference style may hide them through abbreviations and
% styling.  Then errors can be carried on to other documents, as
% reference libraries are often reused, furthermore inconsistencies get
% harder to spot by hand, as the library grows over the years.

\subsection{What \remark{working title}}
Misspellings and mistypings is a common issue in Bib{\TeX}, the
information about an entry should match the information from the given
source.  Having misspellings can be an issue when people want to look
up a references.

The use of abbreviations leads to non-uniform entries, and prevents
the capability of Bib{\TeX} formatting to use shortenings according to
various specifications.  Having abbreviations can provide challenges
for tools that use Bib{\TeX} files, for instance when searching and
doing cross referencing.  Having entries that use abbreaviations will
provide challenges when you need to output references with full names.

A lot of the entries are incorrect, having missing fields or fields
that are not specified.  This can be an issue when one output style
needs different fields than other styles, or if needing to look up
details later on.  Also people use tags that are not specified for an
entry type, for instance having a $volume$ for $InProceedings$.  Some
of these can have practical use, even though they are not specified.

Lastly there are a huge range of inconsistencies, for instance
references from the same journal may not be formatted uniformly.
\remark{reasoning}
\remark{Add that it's relevant to journals!}

\subsection{Why/how \remark{working title}}
The sources for this range of errors are wide.  First off by design
Bib{\TeX} does not prevent this use of the format, it just warns about
missing tags and bad usage, furthermore it completely ignores
unspecified tags.  An example of this can be seen from the Bib{\TeX}
output in figure \ref{fig:bibtex_out}, this is from a file that does
not conform to the specifications and contains unspecified fields,
Bib{\TeX} just warns about the missing fields and that the book has
both author and editor, but completely ignores all the unspecified
tag.  That Bib{\TeX} ignores unspecified tags also enable useful de
facto standards such as prefixing a tag with OPT to comment it out and
adding $crossref$ to the files for easy referencing between entries,
but in turn provides no way of ensuring consistency or that the
entries are even correct.

\begin{figure}[h]
  \centering
  \begin{verbatim}
This is BibTeX, Version 0.99d (TeX Live 2015/Debian)
The top-level auxiliary file: doc.aux
The style file: plain.bst
Database file #1: mybib.bib
Warning--can't use both author and editor fields in book
Warning--empty publisher in book
Warning--empty year in book
Warning--empty journal in Nobody06
Warning--empty year in Nobody06
\end{verbatim}
  \caption{Example of error output from Bib{\TeX}}
  \label{fig:bibtex_out}
\end{figure}

Having this relaxed policy causes a lot of sloppiness with regard to
the use of the format.  A lot people who are using a tool do not care
about following the rules strictly, they do not care whether a journal
name is abbreviated or not, as long as it is there. What they want is
something that works and do not need to bother too much about, so they
can focus on what they care about: their research.

When adding entries to Bib{\TeX} there are various sources of errors.
The simplest way is manual entry, which easily leads to spelling
errors and as the file grows large inconsistencies.  Furthermore when
entering by hand doing copy and paste can cause the mistakes to be
repeated over, if the user is not careful.  Using online resources,
one may be expecting the entries to be correct enough for their use,
but is no way guaranteed so, the entries will be no better than the
source used.  Also automatic tools can extract meta data from articles
and may be prone to further errors.

Another challenge is that some tools for editing Bib{\TeX}-files do
not necessarily follow the specification, and does not assist in
ensuring consistency.  A lot of the tools will converting between
abbreviated and full form, essentially motivating to store
abbreviations in the files rather than having the output formatting
handling how to display information.  Some even suggest tags that does
not belong in the entry type, this has its pros and cons, as remarked
earlier for most users the primary focus is something that works.

\subsection{What can be done - general \remark{working title}}
Addressing the human factor in it is one possibile way of solving
these issues.  ``Simply'' motivating people to do things right.  Alas
people are not machines and thus this approach will be impossible, for
most people the interest is not the tools they use but what they use
them for, so the interest in Bib{\TeX} will for most people being in
ensuring that their articles contain the relevant references.

Using the digital approach is another option.  This can be done in
various ways.

One option would be to change Bib{\TeX}, so it enforce the
specifications.  This would take care of some of the inconsistencies,
ensure that the specification is followed.  However this does not
address abbreviations, misspellings or the overall inconsistency.
Also this would require changes in the specifications to accommodate
the de facto standards that have arisen over time.  Furthermore this
would be immensely invasive as it would cause existing Bib{\TeX} files
not to work and it would impose the tool on the users with
requirements they may not desire nor care about.

Along the lines of changing Bib{\TeX} there are also a few attempts to
make a new and improved bibliography systems.  These mostly address
either annoyance at the format (eg. preferring XML) or to provide more
modern options (support for URLs and UTF-8).  This could be taken one
step further introducing the improved checks.  The disadvantage of
this is that it requires people to chance the format, some tools like
Bib{\LaTeX} is designed, so it's compatible with Bib{\TeX} allowing a
smooth transition, this compatibility would however be lost by that
design.  Also like the previous suggestion it would only take care of
a limited amount of the issues.

In stead of changing or replacing Bib{\TeX} a supportive tool is
another option.  Such a tool can relatively easy check if the
Bib{\TeX} specification is followed, and supportive rules for de facto
standards be provided.  This would allow analyzing the provided
Bib{\TeX}-files, trying to detect inconsistencies, doing online
lookups etc.

\subsection{What can be done - analyzing tool \remark{working title}}



\subsection{From old}


This leads to two ways of approaching the issues: either try to
prevent them from ever happening or handling them when they have been
entered into a library.

Most of the sources of input are intangible to control, as it is hard
(if not impossible) to prevent misspelling in external documents, it
is impossible to prevent people from making typos or being
inconsistent in their use of abbreviations.  For online resources the
errors can also be present and even if this could be corrected, it is
unlikely to get everyone to use them all the time and it would not
correct already existing libraries.  It is therefore more
interesting to focus on handling these when they have entered the
library.

As a lot of \remark{can we find how many, and is it actually the
  mostly used format outside natural sciences?} scientific work are
done in {\LaTeX}, Bib{\TeX} has become a de facto standard for
references.  Having some tool to detect these errors would help
authors of scientific documents to keep their documents presentable.
Some of the possible issues such tool can focus are:

\remark{This list is both for what is going to be done and for further
  work}
\begin{itemize}
\item Lexical concerns
  \begin{itemize}
  \item Titles with misspellings
  \item Names of people with misspellings
  \item Names of locations with misspellings
  \end{itemize}
\item Inconsistent entries
  \begin{itemize}
  \item Names with different formatting: different abbreviations of
    the name and name in full
  \item Entries that are not uniform, for instance some mention the
    publisher others do not.
  \end{itemize}
\item In addition
  \begin{itemize}
  \item Only putting the initials of authors increases ambiguity
    \remark{kind of want this at the ambiguity somehow, as it's a
      good point}
  \item Detection of never revisions
  \item Journal version or reports of conference papers
  \item Missing fields
  \item Detection special characters?
  \end{itemize}
\end{itemize}

For this dissertation the goal is to review the field of bibliographic
references, observe the shortcomings in published references and to
design a tool to analyze bibliographies for occurrences of the
shortcomings identified.


%\remark{People who request similar
%http://stackoverflow.com/questions/13630584/the-best-method-for-handling-bibtex-files
%http://stackoverflow.com/questions/32838020/awk-how-to-clean-bibtex-files
%http://www.latex-community.org/forum/viewtopic.php?f=50&t=12043
%http://tex.stackexchange.com/questions/76420/cleaning-up-a-bib-file
%http://tex.stackexchange.com/questions/174509/is-there-a-tool-service-that-can-enrich-a-bibtex-database
%<http://tex.stackexchange.com/questions/128989/how-can-one-validate-a-bib-file
%http://tex.stackexchange.com/questions/173621/how-to-validate-check-a-biblatex-bib-file
%http://stackoverflow.com/questions/13630584/the-best-method-for-handling-bibtex-files
%}

\section{Bib{\TeX} format}
Bib{\TeX} is a commonly used format for managing lists of references in
academic circles \remark{Now now, can we prove this?}.  The format is
designed for use with {\LaTeX}, but plugins for other formats do
exist. \autocite{bibtex_resource}

The format itself is fairly simple, at it's main level we got
$@STING$, $@PREAMPLE$, $@COMMENT$ and $entries$.  $@STRING$ is for
abbreviations that can be used later in the Bib{\TeX}, $@PREAMPLE$ is for
defining how to format the text and $@COMMENT$ is for comments and
$entries$ are the actual entries \autocite{bibtex_resource}.

Since $@PREAMPLE$ and $@COMMENT$ doesn't affect the contents they
won't be considered further in this document. \remark{Double check
  preample, recall reading somewhere that it's just {\LaTeX} macros.}

Tags is case insensitive. If the contents needs to be enclosed in
either \{ and \} or quotes (depending on the persons taste), numbers
can be written without them and @sting abbreviations has to be without
them. Concatenation is done using \#.\autocite{bibtex_resource}

Designed by Oren Patashnik and Leslie Lamport in 1985

\remark{(Note to self:) should probably have half an eye on things
  like: `G{\"o}del', when working with it}

\remark{(Note to self:) people may use booktitle where title is
  appropriate edition: The edition of a book---for example,
  ``Second''. This should be an ordinal, and should have the first
  letter capitalized, as shown here; the standard styles convert to
  lower case when necessary. \autocite{bibtex_description}}

\remark{http://tug2000.tug.org/TUGboat/Articles/tb24-1/patashnik.pdf}

\section{BibTeX variants and alternatives}
\remark{So far mostly notes to self}
There is a lot of both variants and alternatives to BibTeX (out there in the wild :P ).

\begin{itemize}
\item mlbibtex - multi language http://citeseerx.ist.psu.edu/viewdoc/summary?doi=10.1.1.107.3886
\item biblatex
\item biber
\item bibtexml - YUCK, what a horrible idea! http://www.xj2z.net/UpLoadFile/cms/Ebook/305/ts305045.PDF
\item application managed (Mendeley, EndNote, RefMan, RefWorks, Zotero)
\end{itemize}

\section{Lookup services}
\remark{So far mostly notes to self}
DOI

%%% Local Variables:
%%% mode: latex
%%% TeX-master: "thesis"
%%% End:
