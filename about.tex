The goal of this chapter is to introduce {\bibtex},
\chapref{sec:why_bibtex_came_to_be},
\chapref{sec:principles_of_bibtex}, \chapref{sec:practice_of_bibtex}
and \chapref{sec:how_bibtex_is_used_today}.

% In this chapter {\bibtex} will be surveyed, with a focus on
% understanding {\bibtex} and why it is so influential.  The chapter
% is organized as follows: \textit{\nameref{sec:principles_of_bibtex}}
% contains a cursory description of {\bibtex} showing it in
% principality, and \textit{\nameref{sec:practice_of_bibtex}} contains
% a description of the {\bibtex}-format and how it works.


\section{Why {\bibtex} came to be}
\label{sec:why_bibtex_came_to_be}

The advent of {\TeX} has been a game changer for scientific writing,
witness the number of articles written in {\TeX} (or derivatives such
as {\LaTeX}).  For instance, on arXiv.org, almost all articles have
{\LaTeX} sources.  Apart from being grateful to Donald Knuth for
providing a robust system for articles, {\TeX} has also given the
basis for Oren Patashnik and Leslie Lamport to create {\bibtex} for
managing ones bibliography.

Before the wake of {\bibtex} references was managed entirely by hand
and a lot of manual labor went in to management of references.  For
instance, nearly half of Mary-Claire van Leunen's book, \textit{A
  Handbook for Scholars}~\cite{leunen1992_handbook}, is dedicated to
citing, managing and writing references, 150 pages out of 335 pages
(excluding the index).  A significant fraction of Umberto Eco's book,
\textit{How to Write a Thesis}, is also dedicated to managing and
citing references and ensuring a proper bibliography.  All this manual
labor has almost been made obsolete by {\bibtex}.


\section{What is {\bibtex}}
\label{sec:principles_of_bibtex}

In the same spirit as {\LaTeX}, {\bibtex} is a simple way of managing
references in scientific writing, using an ASCII-file format for the
reference database.  Inside this file, the various information about a
reference are specified, such as the author, title, year and what kind
of medium was used.

When compiling a document {\bibtex} takes care of the citing styles,
ensures the formatting in the reference list, and ensures that only
relevant entries are included.  The references are labeled
consistently in the document and the references to allow fast look ups
for the reader, this can be seen in
\figref{fig:bibtex_example_alpha}.

Today {\bibtex} has also enabled huge online resources with
references, automated extraction tools, sharing of references and easy
version control.  The {\bibtex} format is originally designed for use
with {\LaTeX}, but it has plugins for other formats as
well~\cite{bibtex_resource}.

\begin{figure}[ht]
  \centering
  \includegraphics[width=0.8\textwidth]{bibtex-styles-ex-acm}
  \caption{{\bibtex} output using ACM formatting which uses numbers to
    identify entries}
\label{fig:bibtex_example_acm}
\end{figure}

\section{How {\bibtex} works}
\label{sec:practice_of_bibtex}

Inside a {\bibtex}-file the format itself is fairly simple. At the
main level we got $@STRING$, $@PREAMBLE$, $@COMMENT$ and $entries$.
$@STRING$ is for abbreviations that can be used later in the
{\bibtex}, $@PREAMPLE$ is for defining how to format the text and
$@COMMENT$ is for comments and $entries$.  The $entries$ correspond to
the different medium types, such as $@ARTICLE$, $@BOOK$ and
$PROCEEDINGS$, which in turn contains the relevant tags for the given
entry.  For each entry type, there are a specification of tags
relevant to the given medium where some of them are mandatory. For
instance for an $@ARTICLE$ the tags $author$, $title$, $year$ and
$journal$ is mandatory and supplementary information such as the
$pages$ for the article and $volume$ of the journal can be added.

Tags and entries are case insensitive. The contents needs to be
enclosed in either \{ and \} or quotes and numbers can be written
without. $@STRING$ abbreviations has to be without quotes and curly
brackets.  Concatenation of $@STRING$ and/or literal content is done
using \#~\cite{bibtex_resource}.  {\bibtex} is designed to ignore
unknown entries and tags, so it allows additional information.  An
example {\bibtex}-file can be seen in \figref{fig:bibtex_example}

\begin{figure}[ht]
  \centering
  \begin{small}
\begin{verbatim}
@String{JFP = "Journal of Functional Programming"}
@String{ox_uni_press = "Oxford University Press"}

@Article{Abadi-al:JFP91,
  author =      "Mart\'{\i}n Abadi and Luca Cardelli
                 and Pierre-Louis Curien
                 and Jean-Jacques L\'evy",
  title =       "Explicit substitutions",
  journal =     JFP,
  year =        1991,
  volume =      1,
  number =      4,
  pages =	"375-416",
  note =	"A preliminary version was presented at the Seventeenth
                 Annual {ACM} Symposium on Principles
                 of Programming Languages
                 (POPL 1990)"
}

@InBook{leunen1992_handbook,
  author =       {Mary-{C}laire van Leunen},
  title =        "A Handbook for Scholars",
  publisher =    ox_uni_press,
  year =         {1992},
  pages =        {9--45,154--268}
}
\end{verbatim}
  \end{small}
  \caption{{\bibtex} example}
\label{fig:bibtex_example}
\end{figure}

When citing inside {\LaTeX} the desired key from the {\bibtex} is used
inside the ${\backslash}cite$, for example if one has a reference
named $some\_article$ then the reference is used by writing:
${\backslash}cite\{some\_article\}$.  To link the document and the
bibliography together the command ${\backslash}bibliography$ is used
together with a parameter for the {\bibtex}-file, for example:
${\backslash}bibliography\{mybib\}$.

To select the desired style the command
${\backslash}bibliographystyle$ is used, for example
${\backslash}bibliographystyle\{alpha\}$.  The style in turn controls
the formatting and how the references are labeled, as can be seen on
\figref{fig:bibtex_example_acm} and \figref{fig:bibtex_example_alpha}
where the labels are different, the abbreviation of author names are
different and some of the visual formatting is different.

\begin{figure}[ht]
  \centering
  \includegraphics[width=0.8\textwidth]{bibtex-styles-ex-alpha}
  \caption{{\bibtex} output using alpha formatting which uses author
    names and year to identify entries}
\label{fig:bibtex_example_alpha}
\end{figure}

To build a {\LaTeX}-document with {\bibtex}-references in it, one will
first run the $latex$-command (or one of the derivatives) to produce
(among other things) an aux-file.  The aux-file contains auxiliary
information from the {\LaTeX}-compiler.  Then run the $bibtex$-command
which uses the aux-file to find the entries in use and to give them
labels according to the reference style in use.  The output from
{\bibtex} is a bbl-file with the formatted references, which is then
used by subsequent runs by {\LaTeX}, so the document will have labels
a the appropriate places and an bibliography in accordance with the
labels.

\section{How {\bibtex} is used today}
\label{sec:how_bibtex_is_used_today}


\section{Summary and conclusion}


%\remark{(Note to self:) should probably have half an eye on things
%  like: `G{\"o}del', when working with it}
%
%\remark{(Note to self:) people may use booktitle where title is
%  appropriate edition: The edition of a book---for example,
%  ``Second''. This should be an ordinal, and should have the first
%  letter capitalized, as shown here; the standard styles convert to
%  lower case when necessary~\cite{bibtex_description}.}
%
%\remark{http://tug2000.tug.org/TUGboat/Articles/tb24-1/patashnik.pdf}f