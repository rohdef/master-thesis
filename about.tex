The goal of this chapter is to introduce the importance of {\bibtex}
followed by how it works

The advent of {\TeX} has been a game changer for scientific writing,
witness the number of articles written in {\TeX} (or derivatives such
as {\LaTeX}).  For instance, on arXiv.org, almost all articles have
{\LaTeX} sources.  Apart from being grateful to Donald Knuth for
providing a robust system for articles, {\TeX} has also given the
basis for Oren Patashnik and Leslie Lamport to create {\bibtex} for
bibliography managements.

In the same spirit as {\LaTeX}, {\bibtex} is a simple way of managing
references in scientific writing, using an ASCII file format for the
reference database.  Inside this file, the various information about a
reference are specified, such as the author, title, year, what kind of
medium was used and, depending on the medium, relevant information to
identify the source, for instance for an article there is a journal.
The {\bibtex} format is designed for use with {\LaTeX} and it has
plugins for other formats as well~\cite{bibtex_resource}.

The references are then simply used by in {\LaTeX} by citing a name
given in the {\bibtex}, for instance if one has a reference named
$some\_article$ then the reference is used by writing:
${\backslash}cite\{some\_article\}$.  When the {\bibtex} tool is run
it recognizes the entries and give them unique and uniform labels,
such as a number as seen in Figure~\ref{fig:bibtex_example_acm} or
author initials followed by a year as seen in
Figure~\ref{fig:bibtex_example_alpha}.  When {\bibtex} then prints the
references it uses the labels to allow a reader to quickly identify
the correct source, as can also be seen in the figures.

\begin{figure}[ht]
  \centering
  \includegraphics[width=0.8\textwidth]{bibtex-styles-ex-acm}
  \caption{{\bibtex} output using ACM formatting which uses numbers to
    identify entries}
\label{fig:bibtex_example_acm}
\end{figure}

\begin{figure}[ht]
  \centering
  \includegraphics[width=0.8\textwidth]{bibtex-styles-ex-alpha}
  \caption{{\bibtex} output using alpha formatting which uses author
    names and year to identify entries}
\label{fig:bibtex_example_alpha}
\end{figure}

Inside {\bibtex} the format itself is fairly simple, at the main level
we got $@STRING$, $@PREAMPLE$, $@COMMENT$ and $entries$.  $@STRING$ is
for abbreviations that can be used later in the {\bibtex}, $@PREAMPLE$
is for defining how to format the text and $@COMMENT$ is for comments
and $entries$ are the actual entries with the tags specifying the
content.  Tags and entries are case insensitive. The contents needs to
be enclosed in either \{ and \} or quotes, numbers can be written
without and $@string$ abbreviations has to be without quotes and curly
brackets. Concatenation of $@string$ and/or literal content is done
using \#\cite{bibtex_resource}.  {\bibtex} is designed to ignore
unknown entries and tags, so it allows additional information.  An
example of {\bibtex} can be seen in figure-\ref{fig:bibtex_example}

\begin{figure}[ht]
  \centering
  \begin{small}
\begin{verbatim}
@String{JFP = "Journal of Functional Programming"}
@String{ox_uni_press = "Oxford University Press"}

@Article{Abadi-al:JFP91,
  author =      "Mart\'{\i}n Abadi and Luca Cardelli
                 and Pierre-Louis Curien
                 and Jean-Jacques L\'evy",
  title =       "Explicit substitutions",
  journal =     JFP,
  year =        1991,
  volume =      1,
  number =      4,
  pages =	"375-416",
  note =	"A preliminary version was presented at the Seventeenth
                 Annual {ACM} Symposium on Principles
                 of Programming Languages
                 (POPL 1990)"
}

@InBook{leunen1992_handbook,
  author =       {Mary-{C}laire van Leunen},
  title =        "A Handbook for Scholars",
  publisher =    ox_uni_press,
  year =         {1992},
  pages =        {9--45,154--268}
}
\end{verbatim}
  \end{small}
  \caption{{\bibtex} example}
\label{fig:bibtex_example}
\end{figure}

When a user have a {\LaTeX} document with {\bibtex} references in it,
the user first run the $latex$ command (or one of the derivatives) to
produce (among other things) an aux file, this file contains auxiliary
information from the {\LaTeX} compiler.  When the $bibtex$ command is
executed, it uses the aux file to find the entries are used and to
give them labels according to the reference style in use.  The output
from {\bibtex} is a bbl-file with the formatted references, which is
then used by subsequent runs by {\LaTeX}, so the document will have
labels a the appropriate places and an bibliography in accordance
with the labels.

The combination of these two tools has changed the landscape of
scientific publishing.  Before the wake of {\bibtex} references was
managed entirely by hand and a lot of manual labor was spent on
managing references.  For instance, nearly half of Mary-Claire van
Leunen's book, A Handbook for Scholars~\cite{leunen1992_handbook}, is
dedicated to how to cite, manage and write references, 150 pages out
of 335 pages (excluding the index).  A significant fraction of Umberto
Eco's book, How to Write a Thesis, is also dedicated to how to manage
and cite references and ensuring a proper bibliography.  All this
manual labor has almost been made obsolete by {\bibtex}, which takes
care of the citing styles, ensures a correct formatting in the
reference list, and provides a way of having complete database of
references, but only printing the ones used in an article.  Also today
{\bibtex} has enabled huge online resources with references that can
be used for looking up references, automated extraction tools, sharing
of references and easy version control.


%\remark{(Note to self:) should probably have half an eye on things
%  like: `G{\"o}del', when working with it}
%
%\remark{(Note to self:) people may use booktitle where title is
%  appropriate edition: The edition of a book---for example,
%  ``Second''. This should be an ordinal, and should have the first
%  letter capitalized, as shown here; the standard styles convert to
%  lower case when necessary. \cite{bibtex_description}}
%
%\remark{http://tug2000.tug.org/TUGboat/Articles/tb24-1/patashnik.pdf}f