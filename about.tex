\section{Introduction}
The goal of this chapter is to introduce {\bibtex}: the course of
events leading to {\bibtex} \chapref{sec:why_bibtex_came_to_be}, what
{\bibtex} is \chapref{sec:principles_of_bibtex}, how {\bibtex} is used
in principle \chapref{sec:practice_of_bibtex}, and how {\bibtex} is
used in practice \chapref{sec:how_bibtex_is_used_today}.


\section{Why {\bibtex} came to be}
\label{sec:why_bibtex_came_to_be}

The advent of {\TeX} has been a game changer for scientific writing,
witness the number of articles written in {\TeX} (or derivatives such
as {\LaTeX}).  For instance, on arXiv.org, nearly all articles are
formatted with \LaTeX.  Apart from being grateful to Donald Knuth (the
creator of \TeX) for providing such a robust system for scientific
articles, writers can also be thankful that {\TeX} has given the basis
for Oren Patashnik and Leslie Lamport to create {\bibtex} for managing
scientific bibliographies.

Before the wake of {\bibtex}, bibliographic references were managed
entirely by hand and required a lot of labor.  For instance, nearly
half of Mary-Claire van Leunen's book, \textit{A Handbook for
  Scholars}~\cite{leunen1992_handbook}, is dedicated to citing,
managing, and writing references, \ie, 150 pages out of 335 pages
(excluding the index).  Likewise, a significant fraction of Umberto
Eco's book, \textit{How to Write a Thesis}~\cite{eco1985_thesis}, is
also dedicated to managing and citing references and to ensuring a
proper bibliography.  All this manual labor has almost been made
obsolete by {\bibtex}.

Even more, {\bibtex} entries are now readily available online, and so
the practical problem of bibliographic references is now solved, in
principle.

\section{What is {\bibtex}}
\label{sec:principles_of_bibtex}

In the same spirit as {\LaTeX}, {\bibtex} is a simple software tool
for managing bibliographic references in scientific writing, using an
ASCII file to specify theses references.  Inside this ASCII file, the
components of each reference are specified, such as the author, title,
year and what kind of medium (\eg, a book or article) was used.  This
file will be referred to as \newdef{the {\bibtex} file} or
\newdef{\file{bib}}

Depending on the forum, there are differences in: how references
should be cited, how they should be written, and the order in which
the references should be listed.  Each publisher has a mandatory
setup.  The specific set of rules a publisher has is referred to as a
\newdef{bibliography style} or \newdef{citing style}.

When processing a document, {\bibtex} cites according to the selected
bibliography style, ensures the formatting and the order in the
reference list, and ensures that only relevant entries are included.
The references are labeled consistently in the document according to
the citing style, the labels are used when a reference is cited and
these labels allow the reader to quickly find the reference in the
bibliography, as illustrated in \figref{fig:bibtex_example_alpha}.

Today, {\bibtex} has also enabled huge online resources with
references, automated extraction tools, sharing of references and easy
version control.  The {\bibtex} format is originally designed for use
with {\LaTeX}, but it has plugins for other formats as
well~\cite{bibtex_resource}.

\begin{figure}
  \centering
  \includegraphics[width=0.8\textwidth]{bibtex-styles-ex-acm}
  \caption{{\bibtex} output using ACM citing style which uses numbers to
    index entries.  Source: ShareLaTeX~\cite{sharelatex2016_styles}}
\label{fig:bibtex_example_acm}
\end{figure}

\section{How {\bibtex} is used in principle}
\label{sec:practice_of_bibtex}

\subsection{Macro-use}

A {\bibtex} file consists of entries each corresponding to a
bibliographic reference, such as an article or a book.  Each entry in
turn contains meta information about the reference, through tags that
specify the kind of meta information such as the author or title.
Also, quite commonly, a file will contain a lot of shortenings for
text fragments that are reused.

To select the desired style the command
${\backslash}bibliographystyle$ is used, for example
${\backslash}bibliographystyle\{alpha\}$.  The style in turn controls
the formatting and how the references are labeled, as can be seen on
\figref{fig:bibtex_example_acm} and \figref{fig:bibtex_example_alpha}
where the labels are different, along with the abbreviation of author
names and some of the visual formatting.

To build a {\LaTeX}-document with {\bibtex} references in it, one
first runs the $latex$ command (or one of the derivatives) to produce
(among other things) an \file{aux}.  The \file{aux} contains auxiliary
information from the {\LaTeX}-compiler.  Then run the $bibtex$ command
which uses the \file{aux} to find the entries in use and to give them
labels according to the reference style.  The output from {\bibtex} is
a \file{bbl} with the formatted references, which is then used in
subsequent runs of {\LaTeX}, so the document will have labels at the
appropriate places and a bibliography in accordance with the labels.


\subsection{Micro-use}
\label{sec:about_micro_use}

Inside a {\bibtex} file, the format itself is fairly simple. At the
main level, we have $@STRING$, $@PREAMBLE$, $@COMMENT$ and $entries$.
$@STRING$ is for shortenings that can be used later in the {\bibtex}
file, $@PREAMBLE$ is for defining how to format the text and
$@COMMENT$ is for comments and $entries$.  The $entries$ correspond to
the different medium types, such as $@ARTICLE$, $@BOOK$ and
$@PROCEEDINGS$, which in turn contain the relevant tags for the given
entry.  Each entry consists of an identifying key and a set of tags.
The identifying key will be called \newdef{entry key} to avoid
ambiguity with the tag $key$.

For each entry type, there are a specification of tags relevant to the
given medium where some of them are mandatory. For instance for an
$@ARTICLE$ the tags $author$, $title$, $year$ and $journal$ are
mandatory and supplementary information such as the $pages$ for the
article and $volume$ of the journal can be added.  For ease of use,
{\bibtex} provides predefined strings for the months: $jan$, $feb$,
$mar$ and so on.  The rules for the tags are quite simple, a tag can
either be \newdef{required} or \newdef{optional}.  Furthermore there
are cases where two tags are required, but which an ``or'' between
them, \eg, in the specification of required fields for $@BOOK$ it says
``author or editor'', meaning either author or editor, but not both.
These ``or'' rules will be called \newdef{exclusive}.  The last kind
of rule is ``and/or'', which specifies that at least one of the tags
has to be present and having both is allowed, this rule will be called
\newdef{inclusive}.

Tags and entries are case insensitive. The literal content
(\newdef{text}) needs to be enclosed in either \{ and \} or quotes and
numbers can be written without.  $@STRING$ shortenings has to be
without quotes and curly brackets.  Concatenation of $@STRING$ and/or
text is done using \#~\cite{bibtex_resource}.  {\bibtex} is designed
to ignore unknown entries and tags, so it allows additional
information.  An example {\bibtex} file can be seen in
\figref{fig:bibtex_example}

\begin{figure}
  \centering
  \begin{small}
\begin{verbatim}
@String{JFP = "Journal of Functional Programming"}
@String{OUP = "Oxford University Press"}

@Article{abadi1991_substitutions
  author =      "Mart\'{\i}n Abadi and Luca Cardelli
                 and Pierre-Louis Curien
                 and Jean-Jacques L\'evy",
  title =       "Explicit substitutions",
  journal =     JFP,
  year =        1991,
  volume =      1,
  number =      4,
  pages =	"375--416",
  note =	"A preliminary version was presented at the Seventeenth
                 Annual {ACM} Symposium on Principles
                 of Programming Languages
                 (POPL 1990)"
}

@InBook{leunen1992_handbook,
  author =       "Mary-{C}laire van Leunen",
  title =        "A Handbook for Scholars",
  publisher =    OUP,
  year =         1992,
  pages =        "9--45,154--268"
}
\end{verbatim}
  \end{small}
  \caption{{\bibtex} example}
\label{fig:bibtex_example}
\end{figure}

When citing inside a {\LaTeX} file, the desired entry key from the
{\bibtex} is used inside the ${\backslash}cite$, for example if one
has a reference named $some\_article$ then the reference is used by
writing: ${\backslash}cite\{some\_article\}$.  To link the document
and the bibliography together, the command ${\backslash}bibliography$
is used together with a parameter: the name the {\bibtex} file, for
example: ${\backslash}bibliography\{mybib\}$.

\begin{figure}
  \centering
  \includegraphics[width=0.8\textwidth]{bibtex-styles-ex-alpha}
  \caption{{\bibtex} output using alpha citation style which uses
    author names and year to index entries.  Source:
    ShareLaTeX~\cite{sharelatex2016_styles}}
\label{fig:bibtex_example_alpha}
\end{figure}


\section{How {\bibtex} is used in practice}
\label{sec:how_bibtex_is_used_today}

To this day, {\bibtex} is routinely used by researchers.  Observe that
almost all online resources for finding references provide {\bibtex}
entries.  Many online databases are widely used to lookup entries,
which has also given rise to a variety of ways to identify articles,
such as: arXiv numbers, DOI and ISSN.

Since {\bibtex} is capable of printing only the references used in a
{\LaTeX}-document, most people start using {\bibtex} as a database
having a complete file with the references they have used throughout
time.  A lot of people also find it practical to use their {\bibtex}
file as a way to keep track of what they read.

That {\bibtex} ignores unknown tags is used for de-facto standards, to
add additional information and to comment out tags by prefixing them
with $OPT$.  De-facto standards are so widely in use that they are
commonly used in the results from search engines for bibliographic
references and there are citing styles that make use of the de-facto
standards.


\section{Summary and conclusions}
\label{sec:about_conclusion}


{\bibtex} is a product of the advent of {\TeX} and the need for
managing bibliographic references.  The inside of a {\bibtex} file is
simple and intuitive, dividing the entries into types corresponding
to the medium it represents and having tags relevant to that medium.
{\bibtex} has become widely used and given rise to useful de-facto
standards and tools to assist with bibliographic references.

Like {\TeX}, {\bibtex} has been a game changer, and is something for
every scientific author to be happy about.  So has {\bibtex} solved the
problem of bibliographic references? It would seem so, had it not been
for Murphy's Law, as detailed in the next chapter.


%\remark{(Note to self:) should probably have half an eye on things
%  like: `G{\"o}del', when working with it}
%
%\remark{(Note to self:) people may use booktitle where title is
%  appropriate edition: The edition of a book---for example,
%  ``Second''. This should be an ordinal, and should have the first
%  letter capitalized, as shown here; the standard styles convert to
%  lower case when necessary~\cite{bibtex_description}.}
%
%\remark{http://tug2000.tug.org/TUGboat/Articles/tb24-1/patashnik.pdf}f