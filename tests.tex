For testing a test environment have been set up to test both the
effectiveness of external programs and the progress of the system.

When running spell checks multiple tests has been provided.  Doing a
simple spell check is fairly easy in itself, but things as domain
specific terms can provide some challenges.  The spell checker will
have to be taught those to avoid false positives.  Also some
references may be in various languages and not just one single
language.  Take for instance the French book title ``Le Petit
Prince'', which with an English dictionary will get a false positive
on ``Petit''.  Furthermore as Bib{\TeX} does not support unicode
encoding\cite{bibtex_encoding} all the issues of using 8-bit encodings
such as ISO-8859-1 will potentially arise.  To remedy this Bib{\TeX}
supports a range of escapes including the ones possible by using
{\LaTeX} escapes\cite{bibtex_resource}.  This furthermore enables
trouble as a spellchecker is normally not adjusted to deal with this,
so a book title such as ``Mennesker I En Mistbænk'' will not only
provide the language challenge, but also provide an encoding challenge
from ``Mistbænk'' being written as ``Mistb{\{}{\backslash}ae{\}}nk''
to prevent ambiguity.  To provide challenges for how well these issues
are handled part of the spelling test data is in various languages,
including both correctly typed (to detect false positives) and
misspelled entries.

In the abbreviation tests there is both tests for known and unknown
abbreviation, in the form of standard abbreviations of journal names
compared with custom non-standard definitions.  Furthermore for both
types of abbreviations there are tests where the full form is already
present both in text and in the form of a string.  Also for known
journal abbreviations there are tests to override that standard.


%%% Local Variables:
%%% mode: latex
%%% TeX-master: "thesis"
%%% End:
