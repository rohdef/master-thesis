\rfquote{Testing shows the presence,\newline
  not the absence of bugs.}{Edsger W. Dijkstra}


\section{Introduction}
The goal of this chapter is to show how {\bibtex} files are analyzed.


\section{What should be done about {\bibtex}}
\subsection{In principle}

Due to the practical issues in changing/replacing {\bibtex} and to
ensure separation of concerns an analyzing tool should be an
augmenting tool.

To analyze {\bibtex} one would need to parse a \file{bib} into a
suitable representation.  This representation would need to parse
{\bibtex} entries and strings at a minimum.  If one intents to pretty
print the result after resolving all issues, the parsing also needs
the preambles.  Comments are technically optional, but should be kept
too.

The easiest approach is a two step parse: first taking care of the
lexical and correctness concerns, then the consistency concerns.  By
taking care of the lexical and correctness concerns first the optimal
conditions for taking care of the consistency concerns is made.

The term database i here use for both local and online databases,
since it does not matter if one use a local copy, if it is up to date.

\subsubsection{First step: Lexical and correctness}
\begin{itemize}
\item Spelling errors in general, should be detected either by online
  lookups or a spell checker.  A combination of the two could be
  powerful, since it gives a potential indicator of false positives.
  Using a spell checker requires a way to configure the language for
  the individual entries.

\item Spelling errors in names should use databases with
  names for detection.

\item Initials can be detected by finding cases with a single letter,
  perhaps followed by a punctuation.  For suggestions databases with
  names will be useful.  Handling multiple letters combined can be
  done by looking for relatively few (probably up to 3) letters that
  are all capitalized.

\item Online lookups, will be impossible to detect for certain,
  because the intended search will be unknown.  The detection of the
  other issues can give an indicator of a bad lookup.  Since online
  lookups is a likely way to handle the databases needed for this
  tool, the quality of a lookup is a concern.  The most trusted online
  database available should be used whenever possible.  A system doing
  lookups in multiple databases will give further indications of
  potential issues.  The results should always be confirmed by the
  user, to prevent erroneous data.

\item Conformity to de-facto standards and the {\bibtex} specification
  requires a set of rules specifying: required tags, optional tags,
  exclusive tags, and inclusive tags.

\item Detecting invalid values should be done where there are clear
  rules for a correct value can be specified: such as ISSN, year, and
  month.  Furthermore, values that can be verified using a database
  should also be verified.

\item To detect journal names, in abbreviated form, or in full form,
  is done using a database of know journal names and their
  abbreviations.  It should use a database to de-abbreviate or
  abbreviate.  Furthermore this can be refined by detecting unknown
  abbreviations.

\item To handle {\bibtex} strings that end up as part of the text, the
  text should be compared with the strings in the \file{bib}.
\end{itemize}


\subsubsection{Second step: Consistency}
\begin{itemize}
\item To detect duplicate entries, using unique identifiers is the
  best way, then based on identical entries.  Otherwise potential
  duplicates is detected by a specification of similarity: having the
  title and authors as primary indicators.

\item To detect name changes of forums, a database of known changes is
  used, in conjunction with a way of specifying the changes.

\item Inconsistent tag usage should be handled by mapping entries from
  the same forums and comparing the tags in use for missing and
  additional tags.  Comparing forum entries over time will increase
  the usefulness, but adds the need to handle changes in the tags in
  use over time.

\item Inconsistent entry keys should be handled by having a naming
  scheme based on the data in the entries, with a way to disambiguate
  if two different entries would get the same name.
\end{itemize}


\subsubsection{Configuration}

Both of the steps above require ways to configure the behavior.  The
use of configurations should be as close to {\bibtex} as possible.

The preference to using the {\bibtex} format allows people to use what
they are familiar with.  Using a format that are readily supported in
programming frameworks, like JSON or XML might be considered easy by
the computer scientist, especially if it is one used to working with
those formats.  However, people outside computer science, such as a
physicist or the helpful chemist from earlier, will likely not be
familiar with such formats, nor should they.  A user of {\bibtex}
should at best only be concerned with {\bibtex}, when working with
{\bibtex}.

To use something in anger is an idiom for if something has been tested
in practice.  The idea of coding in anger has been expressed by Philip
Wadler~\cite{wadler1997_functional}.  For the {\bibtex} user, coding
in anger, could be the situation where the deadline is getting closer
and one just need things to work, at best ten minutes ago.  When faced
with the frustrations like that one tend not to care about beautiful
and elegant solutions, but rather wanting something that works with a
minimum of effort.  To accommodate the user writing {\bibtex} in
anger, is another reason to keep the specification as close to
{\bibtex} as possible, since it will minimize the effort.

Configuration should be done via de-facto standards inside the
\file{bib}, whenever possible.  For some configurations, de-facto
standards inside the \file{bib} is unreasonable.  These configurations
is better put into separate files.  However, the specification of
should still be designed to match {\bibtex} as closely as possible,
\ie, still using entries, tags and values to configure.


\subsection{In practice}

\subsubsection{De-facto configurations}

To account for the specifications about introducing two de-facto
standards will be appropriate: \texttt{OLDforum} to mark a previous
name of a forum, and \texttt{OPTanalyze} to configure tell the tool
about entry specific details.

The division into two de-facto standards is twofold: for
\texttt{OLDforum} the additional standard will allow bibliography
styles to make use of the additional information (one could easily
imaging a bibliographic style write ``\texttt{NISSC \textit{formerly
    known as} NCSC}''), and for \texttt{OPTanalyze} the content is
considered unlikely to be relevant to print in a bibliography.
Furthermore, the settings for \texttt{OPTanalyze} is kept in one tag
to prevent a multitude of new tags.

\remark{Yikes, any way to make the following clearer?!?!?}

For the \texttt{OLDforum} tag the value should be the string containing the
old name for the forum.  In some cases a forum can have changed its
name multiple times.  When multiple name changes has happened, and
referring to an old forum name, referring the most recent name
actively used in the \file{bib} is desired.  The design of the tag is
intended for disambiguation within a given file, not all files in
general.  If the tag is used for a bibliography style having a
multitude of names as the value will likely be confusing.  However,
only using names from the \file{bib} will cause the need for
re-detecting name changes, if an entry with a forum name between the
two entries are introduced.  Alternatively the format could be a list
of names (comma separated, since {\bibtex} use a comma as a separator),
this list would allow detailed backtracking.  A design with detailed
backtracking

Defining entry level deviations is done using \texttt{OPTanalyze},
using spaces to separate multiple settings.  The values for desired
deviations is as follows:

\begin{itemize}
\item \texttt{@DUPLICATEOK=X} to specify that an entry marked as a
  potential duplicate if deliberate, replacing \texttt{X} with the
  entry key of the potential duplicate.
\item \texttt{@LANG=XX} to specify the desired spell check language,
  replacing \texttt{XX} with the language code desired.
\item \texttt{@SPELLINGOK} to mark the spelling as correct.
\item \texttt{@NAMESOK} to mark that the names are correct.
\item \texttt{@INITIALSOK} to mark that the initials are correct.
\item \texttt{@NOLOOKUP} to mark that no look up should be done for
  the content of this entry.
\item \texttt{@CONFORMITYOK} to mark conformity to the specification
  and de-facto standards as correct.
\item \texttt{@ABBREVIATIONOK} to mark an abbreviated form as correct.
\item \texttt{@STRINGSOK} to mark that the texts has been checked for
  strings and that is is correct.
\item \texttt{@TAGSOK=forum} to mark that the usage of tags is correct and
  defines the standard for tag use for the entries from the \emph{same
  occurrence} of the forum.
\item \texttt{@TAGSOK=future} to mark that the usage of tags is
  correct and defines the standard for tag use for the entries from
  the \emph{same and future occurrences} of the forum.
\item \texttt{@TAGSOK=single} to mark that the usage of tags is
  correct for this single entry, not affecting other entries from the
  forum.
\item \texttt{@ENTRYKEYOK} to mark the entry key as correct.
\item \texttt{@LEXICALLYOK} to ignore all lexical checks for the
  entry, should be used with care.
\item \texttt{@CONSISTENCYOK} to ignore all consistency checks for the
  entry, should be used with care.
\item \texttt{@OK}, to mark an entry as fully correct, same as
  \texttt{@CONFORMITYOK @LEXICALLYOK @CONSISTENCYOK}, should be used
  with care.
\end{itemize}

The settings: \texttt{@ABBREVIATIONOK}, \texttt{@LEXICALLYOK},
\texttt{@CONSISTENCYOK}, and \texttt{@OK} may be a bit debatable on
necessity, however, their presence do ensure that the configuration is
complete and consistent.

For the conformity and de-facto check, it would be and option to have
settings for explicitly specifying the deviations in the use of tags.
However, explicitly allowing and denying tags will in most cases be
redundant, since once the deviation has been accepted it is accepted.

A similar set of tags could also be defined for the consistency check.
For the consistency check, it can be argued that an entry not
conforming to the standards of a forum may be updated to do so.  For
instance, if the forum standard have an ISSN on all entries and some
of the entries do not have an ISSN at the time.  Those entries might
get an ISSN later, which we want add, and in that case the
\texttt{@CONSISTENCYOK} become redundant, knowing this by explicitly stating
that the missing ISSN is the reason.  Another approach to
configurations becoming redundant would be to expand the tool to
detect configurations that serve no purpose.  This approach is
considered more appropriate, because it ensures more simplicity for
the user.

Another potential option would be to have a configuration for trusted
lookup services.  For example is one knows that an entry is correct in
a certain database then specifying that this database is to be
trusted.  This configuration might lead to a false sense of security,
since there is no way to guarantee that the data will not, some day,
be corrupted in the database of the lookup service.

An example of the configurations can be seen in
\figref{fig:analyzing_added_de_facto_standards}

\remark{Fill in the example}
\begin{figure}
  \centering
\begin{verbatim}
@{
  OLDforum = "",
  OPTanalyze = "@LANG=DA @NAMESOK @STRINGSOK"
}
\end{verbatim}
  \caption{An example using the de-facto standards for configuration}
  \label{fig:analyzing_added_de_facto_standards}
\end{figure}

To make the configurations even more intuitive for a {\bibtex} user,
an option is to add {\bibtex} strings with the relevant options in the
top of one's \file{bib}.  Then when configuring one can just use the
{\bibtex} strings and concatenate the relevant configurations.


\subsubsection{Other configurations}

Some of the configurations desired is not specific to an entry, but
rather the entire \file{bib}.  These configurations need to be
specified outside the specific entries.  Two options is: to put such
options inside one's \file{bib}, or put them in separate files.
Adding them to one's \file{bib} will introduce additional mess in the
file, and also the configurations may not correspond to proper
{\bibtex} formatting.  Having the configurations in separate files,
provides separation of concerns.  Furthermore, if the configuration is
in separate files they can easily be shared, such as a research
department having a standard, or a publisher who want authors to
follow their setup.

The format for such configurations should follow the format for a
{\bibtex} file.  Thus to the extend possible, it should consist of
entries with tags and values for configuration.  For other settings
defining {\bibtex} strings will be the favored choice.

For changes in names of forums a database of such changes is needed.
Currently no such database exist (to the authors knowledge), so the
user will need a way to specify his own.  Even if such a database did
exist, for the same reasons as lookups, the database might not be
complete.  A name change can be specified by having two entries for
the desired forum, adding the \texttt{OLDforum} tag to the entry
marking the name change.

\begin{figure}
  \centering
\begin{verbatim}
@PROCEEDINGS{forum_ncsc,
  title = "National Computer Security Conference"
}

@PROCEEDINGS{forum_nissc,
  title = "National Information Systems Security Conference",
  OLDforum = "ncsc_forum"
}
\end{verbatim}
  \caption{Configuring a name change of a forum}
  \label{fig:analyzing_configuration_name_change}
\end{figure}

The initial draft of the configuration is illustrated in
\figref{fig:analyzing_configuration_name_change}.  However, this
configuration ignores that the names in the actual \file{bib} may be
in their abbreviated form.  This way of configuring will work for most
journals, but proceedings is often, if not always, named according to
how many times a conference has been hold.  Thus in the example above
an entry would be named something like \texttt{Proceedings of the 20th
  national information systems security conference}.  Extracting the
names to strings is a part of solving the abbreviation of forum names.
Referencing the same strings instead will enable a parser to detect
the presence of the same string.  Thus if the \file{bib} looks like
\figref{fig:analyzing_configuration_name_change_bib_file_strings} then
the corresponding configuration will look like \figref{fig:analyzing_configuration_name_change_config_file_strings}.

\begin{figure}
  \centering
\begin{small}
\begin{verbatim}
% Re-usable strings
@STRING{PROCintro = "Proceedings of the"}
@STRING{nissc = "National Information Systems Security Conference"}

% Conferences
@STRING{nissc20 = PROCintro # "20th" # nissc}

% Proceedings
@INPROCEEDINGS{porras1997emerald,
  title = "EMERALD: Event monitoring enabling response to anomalous live disturbances",
  author = "Porras, Phillip A and Neumann, Peter G",
  booktitle = nissc20
}
\end{verbatim}
\end{small}
  \caption{\file{bib} using strings for conference names}
  \label{fig:analyzing_configuration_name_change_bib_file_strings}
\end{figure}

\begin{figure}
  \centering
\begin{verbatim}
@PROCEEDINGS{forum_nissc,
  title = nissc,
  OLDforum = "forum_ncsc"
}
\end{verbatim}
  \caption{\file{bib} using strings for conference names}
  \label{fig:analyzing_configuration_name_change_config_file_strings}
\end{figure}

The configuration of name changes should be using the most general
entry type available, such as \texttt{@ARTICLE}, \texttt{@PROCEEDINGS}
and \texttt{@BOOK}.  Either using rules in the name change checker to
correctly map the values from the general entry types to the specific
types, such as mapping \texttt{@PROCEEDINGS} to
\texttt{@INPROCEEDINGS} and detecting that the title tag in
\texttt{@PROCEEDINGS} correspond to booktitle in
\texttt{@INPROCEEDINGS}.

\remark{Oh my - that exploded rather quickly}

When specifying the de-facto standards, {\bibtex} entries should be
used.  When specifying a standard only deviations should be specified.
The configurations available are:

\begin{itemize}
\item \texttt{@required} for a tag we require to be present.
\item \texttt{@optional} for optional tags.
\item \texttt{@deny} for tags that are in the default configuration,
  that we want to reject
\item \texttt{@exludes=tag} for a tag that excludes the use of another
  tag, replacing \texttt{tag} with the name of another tag.  For
  example, if one allows both \texttt{ISSN} and \texttt{DOI} as tags,
  but want to ensure that only one of the tags is present, one would
  have the following: \texttt{ISSN = "@required @exludes=DOI"} and
  \texttt{DOI = "@required @excludes=ISSN"}.
\item \texttt{@includes=tag} for a tags where one of them is required
  and the other optional.  Usage is similar to \texttt{@excludes=tag}.
\end{itemize}

An example of a configuration of standards can be seen in
\figref{fig:analyzing_standards_config}.  This example sets article
entries to: reject \texttt{address} tags, that either \texttt{DOI} or
\texttt{ISSN} is present (but not both) and adds \texttt{url} as an
optional tag.  For books entries the example sets: that ISBN10 and/or
ISBN13 must be present.

\begin{figure}
  \centering
\begin{verbatim}
@article{standards_article,
  address = "@deny",
  DOI = "@required @excludes=ISSN",
  ISSN = "@required @excludes=DOI",
  url = "@optional"
}

@book{standards_book,
  ISBN10 = "@required @inclusive=ISBN13",
  ISBN13 = "@required @inclusive=ISBN10"
}
\end{verbatim}
  \caption{A snippet of the desired {\bibtex} based configuration for the correctness checker}
  \label{fig:analyzing_standards_config}
\end{figure}

Abbreviations of journal names can be configured by, introducing two
new tags: \texttt{abbreviated} and \texttt{fullname}, specifying the
abbreviated journal name and full journal name respectively.

\remark{Bah, proceedings are not as easy ><}

Inconsistent tags

For the specification of entry keys using a {\bibtex} string with the
predefined name \texttt{ENTRY\_KEY}.  Inside the string some way of
specifying the desired template for entry keys is needed.  Using a
template scheme, such as \texttt{\{tag\}} to match tags, is probably
the best solution.  This template system contradicts the desire to
keep the format close to {\bibtex}, but no better way has been found.
A template could look like: \texttt{\{author\}\{year\}\{title\}}.

However, this template is insufficient when considering how people
actually name their entries.  A lot of people use similar schemes, but
using the first author's last name, and a significant word from the
title.  Specifying these more detailed schemes will move the template
format further away from {\bibtex}, and unfortunately no better
approach has been for this either. \remark{some formatting needed here}

Fortunately, {\bibtex} strings comes to the rescue - at least
partially.  Having strings for the most common matches will ensure
that most users will never need see, nor even know about, the
underlying pattern matching system.  This will allow the user to use
concatenations to build the desired pattern.

\remark{Still need a good design for these strings}

\begin{figure}
  \centering
\begin{verbatim}
@STRING{ENTRY_KEY = author # year # "_" # title}
\end{verbatim}
  \caption{An example of a entry key pattern}
  \label{fig:analyzing_entry_key_pattern}
\end{figure}

\remark{Still need to specify consistency changes and if a tag is the
  desired consistency.}


\section{{\orangutan}}

\subsection{Introduction}

An attempt at implementing is {\orangutan}.


\subsection{Why {\orangutan} came to be}

The advent of {\bibtex} has been a game changer for bibliographic
references, which makes the issues in {\bibtex} undesirable.  There
are a lot of tools that provide partial solutions, so having an
augmenting tool would be an improvement.  The tool will be most
effective if it address first the correctness and lexical concerns
then the consistency concerns.  The attempt at making this tool has
been named \newdef{\orangutan}.


\subsection{What is {\orangutan}}

In the same spirit as {\bibtex}, {\orangutan} is designed to be a
simple software tool for improving bibliographic references.
{\orangutan} analyze and give suggestions for improvements to a
\file{bib}.


\subsection{How {\orangutan} is used in principle}

When analyzing {\bibtex} files using a two step solution is used
taking care of the correctness and lexical concerns first, then the
consistency concerns.  The two steps is used to ensure the best
possible conditions for handling consistency concerns.  The analyzing
tool then outputs the suggestions it has for improving the \file{bib}.

The tool use options, set by introducing a new de-facto standard.  The
options are for specifying if the language for the spell check and
that the entry is already considered to be correct.  To make the
options intuitive for the {\bibtex} user the options are designed to
match the {\bibtex} format.


\subsection{How {\orangutan} is used in practice}

The configuration format make use of that entry tags starting with
$OPT$ is a de-facto standard, for commenting a tag out.  This way of
configuring is chosen to prevent future clashes with standards.  A
simple way of specifying options is provided that feels natural for a
{\bibtex} user.  All the options are specified as an entry tag with
the name $OPTorangutan$.

In the current version there are three analyzing modules in use: a
spell checker, a correctness checker and an abbreviation checker.

The spell checker runs a spell check in the background using aspell.
Currently the spell checker is limited to only titles.  When spell
checking it use a configuration to specify the language, \eg,
$OPTorangutan = {@lang=DA}$.  Just as any other spell checker it marks
words that are misspelled and give a list of suggestions.

The correctness checker currently verifies the conformity with the
{\bibtex} specification and known de-facto standards.  Currently the
format for specifying entries is JSON, a {\bibtex} based format is
desired.  The format is relatively simple as seen in
\figref{fig:correctness_checker_json}.  The desired format for user
configuration is based on {\bibtex} such as can be seen in
\figref{fig:correctness_checker_bibtex}

\begin{figure}
  \centering
\begin{minted}{json}
{
  "book": {
    "author": {
      "required": true,
      "excludes": "editor"
    },
    "editor": {
      "required": true,
      "excludes": "author"
    },
[...]
\end{minted}
  \caption{A snippet of the JSON for configuring the correctness checker}
  \label{fig:correctness_checker_json}
\end{figure}

\begin{figure}
  \centering
\begin{verbatim}
@book{
  author = {@required @excludes=editor}
  editor = {@required @excludes=author}
[...]
\end{verbatim}
  \caption{A snippet of the desired {\bibtex} based configuration for the correctness checker}
  \label{fig:correctness_checker_bibtex}
\end{figure}

The abbreviation checker runs through journal names using a known list
of abbreviations.  The current version just suggest the full name
whenever an abbreviation is detected, these suggestions could be
heavily improved by suggesting strings.  The checking can be improved
by using know abbreviation standards or trying to detect abbreviations
that are unknown.


% Is there already fitting string for an journal name?
% Suggest strings for journal names, matching easier

\section{Summary and conclusions}

To summarize:

Having a tool that can analyze {\bibtex} files for issues is really
useful.  However, just like the lookup services can lure a user into
a false sense of security, so can \orangutan.  An analyzing tool will
not be a guarantee that there are no issues left.
