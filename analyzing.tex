\rfquote{Testing shows the presence, not the absence of bugs}{Edsger
  W. Dijkstra}


\section{Introduction}
The goal of this chapter is to show how {\bibtex} files are analyzed.


\section{What should be done about {\bibtex}}
\subsection{In principle}

To analyze {\bibtex} one would need to parse a \file{bib} into a
suitable representation.  This representation would need to sufficient
for parsing the {\bibtex} entries and strings at a minimum.  If one
intent to pretty print the result after resolving all issues, the
parsing also needs the preambles and preferably the comments.

The easiest approach is to take care of the lexical and correctness
concerns first then the consistency concerns.  By taking care of the
lexical and correctness concerns first the optimal conditions for
taking care of the consistency concerns is made.  After all the
concerns has been addressed a pretty printing utility will make sense
to ensure a clear and consistent structure inside one's \file{bib}.

Due to the practical issues in changing/replacing {\bibtex} and to
ensure separation of concerns such a tool should be an augmenting
tool.

\begin{itemize}
\item Duplicate entries should be detected with certainty when
  possible, using unique identifiers whenever possible otherwise
  identical entries defined by: entry tags and content in the tags
  being the same.  Whenever the entries are not identical a
  specification of similarity should be used having the title and
  authors as primary indicators of similarity.

\item Spelling errors in general, should be handled either by online
  look ups or a spell checker.  A combination of the two could be
  powerful as it would give a potential indicator of false positives
  from either.  Using a spell checker requires a way to configure the
  language for the individual entries.

\item Spelling errors in names can only be handled by databases of
  names, as a spell checker will give too many false positives.

\item Initials can be detected by finding cases with a single letter,
  perhaps followed by a punctuation. The handling will in part be by
  databases with names in combination of online look ups.

\item Online look ups, will be impossible to detect for certain as the
  intended search will be unknown.  The detection of the other issues
  can give an indicator of a bad look up.  To prevent further bad look
  ups, for instance when used as a supportive tool for the other
  issues, the most trusted online database available should be used
  whenever possible.  A system doing look ups in multiple databases
  will give further indications of potential issues.

\item Name changes of forums will require a database of known changes
  and a way of specifying the changes.  When a forum name has changed
  it will be a good idea to introduce a de-facto standard to highlight
  this, for instance a tag named $OLDforum$.

\item Conformity to de-facto standards and the {\bibtex} specification
  requires a set of rules specifying: the required and optional tags,
  the mutually exclusive tags, and the inclusive tags in {\bibtex}.
  \remark{Not good, neither exclusive or inclusive tags is defined}

\item Journal abbreviations should be handled by refactoring journal
  names into strings using a database to de-abbreviate or abbreviate.
  Furthermore detecting unknown abbreviations will be useful.

\item To handle {\bibtex} strings that end up as part of the text, the
  text should be compared with the strings in the \file{bib}.

\item Inconsistent tag usage should be handled by mapping entries from
  the same forums and comparing the tags in use for missing and
  additional tags.  Comparing forum entries over time will increase
  the usefulness but adds the need to handle changes in the tags used
  over time.

\item Inconsistent entry keys should be handled by having a naming
  scheme based on the data in the entries, with a way to disambiguate
  if two different entries would get the same name.
\end{itemize}


\subsection{In practice}
\remark{Olivier: I'm not sure if it's a good idea to specify the
  format fully, just feels the most complete thing to do.}

Configuration should be done via de-facto standards, preferably
corresponding to the {\bibtex} format. \remark{Describe the format}

For forums adding a tag with the name $OLDforum$ and the old forum
name - or even better the string for the old forum would make sense.
Only referring to the previous name, in the rare case that multiple
name changes should be present then the latest previous case in the
document should be adequate. \remark{:/ not sure I agree completely
  here}

A way to specify the deviations from the defaults should be done
through defining a de-facto standard tag containing the accepted
deviations.  Such a standard could be $OPTanalyze$.  In the value the
desired deviations would then be specified:

\begin{itemize}
\item $@OK$, to mark an entry as correct.
\item $@LANG=XX$ to specify the desired spell check language,
  replacing $XX$ with the language code desired.
\item $@SPELLINGOK$ to mark the spelling as correct.
\item $@NAMESOK$ to mark that the names are correct.
\item $@CONFORMITYOK$ to mark conformity as correct.
\item $@STRINGSOK$ to mark that the texts does not contain strings.
\item $@CONSISTENCYOK$ to mark the consistency as correct.
\end{itemize}

These tags can be combined using spaces as the separation character.
An example of an entry can be seen in
\figref{fig:analyzing_added_de_facto_standards}

\begin{figure}
  \centering
\begin{verbatim}
@{
  OLDforum = {},
  OPTanalyze = {@LANG=DA @NAMESOK @STRINGSOK}
}
\end{verbatim}
  \caption{An example using the de-facto standards for configuration}
  \label{fig:analyzing_added_de_facto_standards}
\end{figure}

\remark{Still need to specify consistency changes and if a tag is the desired
consistency.}

The use of de-facto standards should be specifiable through a
{\bibtex} like format.  A suggested way would be using entries
identical with {\bibtex} with the desired rules specified as the value
of the tags.  The idea is illustrated in
\figref{fig:correctness_checker_bibtex}.  The reason for not using the
$OPTanalyze$ is that future bibliographic styles may want to make use
of the knowledge of a previous name of a forum.  One could easily
imaging a bibliographic style write \texttt{National Information
  Systems Security Conference \textit{formerly known as} National
  Computer Security Conference}.

Entry key format should be specified using ?. \remark{Probably some
  kind of general config}.


\section{{\orangutan}}

\subsection{Introduction}

An attempt at implementing is {\orangutan}.


\subsection{Why {\orangutan} came to be}

The advent of {\bibtex} has been a game changer for bibliographic
references, which makes the issues in {\bibtex} undesirable.  There
are a lot of tools that provide partial solutions, so having an
augmenting tool would be an improvement.  The tool will be most
effective if it address first the correctness and lexical concerns
then the consistency concerns.  The attempt at making this tool has
been named \newdef{\orangutan}.


\subsection{What is {\orangutan}}

In the same spirit as {\bibtex}, {\orangutan} is designed to be a
simple software tool for improving bibliographic references.
{\orangutan} analyze and give suggestions for improvements to a
\file{bib}.


\subsection{How {\orangutan} is used in principle}

When analyzing {\bibtex} files using a two step solution is used
taking care of the correctness and lexical concerns first, then the
consistency concerns.  The two steps is used to ensure the best
possible conditions for handling consistency concerns.  The analyzing
tool then outputs the suggestions it has for improving the \file{bib}.

The tool use options, set by introducing a new de-facto standard.  The
options are for specifying if the language for the spell check and
that the entry is already considered to be correct.  To make the
options intuitive for the {\bibtex} user the options are designed to
match the {\bibtex} format.


\subsection{How {\orangutan} is used in practice}

The configuration format make use of that entry tags starting with
$OPT$ is a de-facto standard, for commenting a tag out.  This way of
configuring is chosen to prevent future clashes with standards.  A
simple way of specifying options is provided that feels natural for a
{\bibtex} user.  All the options are specified as an entry tag with
the name $OPTorangutan$.

In the current version there are three analyzing modules in use: a
spell checker, a correctness checker and an abbreviation checker.

The spell checker runs a spell check in the background using aspell.
Currently the spell checker is limited to only titles.  When spell
checking it use a configuration to specify the language, \eg,
$OPTorangutan = {@lang=DA}$.  Just as any other spell checker it marks
words that are misspelled and give a list of suggestions.

The correctness checker currently verifies the conformity with the
{\bibtex} specification and known de-facto standards.  Currently the
format for specifying entries is JSON, a {\bibtex} based format is
desired.  The format is relatively simple as seen in
\figref{fig:correctness_checker_json}.  The desired format for user
configuration is based on {\bibtex} such as can be seen in
\figref{fig:correctness_checker_bibtex}

\begin{figure}
  \centering
\begin{minted}{json}
{
  "book": {
    "author": {
      "required": true,
      "excludes": "editor"
    },
    "editor": {
      "required": true,
      "excludes": "author"
    },
[...]
\end{minted}
  \caption{A snippet of the JSON for configuring the correctness checker}
  \label{fig:correctness_checker_json}
\end{figure}

\begin{figure}
  \centering
\begin{verbatim}
@book{
  author = {@required @excludes=editor}
  editor = {@required @excludes=author}
[...]
\end{verbatim}
  \caption{A snippet of the desired {\bibtex} based configuration for the correctness checker}
  \label{fig:correctness_checker_bibtex}
\end{figure}

The abbreviation checker runs through journal names using a known list
of abbreviations.  The current version just suggest the full name
whenever an abbreviation is detected, these suggestions could be
heavily improved by suggesting strings.  The checking can be improved
by using know abbreviation standards or trying to detect abbreviations
that are unknown.


% Is there already fitting string for an journal name?
% Suggest strings for journal names, matching easier

\section{Summary and conclusions}

To summarize:

Having a tool that can analyze {\bibtex} files for issues is really
useful.  However, just like the look up services can lure a user into
a false sense of security, so can \orangutan.  An analyzing tool will
not be a guarantee that there are no issues left.
