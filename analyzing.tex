\section{Introduction}
The goal of this chapter is?

\section{Content}

The parsing of {\bibtex} is done using a standard recursive parser.
It outputs the {\bibtex} in the JSON-format which is easily to
manipulate with a programming language.

A simple way of specifying options is provided that feels natural for
a {\bibtex} user.  $OPTorangutan = {@lang=FR, @OK}$

Throws each entry at an analyzing module that handles the specific
goals.  Current modules is spell checking, abbreviation checking and
conformity checking.  If no issues found it can delete the entries to give a
clean output that only tells what one need to change, so it will not
complain.

Spell checker runs through all titles using a standard aspell module,
the module can be configured via $@lang=DA$ no multilang support.

The abbreviation checker uses known abbreviation lists and strings to
detect possible abbreviations.

When checking for conformity it has a simple rule based system.  It
validates according to the specification of {\bibtex}, further more it
allows for de facto standards, and ignores tags starting with $OPT$,
and it ignores $doi$, $issn$ and for books it ignores $isbn$.
Configurable via JSON, to be configurable via {\bibtex} files.

Two-phase parsing most sensible since detecting consistency makes more
sense once all conference names are de-abbreviated.


When searching for issues in {\bibtex} it is desired to get an final
output that says all go.

% Is there already fitting string for an journal name?
% Suggest strings for journal names, matching easier

\section{Summary and conclusions}