\rfquote{Testing shows the presence, not the absence of bugs}{Edsger
  W. Dijkstra}


\section{Introduction}
The goal of this chapter is to show how {\bibtex} files are analyzed.


\section{What should be done about {\bibtex}}
\subsection{In principle}

Due to the practical issues in changing/replacing {\bibtex} and to
ensure separation of concerns an analyzing tool should be an
augmenting tool.

To analyze {\bibtex} one would need to parse a \file{bib} into a
suitable representation.  This representation would need to parse
{\bibtex} entries and strings at a minimum.  If one intents to pretty
print the result after resolving all issues, the parsing also needs
the preambles.  Comments are technically optional, but should be kept
too.

The easiest approach is a two step parse: first taking care of the
lexical and correctness concerns, then the consistency concerns.  By
taking care of the lexical and correctness concerns first the optimal
conditions for taking care of the consistency concerns is made.  After
all the concerns has been addressed a pretty printing utility will a
nice final touch, ensuring a consistent structure inside ones
\file{bib} and having the same indentations and formatting for text.

The term database i here use for both local and online databases,
since it does not matter if one use a local copy, if it is up to date.

\subsubsection{First step: Lexical and correctness}
\begin{itemize}
\item Spelling errors in general, should be handled either by online
  lookups or a spell checker.  A combination of the two could be
  powerful as it would give a potential indicator of false positives
  from either.  Using a spell checker requires a way to configure the
  language for the individual entries.

\item Spelling errors in names should be handled by databases with
  names.

\item Initials can be detected by finding cases with a single letter,
  perhaps followed by a punctuation. The handling will in part be by
  databases with names.

\item Online lookups, will be impossible to detect for certain as the
  intended search will be unknown.  The detection of the other issues
  can give an indicator of a bad lookup.  Online lookups, however, is
  a very likely way to handle the databases needed for this tool.  The
  most trusted online database available should be used whenever
  possible.  A system doing lookups in multiple databases will give
  further indications of potential issues.  The user should always
  confirm that results from an online lookup is the desired result,
  due to the risk of false positives.

\item Conformity to de-facto standards and the {\bibtex} specification
  requires a set of rules specifying: the required and optional tags,
  the exclusive tags, and the inclusive tags in {\bibtex}.

\item Validating the values, whenever possible, would be a further
  improvement.  This should be done on values where clear rules for
  correct input can be specified, such as ISSN, year and month.  For
  values that can further be verified using a database, such
  validation is recommended.

\item Journal abbreviations should be handled by refactoring journal
  names into strings.  It should use a database to de-abbreviate or
  abbreviate.  Furthermore detecting unknown abbreviations will be
  useful.

\item To handle {\bibtex} strings that end up as part of the text, the
  text should be compared with the strings in the \file{bib}.
\end{itemize}


\subsubsection{Second step: Consistency}
\begin{itemize}
\item Duplicate entries should be detected.  When possible, using
  unique identifiers.  Otherwise, by detecting identical entries
  defined by: the entry tags and content in the tags being the same.
  When the entries are not identical a specification of similarity
  should be used, having the title and authors as primary indicators,
  to suggest potential duplicates.  The tool should suggest merging
  when duplicates are found.

\item To detect name changes of forums, it is required to have: a
  database of known changes, and a way of specifying the changes.
  When a forum name has changed it should suggest usage of a de-facto
  standard to highlight this.

\item Inconsistent tag usage should be handled by mapping entries from
  the same forums and comparing the tags in use for missing and
  additional tags.  Comparing forum entries over time will increase
  the usefulness, but adds the need to handle changes in the tags in
  use over time.

\item Inconsistent entry keys should be handled by having a naming
  scheme based on the data in the entries, with a way to disambiguate
  if two different entries would get the same name.
\end{itemize}


\subsubsection{Configuration}

Both of the steps above require ways to configure the behavior.

Configuration should be done via de-facto standards inside the
\file{bib}, whenever possible.  The preference to using the {\bibtex}
format allows people to use what they are familiar with.  Using a
format that are readily supported in programming frameworks, like JSON
or XML might be considered easy by the computer scientist, especially
if it is one used to working with those formats.  However, people
outside computer science, such as a physicist or the helpful chemist
from earlier, will likely not be familiar with such formats, nor
should they.  A user of {\bibtex} should at best only be concerned
with {\bibtex}, when working with {\bibtex}.

For some of the configurations they will be unreasonable to have
inside the \file{bib} itself.  These configurations is better put into
separate files.  However, the specification of should still be
designed to match {\bibtex} as closely as possible, \ie, still using
entries, tags and values to configure.


\subsection{In practice}

For forums adding a tag with the name $OLDforum$ and the old forum
name - or even better the string for the old forum would make sense.
Only referring to the previous name, in the rare case that multiple
name changes should be present then the latest previous case in the
document should be adequate. \remark{:/ not sure I agree completely
  here}

A way to specify the deviations from the defaults should be done
through defining a de-facto standard tag containing the accepted
deviations.  Such a standard could be $OPTanalyze$.  In the value the
desired deviations would then be specified:

\begin{itemize}
\item $@OK$, to mark an entry as correct.
\item $@LANG=XX$ to specify the desired spell check language,
  replacing $XX$ with the language code desired.
\item $@SPELLINGOK$ to mark the spelling as correct.
\item $@NAMESOK$ to mark that the names are correct.
\item $@CONFORMITYOK$ to mark conformity as correct.
\item $@STRINGSOK$ to mark that the texts does not contain strings.
\item $@CONSISTENCYOK$ to mark the consistency as correct.
\end{itemize}

These tags can be combined using spaces as the separation character.
An example of an entry can be seen in
\figref{fig:analyzing_added_de_facto_standards}

\begin{figure}
  \centering
\begin{verbatim}
@{
  OLDforum = {},
  OPTanalyze = {@LANG=DA @NAMESOK @STRINGSOK}
}
\end{verbatim}
  \caption{An example using the de-facto standards for configuration}
  \label{fig:analyzing_added_de_facto_standards}
\end{figure}

\remark{Still need to specify consistency changes and if a tag is the desired
consistency.}

The use of de-facto standards should be specifiable through a
{\bibtex} like format.  A suggested way would be using entries
identical with {\bibtex} with the desired rules specified as the value
of the tags.  The idea is illustrated in
\figref{fig:correctness_checker_bibtex}.  The reason for not using the
$OPTanalyze$ is that future bibliographic styles may want to make use
of the knowledge of a previous name of a forum.  One could easily
imaging a bibliographic style write \texttt{National Information
  Systems Security Conference \textit{formerly known as} National
  Computer Security Conference}.

Entry key format should be specified using ?. \remark{Probably some
  kind of general config}.


\section{{\orangutan}}

\subsection{Introduction}

An attempt at implementing is {\orangutan}.


\subsection{Why {\orangutan} came to be}

The advent of {\bibtex} has been a game changer for bibliographic
references, which makes the issues in {\bibtex} undesirable.  There
are a lot of tools that provide partial solutions, so having an
augmenting tool would be an improvement.  The tool will be most
effective if it address first the correctness and lexical concerns
then the consistency concerns.  The attempt at making this tool has
been named \newdef{\orangutan}.


\subsection{What is {\orangutan}}

In the same spirit as {\bibtex}, {\orangutan} is designed to be a
simple software tool for improving bibliographic references.
{\orangutan} analyze and give suggestions for improvements to a
\file{bib}.


\subsection{How {\orangutan} is used in principle}

When analyzing {\bibtex} files using a two step solution is used
taking care of the correctness and lexical concerns first, then the
consistency concerns.  The two steps is used to ensure the best
possible conditions for handling consistency concerns.  The analyzing
tool then outputs the suggestions it has for improving the \file{bib}.

The tool use options, set by introducing a new de-facto standard.  The
options are for specifying if the language for the spell check and
that the entry is already considered to be correct.  To make the
options intuitive for the {\bibtex} user the options are designed to
match the {\bibtex} format.


\subsection{How {\orangutan} is used in practice}

The configuration format make use of that entry tags starting with
$OPT$ is a de-facto standard, for commenting a tag out.  This way of
configuring is chosen to prevent future clashes with standards.  A
simple way of specifying options is provided that feels natural for a
{\bibtex} user.  All the options are specified as an entry tag with
the name $OPTorangutan$.

In the current version there are three analyzing modules in use: a
spell checker, a correctness checker and an abbreviation checker.

The spell checker runs a spell check in the background using aspell.
Currently the spell checker is limited to only titles.  When spell
checking it use a configuration to specify the language, \eg,
$OPTorangutan = {@lang=DA}$.  Just as any other spell checker it marks
words that are misspelled and give a list of suggestions.

The correctness checker currently verifies the conformity with the
{\bibtex} specification and known de-facto standards.  Currently the
format for specifying entries is JSON, a {\bibtex} based format is
desired.  The format is relatively simple as seen in
\figref{fig:correctness_checker_json}.  The desired format for user
configuration is based on {\bibtex} such as can be seen in
\figref{fig:correctness_checker_bibtex}

\begin{figure}
  \centering
\begin{minted}{json}
{
  "book": {
    "author": {
      "required": true,
      "excludes": "editor"
    },
    "editor": {
      "required": true,
      "excludes": "author"
    },
[...]
\end{minted}
  \caption{A snippet of the JSON for configuring the correctness checker}
  \label{fig:correctness_checker_json}
\end{figure}

\begin{figure}
  \centering
\begin{verbatim}
@book{
  author = {@required @excludes=editor}
  editor = {@required @excludes=author}
[...]
\end{verbatim}
  \caption{A snippet of the desired {\bibtex} based configuration for the correctness checker}
  \label{fig:correctness_checker_bibtex}
\end{figure}

The abbreviation checker runs through journal names using a known list
of abbreviations.  The current version just suggest the full name
whenever an abbreviation is detected, these suggestions could be
heavily improved by suggesting strings.  The checking can be improved
by using know abbreviation standards or trying to detect abbreviations
that are unknown.


% Is there already fitting string for an journal name?
% Suggest strings for journal names, matching easier

\section{Summary and conclusions}

To summarize:

Having a tool that can analyze {\bibtex} files for issues is really
useful.  However, just like the lookup services can lure a user into
a false sense of security, so can \orangutan.  An analyzing tool will
not be a guarantee that there are no issues left.
