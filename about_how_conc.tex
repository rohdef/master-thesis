\section{How {\bibtex} is used today}
\label{sec:how_bibtex_is_used_today}

{\bibtex} is to this day heavily used by researchers, witness that
almost all online resources for finding references provide {\bibtex}
output.  There are a lot of databases online that is widely used to
look up entries, which has also given rise to a lot of ways to
identify articles, such as: arXiv numbers, DOI and ISSN.

Since {\bibtex} is capable of only printing the references used in a
{\LaTeX}-document most people start using {\bibtex} as a database
having a complete file with the references they have used.  A lot of
people also find it practical to use their {\bibtex}-file as a way to
keep track of what they read.

That {\bibtex} ignores unknown tags is also widely used, both for de
facto standards and to add additional information that the format does
not specify, such as the before mentioned arXiv numbers, DOI and ISSN.
It is so widely used that is is common to see article search engines
make use of this and there is libraries that provide {\bibtex}-styles
that actually make use of these tags.


\section{Summary and conclusion}

To summarize: {\bibtex} is a product of the advent of {\TeX} and the
need for managing references that {\TeX} made it possible to automate.
{\bibtex} is a fairly simple format.  {\bibtex} provides a simple way
to manage references and it is fairly flexible and thus allows for
useful de facto standards.  The format inside a {\bibtex}-file is a
simple and relatively intuitive format, dividing the entries in to
types corresponding to the medium it represents and having tags
relevant to that medium.  {\bibtex} has become widely used and given
rise to useful de facto standards and tools to assist with
bibliographic references.

What to conclude? :S


\chapter{Our approach}
\section{Intro}

The goal of this chapter is to organize the previously identified
problems and reach a suitable approach for how to handle them.


\section{Conclusion}

To summarize: in practice {\bibtex} contains lexical and inconsistency
concerns.  These concerns arise from a variety of sources, in part
from the design and tools and in part because a lot of users are not
concerned with writing good {\bibtex}
