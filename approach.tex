\section{Introduction}
The goal of this chapter is to organize the previously identified
problems and reach a suitable approach for how to handle them.

The goal of this chapter is to organize the issues people have with
{\bibtex}, answering \chapref{sec:intro_what_issues},
\chapref{sec:intro_whyhow}, \chapref{sec:intro_what_to_do}

\section{What are the issues in {\bibtex}}
\label{sec:intro_what_issues}

{\bibtex} has changed the landscape for scientific writing and eased a
lot of peoples lives, however it is not without issues.  The
challenges range widely, trying to group similar looking issues we
have:

The use of abbreviations for journal names and the use of initials in
author names are fairly similar apart from {\bibtex} being able to make
names into initials whereas it cannot abbreviate journal names.
Misspellings and {\bibtex} strings that end up being text is also
similar.  As all of these issues are related to the words these will
be considered as lexical concerns.

Having inconsistent use of tags, inconsistent entry keys, names that
change, lack of conformance to the desired rules (specification and de
facto standards) and duplicate entries also seems similar.  Together
these will be considered as conformity concerns.

The Utopian goal is to provide a structural solution for all these
issues so that if any further issues exist they would be purely
conjunctural.

%Lexical:
%- Journal abbreviations
%- Initials
%- Spelling errors
%- Strings that end up as text
%
%Consistency:
%- Inconsistent tags
%- Duplicate entries
%- Inconsistent entry keys
%- Name changes in conferences, journals etc.
%- Conformity
%
%Online look ups


\section{Why are the {\bibtex} issues there and how do they come into
  existence}
\label{sec:intro_whyhow}

The sources of the issues can be considered from the perspective of
structural vs conjunctural, which shows that there is no uniform way
of considering the reason for the challenges.  Using this approach has
uncovered a lot of the underlying reasons.

By design {\bibtex} does not prevent using the format however the user
desires, for good or bad.  This design results in both a lot of
inconsistencies and missing information, but also in useful de facto
standards.

An example of that {\bibtex} does not prevent deviations from the
format can be seen from output in \figref{fig:bibtex_out}, the output
is from a file that does not conform to the specifications and
contains unspecified tags.  {\bibtex} just warns about the missing
tags, it warns that the book has both author and editor and it
completely ignores all the unspecified tags.

\begin{figure}
  \centering
  \begin{verbatim}
This is BibTeX, Version 0.99d (TeX Live 2015/Debian)
The top-level auxiliary file: doc.aux
The style file: plain.bst
Database file #1: mybib.bib
Warning--can't use both author and editor fields in book
Warning--empty publisher in book
Warning--empty year in book
Warning--empty journal in Nobody06
Warning--empty year in Nobody06
\end{verbatim}
  \caption{Example of error output from {\bibtex}}
\label{fig:bibtex_out}
\end{figure}

When adding entries to {\bibtex} there are various sources of errors.
Manual entry easily leads to spelling errors and as the file grows
large inconsistencies.  Entering by hand using copy and paste can
cause existing issues to be repeated.  Using online resources, one may
expect the entries to be correct, but is not given that they are.  The
entries will be no better than the source used.  Using the automatic
tools for extraction of meta data, from articles, may be prone to
further errors.

Some tools for editing {\bibtex} files do not necessarily follow the
specification, and does not assist in ensuring consistency.  A lot of
the tools can convert between abbreviated and full form in the files,
essentially motivating to store abbreviations rather than full names.
Some even suggest tags that does not belong in the entry type.


\section{What can be done about the {\bibtex} issues}
\label{sec:intro_what_to_do}

\remark{Hmm this is actually the reason why we will address this
  structurally}

Addressing the human factor in it is one theoretically possibile way
of solving these issues.  ``Simply'' motivating people to do things
right.  Alas people are not machines and thus this approach will be
impossible in practice, for most people the interest is not the tools
they use but what they use them for, so the interest in {\bibtex} will
for most people being in ensuring that their articles contain the
relevant references.


As previously stated the structural approach to the issues is desired.
This choice of approach basically means ensuring that the tools around
{\bibtex} supports handling the issues, preferably to the level where
all issues are solved.  Naturally, as touched upon shortly when
inspecting the problems, it is not likely that all issues can be
solved perfectly.

\subsection{Updating {\bibtex}}

One way of handling the issues structurally would be to change or even
replace {\bibtex}, so it handles all lexical and consistency concerns.
This way would include changing the {\bibtex} specification to account
for relevant de facto standards, enforcing conformity, handling
abbreviations and controlling all data.  The updated version of
{\bibtex} could then either correct the issues when running into them
or fail building the bbl-file with appropriate error messages for
issues that the user need to take care of.



The lack of a reliable way to actually control all data and that it
would be impossible to both account for current and future de facto
standards would be an issue.  The current and future standards could
be solved by having an option for the standards.  Furthermore this
approach would be probably be perceived as invasive as it would cause
existing {\bibtex} files not to work and it would impose the tool on
the users with requirements they may not desire nor care about.  The
perception would of course depend of perspective because the user
wanting structure and control might find it good that it is being
enforced.

As will be inspected in \nameref{ch:related} there is actually a few
attempts at both changing and replacing {\bibtex}.

\subsection{Augmenting {\bibtex}}

In stead of changing or replacing {\bibtex} an augmenting tool is
another option.  Such a tool would have to analyze a {\bibtex} file
and either provide or suggest improvements.  As covered in
\nameref{ch:problem-description} there are a lot of cases where
automatically providing solutions would introduce new structural
issues.

%\subsection{What can be done - analyzing tool \remark{working title}}

%\remark{remember the new revisions desire}

%\remark{People who request similar
%http://stackoverflow.com/questions/13630584/the-best-method-for-handling-bibtex-files
%http://stackoverflow.com/questions/32838020/awk-how-to-clean-bibtex-files
%http://www.latex-community.org/forum/viewtopic.php?f=50&t=12043
%http://tex.stackexchange.com/questions/76420/cleaning-up-a-bib-file
%http://tex.stackexchange.com/questions/174509/is-there-a-tool-service-that-can-enrich-a-bibtex-database
%<http://tex.stackexchange.com/questions/128989/how-can-one-validate-a-bib-file
%http://tex.stackexchange.com/questions/173621/how-to-validate-check-a-biblatex-bib-file
%http://stackoverflow.com/questions/13630584/the-best-method-for-handling-bibtex-files
%}

%\section{Lookup services}
%\remark{So far mostly notes to self}
%DOI

\section{What to be done}
\subsection{Introduction}
\subsection{Filler}
\subsection{\nameref{sec:problems_de_facto}}
\subsection{\nameref{sec:problems_abbreviations}}
\subsection{\nameref{sec:problems_inconsistent_tags}}
\subsection{\nameref{sec:problems_duplicates}}
\subsection{\nameref{sec:problems_look_ups}}
\subsection{\nameref{sec:problems_inconsistent_keys}}
\subsection{\nameref{sec:problems_name_changes}}
\subsection{\nameref{sec:problems_initials}}
\subsection{Summary  and conclusions}


\section{Summary and conclusions}

To summarize: in practice {\bibtex} contains lexical and inconsistency
concerns.  These concerns arise from a variety of sources, in part
from the design and tools and in part because a lot of users are not
concerned with writing good {\bibtex}.

%%% Local Variables:
%%% mode: latex
%%% TeX-master: "thesis"
%%% End:
