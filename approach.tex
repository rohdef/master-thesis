\section{Introduction}
The goal of this chapter is to organize the issues people have with
{\bibtex}.  Answering what the issues in {\bibtex} are
\chapref{sec:intro_what_issues} and how to approach the {\bibtex}
issues \chapref{sec:intro_what_to_do}

\section{What are the issues in {\bibtex}}
\label{sec:intro_what_issues}

{\bibtex} has changed the landscape for scientific writing and eased a
lot of peoples lives, however not without any issues.  The challenges
range widely and trying to group similar looking issues we have:

\remark{We may want the same order as in the list of issues, probably
  reformulate this}

Conforming to the specification and de-facto standards is considered
as a correctness concern.  The use of abbreviations for journal names,
initials in author names, misspellings in general, misspellings in
names and {\bibtex} strings that end up being text will be considered
as lexical concerns.  All of these issues will be considered combined
as \newdef{correctness and lexical concerns}.

Inconsistent use of tags, inconsistent entry keys, forum names that
change and duplicate entries will be considered as \newdef{consistency
  concerns}.

The Utopian goal is to provide a structural solution for all these
issues so that if any further issues exist they would be purely
conjunctural.


\section{What can be done about the {\bibtex} issues}
\label{sec:intro_what_to_do}

As previously stated, the structural approach to the issues is desired.
This choice basically means ensuring that the tools handles the
issues, preferably to the level where all issues are solved.  As
touched upon shortly when inspecting the problems, it is not likely
that all issues can be solved perfectly.  For a structural solution
there are two approaches:


\subsection{Updating or replacing {\bibtex}}

One way of handling the issues structurally would be to change or
replace {\bibtex}, so it handles all lexical and consistency concerns.
This way would include changing the {\bibtex} specification to account
for relevant de facto standards, enforcing conformity, handling
abbreviations and controlling all data.  The updated version of
{\bibtex} could then either correct the issues when running into them
or fail building the \file{bbl} with appropriate error messages for
issues that the user need to take care of.

This approach would be probably be perceived as invasive as it would
cause existing {\bibtex} files not to work and it would impose the
tool on the users with requirements they may not desire.  The
perception would of course depend of perspective because the user who
wants structure and control might find it good that it is enforced.

As will be inspected in Chapter~\ref{ch:related} there are actually a
few attempts at both changing and replacing {\bibtex}.


\subsection{Augmenting {\bibtex}}

Instead of changing or replacing {\bibtex} an augmenting tool is
another option.  Such a tool be used together on {\bibtex} file
provide or suggest improvements, in stead of changing specifications.
An augmenting tool will be a supplement to current use of {\bibtex}
and be optional, rather than imposed on the users.


\section{How do we approach the {\bibtex} issues}
\subsection{Introduction}

The goal of this section is to introduce our choice of solutions for
the issues.


\subsection{Lexical and correctness concerns vs. consistency concerns}
\label{sec:approach_lexical_consistency}

The relation between the lexical and correctness concerns and the
consistency concerns reveals a dependency in the analysis.  Going
through the consistency concerns observing their relation gives:

\begin{itemize}
\item For inconsistent use of tags, a way to determine if entries are
  from the same forum is needed.  Such a way depends on consistent
  naming of the forum and a way to detect name changes.

\item For duplicate entries having unique identifiers such as arXiv
  numbers, ISSN or DOI will make the detection trivial.  Otherwise the
  detection has to be based on the similarity of the information, at
  best the information is identical, otherwise it has to be a similar
  as possible to improve the detection.  Thus solving the lexical
  concerns will be of use.

\item Inconsistent naming of entry keys can be handled by a naming
  scheme.  Such a naming scheme is usually based on the information in
  the entries.  So having the relevant tags and correct content in
  them will ensure a way to ensure consistent entry keys.

\item For name changes of forums we need to be able to recognize the
  names which is easier with correct and consistent names.
\end{itemize}

A common property about the consistency concerns is that they are
easier to handle, once the lexical and correctness concerns have been
handled.  This property indicates that a two phase solution may be
desired: first handling the lexical and correctness concerns, then
handling the consistency concerns.


\subsection{\nameref{sec:problems_duplicates}}
\label{sec:approach_duplicates}

Duplicate entries are fairly straight forward if the tags and the
contents are identical.  If the content and tags deviate a way of
detecting ``similarity'' will be needed, the easiest definition of
similarity is if the title and author is identical, however this
definition might need to take things like revisions and year into
account, as an author may decide to write a new version later or if
one for some reason desires to refer to different revisions.  Further
challenges may arise if there is lexical and correctness challenges as
per Section~\ref{sec:approach_lexical_consistency} fixing these first
is desired.


\subsection{\nameref{sec:problems_spelling}}

To detect misspellings a spell checker can be used.  Alternatively
checking the resources in online databases is an option.  If a spell
checker is used one should be aware false positives.  Domain specific
terms might not be present in the dictionary and if the original
source is misspelled, it would be a mistake to correct it (once
published the name published is the correct name of the reference!).
Provided a solution for the issues in online look ups the correct
spelling will be a matter of looking up, but references may not be in
the databases, and as stated in Section~\ref{sec:approach_look_ups}
there is no good way to ensure correct look ups.  So a spell checker
seems like a good way to get an indication of possible errors, but one
would still have to verify them.  As entries may be in different
languages a way of specifying the language should be considered.


\subsection{\nameref{sec:problems_spelling_names}}
\label{sec:approach_spelling_names}

Spell checking names with a normal spell checker will cause errors.  A
possible way to approach this would be to make online look ups in
databases with scientific authors, such as DBLP and Google Scholar.
This will still have the issue with name of authors who have not
published anything scientific, such the author of a book.  Extending
this solution to contain more databases such as book authors will
improve the solution, but will still be limited to known author names.


\subsection{\nameref{sec:problems_initials}}

Finding the initials is a matter of being able to detect single
letters with or without a period after it.  However if one for some
reason group initials together, \eg, making George R. R. Martin into
George RR Martin, then further detection will be needed.  Replacing
the initials with full names is appropriate whenever possible, but
since the full names may not be known some way of specifying that
initials is the only thing available is needed.  The best approach
will probably be the one described for spell checking names in
Section~\ref{sec:approach_spelling_names}.


\subsection{\nameref{sec:problems_look_ups}}
\label{sec:approach_look_ups}

Online database look ups can be a very useful tool for handling the
lexical and correctness concerns, but getting incorrect data can cause
problems.  The best approach to ensure correct look ups is if it is
possible to use services that are known to be correct.

A situation where relatively reliable look ups is possible, as in the
``EPTCS'' look up seen in \figref{fig:eptcs_lookup}, can be used to
improve the reliability of the results.  However there is still no
certain way to know if the database of the service is correct, so it
is still not certain.  Most likely the ID systems, such as arXiv
numbers, DOI and ISSN, will also provide a relatively reliable look up
mechanism, but it is still not guaranteed.

Another way to approach the bad look ups could be by doing the same
look up in multiple databases and then have some kind of voting system
that decides on which entry to trust.  This would however require
knowledge information sources each online service use, because their
source of information may be the same and then the same error could
get multiple votes.  The approach can be refined by having increased
trust in databases that are likely to be correct.

The most appropriate strategy is probably selecting the database most
likely to be correct and then have the user select if one agrees with
the result.  Doing the vote system would be overkill in most
situations, and the user still have to validate the result, since the
voting system will not provide a certain correct result.  Having the
user validate the result will make issues partly conjunctural, if one
just accept any result from the look up.


\subsection{\nameref{sec:problems_name_changes}}

Handling name changes of forums is supportive to ensuring consistency.
Since name changes cannot be derived automatically one approach would
be a database of known name changes, which has the disadvantage that
it needs to be maintained.  Adding a configuration to specify name
changes may also be appropriate.


\subsection{\nameref{sec:problems_de_facto}}

As stated in Section~\ref{sec:problems_de_facto} it is desired to be
able to validate if the file conforms to the specification and the
desired de-facto standards.  Validating conformity to the
specification is a simple task, as the specification is just a set of
rules.  A set of de-facto standards, likewise, is also a simple set of
rules.

De-facto standards however provide challenges, as they are the
standards currently in use.  This means that they both depends on who
the user is and the standards are subject to change.

A tool handling conformity to the specification and de-facto standards
should thus be configurable to account for changes in de-facto
standards.  For practicality the de-facto standards that are not
likely to change (such as ISSN and DOI) could be accounted for with a
default setting.


\subsection{\nameref{sec:problems_abbreviations}}

Ensuring a consistent use of either abbreviations or full names is
desired.  From the point of having the information in a complete
version converting full names, \newdef{de-abbreviating} is desired.
Using a database of standard abbreviations for forums will be useful
to de-abbreviate.  Taking care of a consistent way to switch between
full names and abbreviations is also desired.  Making use of strings
to handle the switching between full names and abbreviations is
probably the best approach since this will keep it clear which forum
is which.  This also allows the user to use string names that are:
full names, official names or their own style of abbreviations, to
their choice.


\subsection{\nameref{sec:problems_strings_as_text}}

A {\bibtex} string can end up being part of a text by mistake, in the
example used in Section~\ref{sec:problems_strings_as_text} where the
month ended up as a text rather than a string a simple check is
possible, because for a month we know what to expect.  Whenever
something in the middle of the text should have been a string, the
text would have to be checked for potential strings.  Automatically
correcting it would introduce a potential source of errors, because a
text being identical to a string name could just be a coincidence, so
it has to be the user's choice.


\subsection{\nameref{sec:problems_inconsistent_tags}}

Detecting inconsistent use of tags require a way of detecting when
entries are from the same forum.  When such a way is provided it is
possible to check if the set of fields are the same.  Having some kind
of statistics on the usage may further improve the feedback, since it
will be possible to suggest that the shortest path to consistency: if
it is by adding, or removing tags.

Since a lot of the forums are continuous, such as a conference being
held each year, the detail level of the information may change over
time.  Also in some cases a single item can have additional
information that are not general to the forum or for some reason not
have information according to the general standard.  Optimally there
should be a way to account for these cases, either by the user
enforcing conformity or having options for when a deviation occur.


\subsection{\nameref{sec:problems_inconsistent_keys}}

Provided that the lexical and correctness concerns has been solved,
handling inconsistent entry keys require very little effort.  Having a
rule for how the key names should be formatted is all that is needed.
Like in Section~\ref{sec:approach_duplicates} there is the issue of
similar, but different, entries.  Similar entries could result in the
same entry key, so there is the need for a way to disambiguate the key
names.  Since a lot of users already have databases in use, support
for one specifying a naming scheme would be appropriate.


%\subsection{Summary  and conclusions}
%To summarize:

\section{Summary and conclusions}

The issues in {\bibtex} files have been grouped in correctness and
lexical concerns and in consistency concerns.  Updating or replacing
{\bibtex} was compared to augmenting {\bibtex}.  It was observed that
the consistency concerns in general depends on the solution of the
correctness and lexical concerns.

\begin{itemize}
\item Duplicate entries can be found if there is a unique identifier
  or identical entries.  To detect deviating duplicates a definition
  of ``similarity'' will be needed.

\item Spelling errors in general can be solved by a spell checker and
  the usage of online look ups.  For a spell checker one must be aware
  of false positives and language.

\item Spelling errors in names will challenge normal spell checkers.
  Using online databases of authors will enable some checking but the
  solution will be limited to known author names.

\item Initials hiding peoples names can be handled by online
  resources, just as misspellings in names.

\item To get online look ups that contains the correct data is
  impossible, the results can be improved by selecting the most
  appropriate database for the look up and by introducing detection of
  erroneous look ups.

\item Name change of a forum can be handled by a database and by a way
  to specify name changes.

\item Conformity to de-facto standards and the {\bibtex} specification
  should be handled by checking the rules for the standards and being
  able to specify the desired de-facto standards.

\item Journal abbreviations should be moved into strings and be
  consistently de-abbreviated or abbreviated.

\item {\bibtex} strings that end up as part of the text should be
  detected by matching string names that appear in text.

\item Inconsistent tags should be detected based on when entries are
  from the same forum.  One should be able to specify deviations from
  the general set of information for the forum.

\item Inconsistent entry keys should be handled by a rule for the
  desired format for entry key names.
\end{itemize}


%\begin{figure}
%  \centering
%  \begin{verbatim}
%This is BibTeX, Version 0.99d (TeX Live 2015/Debian)
%The top-level auxiliary file: doc.aux
%The style file: plain.bst
%Database file #1: mybib.bib
%Warning--can't use both author and editor fields in book
%Warning--empty publisher in book
%Warning--empty year in book
%Warning--empty journal in Nobody06
%Warning--empty year in Nobody06
%\end{verbatim}
%  \caption{Example of error output from {\bibtex}}
%\label{fig:bibtex_out}
%\end{figure}


%\remark{People who request similar
%http://stackoverflow.com/questions/13630584/the-best-method-for-handling-bibtex-files
%http://stackoverflow.com/questions/32838020/awk-how-to-clean-bibtex-files
%http://www.latex-community.org/forum/viewtopic.php?f=50&t=12043
%http://tex.stackexchange.com/questions/76420/cleaning-up-a-bib-file
%http://tex.stackexchange.com/questions/174509/is-there-a-tool-service-that-can-enrich-a-bibtex-database
%<http://tex.stackexchange.com/questions/128989/how-can-one-validate-a-bib-file
%http://tex.stackexchange.com/questions/173621/how-to-validate-check-a-biblatex-bib-file
%http://stackoverflow.com/questions/13630584/the-best-method-for-handling-bibtex-files
%}


%%% Local Variables:
%%% mode: latex
%%% TeX-master: "thesis"
%%% End:
