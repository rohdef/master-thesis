\rfquote{All that matters on the chessboard is good moves.}{Robert
James Fischer (Bobby Fischer)}

In this dissertation the following has been covered: {\bibtex} have
been described specifying how to use it and how it is used
(Chapter~\ref{ch:about}), a range of challenges when working with
{\bibtex} (Chapter~\ref{ch:problem-description}), approaches to the
covered issues in {\bibtex} (Chapter~\ref{ch:approach}), how related
work handles these issues (Chapter~\ref{ch:related}), a way to analyze
{\bibtex} files to detect the issues (Chapter~\ref{ch:analyzing}), and
how organize {\bibtex} files using the results from the analysis.
(Chapter~\ref{ch:organizing}).

{\bibtex}, despite its wide use, is far from a perfect tool, and the
challenges faced range far and wide.  There are a lot of tools working
on {\bibtex} and a lot of alternatives, some of these do provide
partial solutions to some of the issues faced, but is far from
sufficient.  For most of these issues it is possible to provide
analysis tools that detect potential issues.  Such an analysis cannot
be perfect though, since any set of rules will provide false positives
or relies on assumptions that are not correct.  Because of the false
positives the analysis need to have configurations, through de-facto
standards, to tell it to ignore false positives.  The analysis can in
most cases also provide suggestions for improvements, such as the
suggestions from a spell checker.  These suggestions can be used to
organize the \file{bib} either correcting the issues or to add
de-facto tags to tell the analysis when it has a false positive.

Perspectives: