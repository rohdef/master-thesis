\rfquote{All that matters on the chessboard is good moves.}{Bobby
Fischer}

\noindent
Let us recapitulate: we have first described {\bibtex} -- both how to
use it in principle and how it is used in practice
(Chapter~\ref{ch:about}); we have then listed a range of practical
issues {\bibtex} users encounter
(Chapter~\ref{ch:problem-description}), we have proposed an approach
to hand\-ling them (Chapter~\ref{ch:approach}), and we have reviewed
how they are tackled in related work (Chapter~\ref{ch:related}); we
have then presented an analysis of {\bibtex} files that detects these
issues (Chapter~\ref{ch:analyzing}), and we have described how to
solve them by organizing {\bibtex} files using the results from this
analysis (Chapter~\ref{ch:organizing}).  We have implemented a part of
this analysis in a prototype, {\orangutan}.

{\bibtex}, despite its wide use, is far from a perfect tool, and
{\bibtex} user's face challenges that range far and wide.  Many tools
surround {\bibtex} and so do a lot of alternatives, some of which do
provide partial solutions to some of the issues one faces.  However,
as analyzed in Chapter~\ref{ch:related}, these tools are far from
sufficient.  For most of the challenges a {\bibtex} user faces, it is
possible to provide analysis tools that detect potential issues.

Such analyses cannot be perfect though, since any set of rules yields
false positives or relies on assumptions that are ill-founded since
{\bibtex} is not formally specified: while the soundness of the
specification of {\bibtex} is not questioned, its completeness is
unknown.  Analyses of {\bibtex} files therefore need to have
configurations, through de-facto standards, to detect and ignore false
positives.  In most cases, these analyses can also provide suggestions
for improvements, such as the suggestions from a spell checker.  These
suggestions can then be used to organize one's \file{bib}, either by
correcting the issues that have been detected or by adding de-facto
tags to prevent false positives.
%%%% Rewrite from here
% [and here report the status of orangutan at this time of writing,
% and its future as you see it today]

Just like chess players who strive to always improve the quality of
their moves, one can wish to improve the quality of one's {\bibtex}
files.  We have designed {\orangutan} as a proof of concept for
suggesting improvements.  It is our hope that this proof of concept
can contribute to improving the general quality of {\bibtex} files
and, consequently, can improve the precision of bibliographic
references in documents as well as save time for their authors and
their readers.