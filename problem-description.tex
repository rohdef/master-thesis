\rfquote{Anything that can go wrong, will go wrong.}{Murphy's Law}

\section{Introduction}
The goal of this chapter is to introduce the challenges of {\bibtex}
by showing concrete examples.  The terms of discourse, structural and
conjunctural, will be defined
\chapref{sec:problems_structural_conjunctural}.  The use of de-facto
standards and conformity to the {\bibtex} specification will be
covered \chapref{sec:problems_de_facto}.  The problems with: journal
abbreviations \chapref{sec:problems_abbreviations}, inconsistent tags
\chapref{sec:problems_inconsistent_tags}, duplicate entries
\chapref{sec:problems_duplicates}, online look ups
\chapref{sec:problems_look_ups}, inconsistent entry keys
\chapref{sec:problems_inconsistent_keys}, name changes in for instance
conference names \chapref{sec:problems_name_changes}, covering the
issues of initials \chapref{sec:problems_initials}, spelling errors
\chapref{sec:problems_spelling} and {\bibtex} strings that end up in
the text \chapref{sec:problems_strings_as_text} will be covered.

\section{Structural vs conjunctural issues}
\label{sec:problems_structural_conjunctural}

Unfortunately, even though {\bibtex} has made life a lot better for
scientific authors, it is far from perfect and there are a lot of
issues that can be found.  Inspired by economics, the challenges in
{\bibtex} can be divided into structural and conjunctural issues. 

\begin{itemize}
\item The structural issues are the ones intrinsic to {\bibtex}, thus
  it is caused by the design, the standard tools for {\bibtex} and
  information that can not be found.

\item The conjunctural issues are the combination of circumstances,
  for instance if the source used does not contain complete
  information (\eg, extracting a references from an article where the
  author only have initials even though the full name is known) or the
  users having other priorities than bibliographies.
\end{itemize}

Whether an issue is seen as conjunctural or structural is in part a
matter of opinion, as the definitions can be stretched in either
direction - an issue also faced in economics.  Most of the issues are
arguably a combination the two as they could be fixed by careful labor
or by having the right tools available, \eg, most bibliography
managers have tools to switch between abbreviated and full journal
names. 

Addressing the human factor, \ie, a conjunctural solution, is one
theoretically possibile way of solving these issues.  ``Simply''
motivating people to do things right.  Alas people are not machines
and thus this approach will be impossible in practice, for most people
the interest is not the tools they use, but what they use them for.
The interest in {\bibtex} will for most people be to ensure that their
documents contain the relevant references.

Since a conjunctural solution is not realistic, the goal is to provide
a structural solution to the issues.  In the perfect world the
structural solution will be so complete that if there is any issues
left then they will be entirely conjunctural, because one is either:
not using, or misusing the structural solution.


\section{De-facto standards and specification conformity}
\label{sec:problems_de_facto}

An interesting point is that not all the structural issues are bad.
There are practical ways to use the relaxed properties of {\bibtex}.
For instance {\bibtex} ignores unknown tags by design which is useful
in de-facto standards such as commenting entries out by prefixing with
$OPT$ or adding information that are not a part of the {\bibtex}
specification, such as ISSN and DOI.  The $crossref$ tag is
technically not specified in {\bibtex}, but still is part of the tool.
In \figref{fig:mendeley_output}, an example is provided by a PhD
student from the Chemistry Department at Aarhus University, this
example is created from Mendeley (see
Section~\ref{sec:related_mendeley}) and shows a lot of additional
information about the article.

\begin{figure}
  \centering
\begin{small}
\begin{verbatim}
@article{Acatrinei2003,
author = {Acatrinei, Alice I and Browne, D and Losovyj, Y B
          and Young, D P and Moldovan, M and Chan, Julia Y
          and Sprunger, P T and Kurtz, Richard L},
doi = {10.1088/0953-8984/15/33/101},
file = {:C$\backslash$:/Users/[...]pdf},
issn = {0953-8984},
journal = {Journal of Physics: Condensed Matter},
month = {aug},
number = {33},
pages = {L511--L517},
title = {{Angle-resolved photoemission study
          and first-principles calculation
          of the electronic structure of LaSb 2}},
url = {http://iopscience.iop.org/[...]},
volume = {15},
year = {2003}
}
\end{verbatim}
\end{small}
  \caption{Output from Mendeley containing additional information}
\label{fig:mendeley_output}
\end{figure}

This design choice is an issue, if strict conformity to the
specification is desired, however as it is a very practical and widely
used feature a strict validation would be counterproductive.  Also
some formatting styles actually make use of some of the unspecified
tags as can be seen in \figref{fig:entry_with_issn}. It is still
interesting to find tags that are not desired, \ie, tags that do not
conform to the specification and the de-facto standards in use.

% Some issues such as spelling errors are by nature conjunctural, these
% errors could be prevented by having a watchful eye and doing a spell
% check of a bib file.

%%
%% Internal ref 4
%%
\section{Journal abbreviations}
\label{sec:problems_abbreviations}

\remark{Wouldn't forum abbreviations be more appropriate?}

\remark{Hmm, comment that abbreviations follow standards}

Most, if not all, journals require that journal names should be
abbreviated when publishing.  However internally in the {\bibtex} file
the owner's personal priorities are: consistent and correct naming.
As {\bibtex} can be seen as a database of references, it makes sense
to consider full names as correct and the abbreviations to be a matter
of formatting.  Unfortunately {\bibtex} does not handle abbreviations
at all, which for instance is apparent in articles from arXiv.org, as
can be seen in the bbl output in \figref{fig:inconsistent_naming}.

\begin{figure}
  \centering
\begin{small}
\begin{verbatim}
\bibitem[\protect\citename{Baroni \bgroup et al.\egroup }2014b]
          {baroni2014don}
          Marco Baroni, Georgiana Dinu, and Germ{\'a}n Kruszewski.
\newblock 2014b.
\newblock Don't count, predict!
          a systematic comparison of context-counting vs.
          context-predicting semantic vectors.
\newblock In {\em Proceedings of the 52nd Annual Meeting of
          the Association for Computational Linguistics},
          volume~1, pages 238--247.

\bibitem[\protect\citename{Bruni \bgroup et al.\egroup}2014]
          {bruni2014multimodal}
          Elia Bruni, Nam-Khanh Tran, and Marco Baroni.
\newblock 2014.
\newblock Multimodal distributional semantics.
\newblock {\em J. Artif. Intell. Res. (JAIR)}, 49:1--47.

[...]

\bibitem[\protect\citename{Collobert \bgroup et al.\egroup}2011]
          {collobert2011natural}
          Ronan Collobert, Jason Weston, L{\'e}on Bottou,
          Michael Karlen, Koray Kavukcuoglu, and Pavel Kuksa.
\newblock 2011.
\newblock Natural language processing (almost) from scratch.
\newblock {\em The Journal of Machine Learning Research},
          12:2493--2537.

[...]

\bibitem[\protect\citename{Kalchbrenner \bgroup et al.\egroup}2014]
          {kalchbrenner2014convolutional}
          Nal Kalchbrenner, Edward Grefenstette, and Phil Blunsom.
\newblock 2014.
\newblock A convolutional neural network for modelling sentences.
\newblock In {\em Proceedings of EMNLP}.
\end{verbatim}
\end{small}
  \caption{Inconsistent naming of journal and conference names}
\label{fig:inconsistent_naming}
\end{figure}

From the point of view that the style of {\bibtex} should format
abbreviations properly, the issue is structural.  In cases where the
abbreviation is wrong (\eg, due to a typo), the issue moves towards
being conjunctural, unless some kind of abbreviation specific spell
checker is being used.  Using full names and then formatting them
accordingly is the most sensible idea, since the style of abbreviation
could be interchanged, should the need arise, it is more readable and
it would create better conditions for output tools to provide
consistent formatting.

Currently there are multiple strategies for ensuring consistency in
abbreviations, some do a search and replace on the bib-file, an
approach that is a bit more structured is the use of strings in
{\bibtex} to ensure a consistent naming of a journal which can further
be combined with the usage of crossref.  Another approach is the use
of Bib{\LaTeX} and biber, which provide the solution in the formatting
options\cite{koppensteiner2011abbreviate}, provided that the
abbreviation handing of the style is correct.  This causes the
formatting issue to become a conjunctural, as the issue will then be
is if the correct names are written, however how conjunctural depends
on the correctness of the tools.  Bibliography managers (see
Section~\ref{sec:bibliography_managers}) such as JabRef, Mendeley
etc. tend to go with the strategy that it stores the references using
full names and then exports it to a {\bibtex} file (or whatever format
one decide to export to) having the desired abbreviation style applied
to the export, this strategy moves the issue towards being
conjunctural, for the same reasons as the Bib{\LaTeX} and biber
solution.

As the purpose is to work on the {\bibtex} files, the formatting in
the end is technically not the primary concern.  The concern would
optimally just be to ensure a consistent document, so the style can do
its work.  Since {\bibtex} styles currently does not take care of
abbreviations directly, there might be the need for considerations on
how to easily and consistently ensuring abbreviations according to a
desired format.  Having a solution that ensures a consistent
structure, which is easy to modify to a desired style of
abbreviations, would move the issue towards the conjunctural.  Having
the styles actually handling the abbreviations (such as the
Bib{\LaTeX} and biber combination does) would make the issue entirely
conjunctural if there is a an easy way to ensure that all journal
names, conference names etc. are always in correct full names.

%%
%% Internal ref 1
%% journal unknown oO
%%
\section{Inconsistent tags}
\label{sec:problems_inconsistent_tags}

Take the inconsistency in \figref{fig:entry_with_issn}, found in a
article on arXiv.org: two references from the same conference, but
different years.  The inconsistency is easy to identify due to the
consistent content, which will make the issue even more visible to the
reader.  Correct and consistent content will help tools in detecting
inconsistencies.  This exposes a structural part of the issue, as no
such tools exist (to the authors knowledge).

\begin{figure}
  \centering
  \begin{small}
\begin{verbatim}
\bibitem[Bernardy and Claessen(2015)]{bernardy_efficient_2015}
J.-P. Bernardy and K.~Claessen.
\newblock Efficient parallel and incremental parsing
          of practical context-free languages.
\newblock \emph{J. of Funct. Prog.}, 25, 2015.
\newblock ISSN 1469-7653.
\newblock \doi{10.1017/S0956796815000131}.

[...]

\bibitem[Mu et~al.(2009)Mu, Ko, and Jansson]{MuKoJansson2009AoPA}
S.-C. Mu, H.-S. Ko, and P.~Jansson.
\newblock Algebra of programming in {Agda}:
          dependent types for relational program derivation.
\newblock \emph{J. Funct. Program.}, 19:\penalty0 545--579, 2009.
\newblock \doi{10.1017/S0956796809007345}.
\end{verbatim}
  \end{small}
  \caption{Additional tag ISSN is provided in one of the entries}
\label{fig:entry_with_issn}
% consider the pages
\end{figure}

The ISSN might not exist for the $MuKoJansson2009AoPA$-entry, in this
case a structural detection system might be the cause of new
structural issue, either the removal of relevant data or forcing
entries when the data does not exist.  In this specific case the
search result in \figref{fig:entry_issn_found} reveals that the
missing ISSN does exist and thus a tool pointing out the inconsistency
would in this case make the issue conjunctural.

% Source http://journals.cambridge.org/action/displayAbstract?fromPage=online&aid=6171388&fileId=S0956796809007345#
\begin{figure}
  \centering
\begin{verbatim}
@article{Mu:2009:APA:1630623.1630627,
 author = {Mu, Shin-cheng and Ko, Hsiang-shang 
           and Jansson, Patrik},
 title = {Algebra of Programming in Agda: 
          Dependent Types for Relational Program Derivation},
 journal = {J. Funct. Program.},
 issue_date = {September 2009},
 volume = {19},
 number = {5},
 month = sep,
 year = {2009},
 issn = {0956-7968},
 pages = {545--579},
 numpages = {35},
 url = {http://dx.doi.org/10.1017/S0956796809007345},
 doi = {10.1017/S0956796809007345},
 acmid = {1630627},
 publisher = {Cambridge University Press},
 address = {New York, NY, USA},
 }
\end{verbatim}
  \caption{Search revealing the ISSN}
\label{fig:entry_issn_found}
\end{figure}

%%
%% Internal ref 2
%% Source Electronic Proceedings in Theoretical Computer Science
%%

Provided a reliable way to look up correct entries a tool could move
the issues towards being conjunctural.  Take the inconsistency in
\figref{fig:inconsistent_proceedings} where two entries from the same
conference has different information.  One has an additional ``ICFP
'10'' and ``ACM'' in there, the other one does not.

\begin{figure}
  \centering
  \begin{small}
\begin{verbatim}
\bibitem[Bernardy and Claessen(2013)]{bernardy_efficient_2013}
J.-P. Bernardy and K.~Claessen.
\newblock Efficient divide-and-conquer parsing 
          of practical context-free languages.
\newblock In \emph{Proc. of ICFP 2013}, pages 111--122, 2013.

[...]

\bibitem[Danielsson(2010)]{danielsson_total_2010}
N.~A. Danielsson.
\newblock Total parser combinators.
\newblock In \emph{Proc. of ICFP 2010}, ICFP '10,
          pages 285--296. ACM, 2010.
\end{verbatim}
  \end{small}
  \caption{Capt}
\label{fig:inconsistent_proceedings}
\end{figure}

An online search for {\bibtex} information give the entries in
\figref{fig:missing_org_scholar_lookup} for the two articles, which
provides one possible option for a set of consistent entries.  As can
be seen $ACM$ is the name of the organization and is probably missing
in the original {\bibtex} that produced the bbl file inspected above.
The Danielsson has additional information with the content ``ICFP
'10'', which is not apparent in the search result.

\begin{figure}
  \centering
\begin{verbatim}
@inproceedings{bernardy2013efficient,
  title={Efficient divide-and-conquer parsing
         of practical context-free languages},
  author={Bernardy, Jean-Philippe and Claessen, Koen},
  booktitle={ACM SIGPLAN Notices},
  volume={48},
  number={9},
  pages={111--122},
  year={2013},
  organization={ACM}
}

@inproceedings{danielsson2010total,
  title={Total parser combinators},
  author={Danielsson, Nils Anders},
  booktitle={ACM Sigplan Notices},
  volume={45},
  number={9},
  pages={285--296},
  year={2010},
  organization={ACM}
}
\end{verbatim}
  \caption{Scholar lookup}
\label{fig:missing_org_scholar_lookup}
\end{figure}

\section{Duplicate entries}
\label{sec:problems_duplicates}

When a group of authors are writing on the same document work is often
divided and each author will have their own bib-file that they use for
their references.  All might be fine until the group combines their
document and it turns out that multiple authors has used the same
reference but with different keys, ending up having duplicate entries
in the bibliography.

Arguably it can be seen as a structural or a conjunctural problem.
The similarity of such entries will make them easier to spot with the
naked eye, arguably making a conjunctural solution relevant.  However
the due to the similarity (at least if abbreviated the same way) a
structural solution is arguably also easy to make.  Making the problem
purely conjunctural would require a way to detect and merge these
entries, furthermore if the entries use different key names the
corresponding document should also be corrected. A challenge in this
case is when similar entries, that are still different entries, occur.

A similar issue, which actually happened in one of the drafts for this
document, is when an author copy and pastes a template for en entry
with the intend to adjust for a new entry.  Then giving it a key and
forgetting to adjust the entry.  This will also result in a duplicate
entry, but unlike the situation above the duplicate should not be
merged but contain different content.

Having a tool that just merges duplicate entries may provide a
structural solution to the first case but will create a structural
issue in the second as it will automatically introduce new issues.
Thus the solution should not merge automatically.


\section{Online look ups}
\label{sec:problems_look_ups}

In the utopic case all entries could be looked up at all times.  Even
though the databases out there are really good, erroneous results can
be found.  A lookup on Google Scholar in the beginning of February
2016 for: ``Results and Analysis of SyGuS-Comp’15'' can be seen in
\figref{fig:scholar_bad_result}, which contains an erroneous output.

\begin{figure}
  \centering
\begin{verbatim}
@article{alurresults,
  title={Results and Analysis of SyGuS-Comp’15},
  author={Alur, Rajeev and Fisman, Dana and Singh, Rishabh
          and Solar-Lezama, Armando}
}
\end{verbatim}
  \caption{Bad result from Google Scholar}
\label{fig:scholar_bad_result}
\end{figure}

Having found the article originally on arXiv.org the source of the
article is known to be EPTCS - Electronic Proceedings in Theoretical
Computer Science.  So not only does the Google Scholar result actually
not conform to the requirements of an article, the resource are in
fact not an article at all, but in the proceedings to a conference.
Finding the correct entry details at the EPTCS page reveals the entry
in \figref{fig:eptcs_lookup}.  Relying blindly on these being correct
causes a structural issue as the tools could automatically introduce
new errors.

The entries in those search engines cannot account for unpublished
work either, and expecting all published work to be represented in the
databases would be naive.  Having a reliable way to ensure that all
entries are present, is not a likely scenario.

Apart from the desire to detect erroneous entries from bad look ups,
using look ups for suggestions to most of the other issues can be a
useful part of a structural solution.  An optimal structural solution
to the bad look ups would be if it was possible to detect the users
intended result.  As there is no way to know for certain what the user
intends this is an Utopian idea.  A realistic idea would be to provide
a solution that limits the risk of bad look ups which is not only a
partial structural solution as the user would still have to verify.

\begin{figure}
  \centering
\begin{small}
\begin{verbatim}
@Inproceedings{EPTCS202.3,
  author    = "Alur, Rajeev and Fisman, Dana  and Singh, Rishabh 
               and Solar-Lezama, Armando",
  year      = "2016",
  title     = "Results and Analysis of SyGuS-Comp'15",
  editor    = "\v{C}ern\'y, Pavol and Kuncak, Viktor 
               and Parthasarathy, Madhusudan"b,
  booktitle = "{\rm Proceedings Fourth Workshop on}
               Synthesis,
               {\rm San Francisco, CA, USA, 18th July 2015}",
  series    = "Electronic Proceedings in 
               Theoretical Computer Science",
  volume    = "202",
  publisher = "Open Publishing Association",
  pages     = "3-26",
  doi       = "10.4204/EPTCS.202.3",
}
\end{verbatim}
\end{small}
  \caption{Correct lookup on EPTCS, after failed lookup on Google Scholar}
\label{fig:eptcs_lookup}
\end{figure}


\section{Inconsistent entry keys}
\label{sec:problems_inconsistent_keys}

The naming scheme for entry keys may vary throughout a bib-file, for
instance one of the users collaborating earlier might use entries from
various online databases getting keys corresponding to
\figref{fig:missing_org_scholar_lookup}, \figref{fig:eptcs_lookup} and
other structures in one big mess.  One of the users collaborating
might also be new to using {\bibtex} (could be a student learning) and
need to find a nice an consistent way of writing the keys.

A challenge could be to avoid duplicate key names, which with a
consistent structure is more likely.  The duplicate entry keys can
have the advantage that it can be an indicator of a duplicate entry
Section~\ref{sec:problems_look_ups}.

Inconsistencies in keys might range from not a problem at all to fully
conjectural or structural, highly depending on the one's point of
view.  A user may simply not care and apart from the potential to
detect duplicate entries may not feel a reason to.  In the
collaboration scenario the additional way of detecting duplicates may
be valuable.  In this case it arguably becomes conjunctural as the
authors should have agreed on a style - which might also have helped a
newcomer to have a good practice, but it can yet again be argued that
it is structural as {\bibtex} does not provide naming guidelines nor
tools for ensuring those.

To make this issue structural a naming convention would have to be
specified and some way of detecting deviations provided.  This could
result in two entries with the same key, causing a key name clash,
provided a that a solution for the duplicate entry issue is already in
place a need for way to handle key name clashes.


% ref 3
\section{Name changes in conferences, journals etc.}
\label{sec:problems_name_changes}

In \figref{fig:missing_org_scholar_lookup} spotting the consistency
issues were relatively simple.  When looking at
\figref{fig:entry_journal_name_authors} it can be seen that the
conference name is slightly different in one of the entries, but so
close that they are probably the same conference.

\begin{figure}
  \centering
\begin{small}
\begin{verbatim}
\bibitem{stanifordchen96grids}
S.~S.-C. \emph{et al}.
\newblock {GrIDS} -- {A} graph-based intrusion detection system 
          for large networks.
\newblock In {\em Proceedings of the 19th
          National Information Systems Security Conference},
          1996.

[...]

\bibitem{porras97emerald}
P.~A. Porras and P.~G. Neumann.
\newblock {EMERALD}: Event monitoring enabling responses 
          to anomalous live disturbances.
\newblock In {\em Proc. 20th {NIST}-{NCSC}
          National Information Systems Security Conference},
          pages 353--365, 1997.

\end{verbatim}
\end{small}
  \caption{Inconsistent reference to the conference and heavily abbreviated author names}
\label{fig:entry_journal_name_authors}
\end{figure}

A visit to the homepage of the conference reveals that ``National
Information Systems Security Conference'' used to be named ``National
Computer Security Conference'', which is probably the reason for the
$\{NIST\}--\{NCSC\}$ part of the first entry\cite{nist2014_nissc}.  In
the same source it turns out that there are also references to the old
conference name, as seen in \figref{fig:conference_name}, so to
correctly identify potential inconsistencies it should also recognize
name changes and variations.

\begin{figure}
  \centering
\begin{small}
\begin{verbatim}
\bibitem{snapp91dids}
S.~R.~S. \emph{et al}.
\newblock {DIDS} (distributed intrusion detection system) -
          motivation, architecture, and an early prototype.
\newblock In {\em Proceedings of the 14th
          National Computer Security Conference},
          pages 167--176, Washington, DC, 1991.
\end{verbatim}
\end{small}
  \caption{Name change of a conference}
\label{fig:conference_name}
\end{figure}

In {\bibtex} there is no way of specifying that the same conference
has different names so there is a big structural part as there is no
support for identifying this issue, furthermore the owner of the
{\bibtex} file might not even be aware which arguably could be
conjunctural as the user could do his research or structural because
the user should have a tool that can assist.  If a tool could reliably
detect inconsistencies from the same forum with different names the
name changes would become conjunctural.


\section{Initials}
\label{sec:problems_initials}

Another issue in \figref{fig:conference_name} is the list of author
names that are so heavily abbreviated to initials that one cannot
realistically distinguish who the authors are.  This might originate
from the used citation style or from a resource where they are already
abbreviated.

If the initials come from the citation style the issue is of a
structural nature.  However if it is copied from another source, or
just written like that, it is conjunctural, as the user did not ensure
full names.  A structural variant of the copying issue is if the
reference is copied by a tool, then it can be argued that such a tool
should try to detect initials.  Having a way to detect the initials
and the full names will provide a structural solution.  However to
make an entirely structural solution a reliable way to know when the
initials are deliberate or not is needed.


\section{Spelling errors}
\label{sec:problems_spelling}

In almost, if not, all cases where people type text, spelling errors
will arise, sooner or later, bib-files is no exception.  Extracting
meta-information automatically from documents may also run into a bad
extraction leading to spelling errors.  Furthermore a spelling error
might not even be in the {\bibtex} document, as the misspelling can be
from the source, for instance if a title of a document is misspelled,
then the correct {\bibtex} will have to contain this error, as it is
the title of the source.

A user typing the entries manually is arguably the cause of spelling
errors and thus it can be argued that the issue is conjunctural, since
one could just do a spell check.  However one might also argue that
{\bibtex} should do a spell check.  In the case where automatic
extraction is the source of the spelling error the tool will give a
structural component to the issue.  Having a way of ensuring a spell
check and if such a way could ensure that the spelling error
corresponds to the source would make this issue conjunctural.


\section{{\bibtex} strings ending up as text}
\label{sec:problems_strings_as_text}

When working with {\bibtex}, strings can accidentally end up being
textual content rather than a string, for instance if a user is being
careless or if the user, as the PhD student earlier, exports from a
program that does not make use of the strings.  In the output from the
aforementioned student seen in \figref{fig:mendeley_output} the month
is actually a text and not a string as one would expect.  The use of
text over strings prevents re-use of fragments and localization.

When the usage as text over strings come from a tool, such as in the
example above, the issue is of a structural nature, whereas when one
types it by accident then again it can arguably be both structural and
conjunctural.  Providing a structural solution to this would be by
being able to detect this and correct it.


\section{Summary and conclusions}
\label{sec:problems_conclusion}

{\bibtex} has a lot of issues which can be analyzed from the
perspective of if they are structural and conjunctural with the goal
of how the issues can become more conjunctural.  The problems covered
are:

\begin{itemize}
\item For de-facto standards and specification conformance desired to
  ensure that entries conform to the {\bibtex} specification, but
  allowing de-facto standards.  A structural solution requires
  something that checks the conformity to a combination of
  specifications and de-facto standards.

\item Journal abbreviations where they can be cumbersome for analyzing
  tools and that {\bibtex} does not handle abbreviations fully.  A
  structural solution will be able to ensure that all entries are
  consistently abbreviated or de-abbreviated, preferably accounting
  for that people may want to switch between for formats.

\item Inconsistent tag usage in similar entries causes messy
  bibliographies. A structural solution will detect these
  inconsistencies and be able so suggest a course of action, the
  solution may be challenged by deviations in information and forums
  that change name.

\item Duplicate entries is not desired.  A structural solution will be
  to detect and merge duplicates, as a duplicate come from missing or
  wrong data automatically merging may cause issues.

\item Online look ups which can contain wrong data.  Optimally we
  would be able to detect this reliably, the most realistic idea is to
  provide a way that limits the likelihood of erroneous results.

\item Inconsistent entry keys which can be an issue in collaboration
  and may make it harder to detect duplicates.  A structural solution
  could be to apply a naming scheme for the entry keys, accounting for
  similar entries that would result in identical entry keys.

\item Name change of a forum which can affect detection of
  inconsistent tag use.  A structural solution will have some way of
  detecting when sources are from the same forum, either through a
  database or some configuration options.

\item Use of initials can hide peoples names to the level where
  finding a resource is hard.  A structural solution is to detect
  initials and the full names, accounting for when the initials are
  deliberate.

\item Spelling errors.  A structural solution will be able to find and
  correct the spelling errors, preferably by having the same spelling
  as the published version.  Alternatively running a spell checker
  having a way to account for deliberate misspellings, domain specific
  words and choice of language.

\item {\bibtex} strings that end up as part of the text, in the worst
  case resulting in wrong data.  A structural solution will be able to
  detect and make text into strings when a string is desired, for same
  cases such as month this is trivial, but inside texts a word sharing
  the name of a string can just be a coincidence.
\end{itemize}

With all these problems at hand, {\bibtex} may not seem like the
optimal tool after all.  Being the widely used tool that it is, doing
something about it quite desirable.  But what can be done with such a
range of issues?
