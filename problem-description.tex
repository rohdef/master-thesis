Inspired by economics, the challenges in Bib{\TeX} can be divided into
structural and conjunctural issues.  The structural issues are the
ones intrinsic to Bib{\TeX}, thus it is caused by the design of and
the standard tools for Bib{\TeX} and information that can not be
found.  The conjunctural issues are the combination of circumstances,
for instance if the source used not having the complete information
(eg. extracting a references from an article where the author only
have initials even though the name is known) or the users having other
priorities than be bibliographies.

Whether an issue is seen as conjectural or structural is in part a
matter of opinion, as the difference can be stretched in either
direction - an issue also faced in economics.  Most of the issues are
arguably a combination the two as they could be fixed by careful labor
or by having the right tools available, eg. most bibliography managers
have tools to switch between abbreviated and full journal names.  The
ultimate goal of {\Orangutan} is essentially to move the issues to the
point where they are purely conjectural as it would be a matter of
using the tool to ensure a well structured Bib{\TeX} document.

An interesting point is that not all of the structural issues are
necessarily bad.  As pointed out earlier there are a lot of ways to
make use of the relaxed properties of Bib{\TeX}, for instance that it
ignores unknown tags is useful in de facto standards such as
commenting entries out by prefixing with $OPT$ or adding information
that are not a part of the Bib{\TeX} specification, such as ISSN, DOI
or the crossref, which is technically not specified by Bib{\TeX} but
still is part of the tool.  In figure-\ref{fig:mendeley_output}
an example is provided by a PhD student from the Chemistry Department
at Århus University, this example is created from Mendeley and shows a
lot of additional information about the article.

\begin{figure}[ht]
  \centering
\begin{small}
\begin{verbatim}
@article{Acatrinei2003,
author = {Acatrinei, Alice I and Browne, D and Losovyj, Y B 
          and Young, D P and Moldovan, M and Chan, Julia Y
          and Sprunger, P T and Kurtz, Richard L},
doi = {10.1088/0953-8984/15/33/101},
file = {:C$\backslash$:/Users/[...]pdf},
issn = {0953-8984},
journal = {Journal of Physics: Condensed Matter},
month = {aug},
number = {33},
pages = {L511--L517},
title = {{Angle-resolved photoemission study 
          and first-principles calculation 
          of the electronic structure of LaSb 2}},
url = {http://iopscience.iop.org/[...]},
volume = {15},
year = {2003}
}
\end{verbatim}
\end{small}
  \caption{Output from Mendeley containing additional information}
  \label{fig:mendeley_output}
\end{figure}

This is an issue, if strict conformity to the specification is desired,
however as this is a very practical and widely used feature a strict
validation would be counterproductive.  Also some formatting styles
actually make use of some of the unspecified tags as can be seen in
figure-\ref{fig:entry_with_issn}. It is still interesting to find tags
that does not conform to the specification and the de facto standards
in use, thus treating de facto standards as a feature and only treating
further deviations as actual issues seems to make most sense. 

% Some issues such as spelling errors are by nature conjunctural, these
% errors could be prevented by having a watchful eye and doing a spell
% check of a bib file.

%%
%% Internal ref 4
%%

Most, if not all, journals require that journal names should be
abbreviated when publishing.  However internally in the Bib{\TeX} file
the owner's personal priorities are consistent and correct naming.  As
Bib{\TeX} can be seen as a database of references, it would make most
sense to consider full names as the most correct and the abbreviations
to be a matter of formatting.  Unfortunately Bib{\TeX} does not handle
abbreviations at all, which for instance is apparent in articles from
arXiv.org, as can be seen in the bbl output in
figure-\ref{fig:inconsistent_naming}.

\begin{figure}[ht]
  \centering
\begin{small}
\begin{verbatim}
\bibitem[\protect\citename{Baroni \bgroup et al.\egroup }2014b]
          {baroni2014don}
          Marco Baroni, Georgiana Dinu, and Germ{\'a}n Kruszewski.
\newblock 2014b.
\newblock Don't count, predict!
          a systematic comparison of context-counting vs.
          context-predicting semantic vectors.
\newblock In {\em Proceedings of the 52nd Annual Meeting of
          the Association for Computational Linguistics},
          volume~1, pages 238--247.

\bibitem[\protect\citename{Bruni \bgroup et al.\egroup}2014]
          {bruni2014multimodal}
          Elia Bruni, Nam-Khanh Tran, and Marco Baroni.
\newblock 2014.
\newblock Multimodal distributional semantics.
\newblock {\em J. Artif. Intell. Res. (JAIR)}, 49:1--47.

[...]

\bibitem[\protect\citename{Collobert \bgroup et al.\egroup}2011]
          {collobert2011natural}
          Ronan Collobert, Jason Weston, L{\'e}on Bottou,
          Michael Karlen, Koray Kavukcuoglu, and Pavel Kuksa.
\newblock 2011.
\newblock Natural language processing (almost) from scratch.
\newblock {\em The Journal of Machine Learning Research},
          12:2493--2537.

[...]

\bibitem[\protect\citename{Kalchbrenner \bgroup et al.\egroup}2014]
          {kalchbrenner2014convolutional}
          Nal Kalchbrenner, Edward Grefenstette, and Phil Blunsom.
\newblock 2014.
\newblock A convolutional neural network for modelling sentences.
\newblock In {\em Proceedings of EMNLP}.
\end{verbatim}
\end{small}
  \caption{Inconsistent naming of journal and conference names}
\label{fig:inconsistent_naming}
\end{figure}

From the point of view that the style of Bib{\TeX} should format
abbreviations properly the issue seen here is structural, as the long
names should have been formatted by the tool.  In cases where the
abbreviation is wrong (eg. due to a typo) the issue would move towards
being conjectural again, unless some kind of abbreviation specific
spell checker is being used.  Using full names and then formatting
them accordingly seems like the most sensible idea, since the style of
abbreviation could be interchanged, should the need arise, it is more
readable and it would create better conditions for output tools to
provide consistent formatting.

Currently there are multiple strategies for ensuring consistency in
abbreviations, some do a complete search and replace on the bib-file
depending on the formatting they need, a bit more structured approach
is the use of strings in Bib{\TeX} to ensure a consistent naming of a
journal which can further be combined with the use of crossref.
Another approach is the use of Bib{\LaTeX} and biber, which provides
the solution in the formatting
options\cite{koppensteiner2011abbreviate}, provided that the
abbreviation handing of the style is correct.  This causes it to
become a more conjectural issue, as the issue now is if the right
names are written, however how conjectural depends on the correctness
of the tools.  Reference mangers such as JabRef, Mendeley etc. tend to
go with the strategy that it stores the references using full names
and then export it to a Bib{\TeX}-file (or whatever format you decide
to export to) having the desired abbreviation style applied to the
export, this again moves the issue towards being conjectural, for the
same reasons as the Bib{\LaTeX} and biber solution.

As the purpose of {\Orangutan} is to work on the Bib{\TeX} file
itself, the formatting in the end is technically not the primary
concern.  The concern would optimally just be to ensure a consistent
document, so the style can do its work.  Since Bib{\TeX} styles
currently does not take care of abbreviations directly, there might be
the need for considerations on how to easily and consistently ensuring
abbreviations according to a desired format.  Having a solution that
ensures a consistent structure, which is easy to modify to a desired
style of abbreviations, would move the issue towards the conjectural.
Having the styles actually handling the abbreviations (such as the
Bib{\LaTeX} and biber combination does) would make the issue entirely
conjectural if there is a an easy way to ensure that all journal
names, conference names etc. are always in correct full names.

%%
%% Internal ref 1
%% journal unknown oO
%%

Take the inconsistency in figure-\ref{fig:entry_with_issn}, found in a
article on arXiv.org, have two references from the same conference,
but different years.  The inconsistency is easy to identify due to the
consistent naming, which in turn affects how the mistake is percieved.
Correct and consistent naming will make it possible for tools to
detect inconsistencies when some fields are missing.  This exposes a
structural part of the issue, as no such tools exists (to the authors
knowledge).

\begin{figure}[ht]
  \centering
  \begin{small}
\begin{verbatim}
\bibitem[Bernardy and Claessen(2015)]{bernardy_efficient_2015}
J.-P. Bernardy and K.~Claessen.
\newblock Efficient parallel and incremental parsing
          of practical context-free languages.
\newblock \emph{J. of Funct. Prog.}, 25, 2015.
\newblock ISSN 1469-7653.
\newblock \doi{10.1017/S0956796815000131}.

[...]

\bibitem[Mu et~al.(2009)Mu, Ko, and Jansson]{MuKoJansson2009AoPA}
S.-C. Mu, H.-S. Ko, and P.~Jansson.
\newblock Algebra of programming in {Agda}:
          dependent types for relational program derivation.
\newblock \emph{J. Funct. Program.}, 19:\penalty0 545--579, 2009.
\newblock \doi{10.1017/S0956796809007345}.
\end{verbatim}
  \end{small}
  \caption{Additional tag ISSN is provided in one of the entries}
\label{fig:entry_with_issn}
% consider the pages
\end{figure}

It might be the case that an ISSN simply does not exist for the
$MuKoJansson2009AoPA$-entry, in this case a structural detection
system might be the cause of new structural issue, either the removal
of relevant data or forcing entries for cases where it is not
relevant.  In this specific case the search result in
figure-\ref{fig:entry_issn_found} reveals that the missing ISSN do
exist and thus a tool pointing out the inconsistency would in thise
case make the issue conjunctional.

% Source http://journals.cambridge.org/action/displayAbstract?fromPage=online&aid=6171388&fileId=S0956796809007345#
\begin{figure}[ht]
  \centering
\begin{verbatim}
@article{Mu:2009:APA:1630623.1630627,
 author = {Mu, Shin-cheng and Ko, Hsiang-shang 
           and Jansson, Patrik},
 title = {Algebra of Programming in Agda: 
          Dependent Types for Relational Program Derivation},
 journal = {J. Funct. Program.},
 issue_date = {September 2009},
 volume = {19},
 number = {5},
 month = sep,
 year = {2009},
 issn = {0956-7968},
 pages = {545--579},
 numpages = {35},
 url = {http://dx.doi.org/10.1017/S0956796809007345},
 doi = {10.1017/S0956796809007345},
 acmid = {1630627},
 publisher = {Cambridge University Press},
 address = {New York, NY, USA},
 }
\end{verbatim}
  \caption{Search revealing the ISSN}
\label{fig:entry_issn_found}
\end{figure}

%%
%% Internal ref 2
%% Source Electronic Proceedings in Theoretical Computer Science
%%

Provided a reliable way to look up correct entries a tool could move
the issues towards being conjuntural, since it would then just being a
matter of doing the lookup on all entries and then be done (apart from
the abbreviation issues above).  In the utopic case all entries could
be looked up at all times, but this is not likely as unpublished works
would need a reliable way to be registered.  Also even though the
databases out there are really good, most of the time, erronous
results can be found.  A lookup on Google Scholar in the beginning of
February 2016 for: ``Results and Analysis of SyGuS-Comp’15'' can be
seen in figure-\ref{fig:scholar_bad_result}.

\begin{figure}[ht]
  \centering
\begin{verbatim}
@article{alurresults,
  title={Results and Analysis of SyGuS-Comp’15},
  author={Alur, Rajeev and Fisman, Dana and Singh, Rishabh
          and Solar-Lezama, Armando}
}
\end{verbatim}
  \caption{Bad result from Google Scholar}
\label{fig:scholar_bad_result}
\end{figure}

Having found the article originally on arXiv.org the source of the
article is known to be EPTCS - Electronic Proceedings in Theoretical
Computer Science.  So not only does the Google Scholar result actually
not conform to the requirements of an article, the resource are in
fact not an article at all, but in the proceedings to a conference.
Finding the correct entry details at the EPTCS page reveals the entry
in figure-\ref{fig:eptcs_lookup}.  Relying on these being correct
would cause structural issues as it would the tools would now
automatically introduce new errors.

\begin{figure}[ht]
  \centering
\begin{small}
\begin{verbatim}
@Inproceedings{EPTCS202.3,
  author    = "Alur, Rajeev and Fisman, Dana  and Singh, Rishabh 
               and Solar-Lezama, Armando",
  year      = "2016",
  title     = "Results and Analysis of SyGuS-Comp'15",
  editor    = "\v{C}ern\'y, Pavol and Kuncak, Viktor 
               and Parthasarathy, Madhusudan"b,
  booktitle = "{\rm Proceedings Fourth Workshop on}
               Synthesis,
               {\rm San Francisco, CA, USA, 18th July 2015}",
  series    = "Electronic Proceedings in 
               Theoretical Computer Science",
  volume    = "202",
  publisher = "Open Publishing Association",
  pages     = "3-26",
  doi       = "10.4204/EPTCS.202.3",
}
\end{verbatim}
\end{small}
  \caption{Correct lookup on EPTCS, after failed lookup on Google Scholar}
\label{fig:eptcs_lookup}
\end{figure}

% mark

\begin{figure}[ht]
  \centering
  \begin{small}
\begin{verbatim}
\bibitem[Bernardy and Claessen(2013)]{bernardy_efficient_2013}
J.-P. Bernardy and K.~Claessen.
\newblock Efficient divide-and-conquer parsing 
          of practical context-free languages.
\newblock In \emph{Proc. of ICFP 2013}, pages 111--122, 2013.

[...]

\bibitem[Danielsson(2010)]{danielsson_total_2010}
N.~A. Danielsson.
\newblock Total parser combinators.
\newblock In \emph{Proc. of ICFP 2010}, ICFP '10,
          pages 285--296. ACM, 2010.
\end{verbatim}
  \end{small}
  \caption{Capt}
\label{fig:inconsistent_proceedings}
\end{figure}

A search for Bib{\TeX} entries give then entries in
figure-\ref{fig:missing_org_scholar_lookup} for the two articles, which
provides one possible option for a set of consistent entries.

\begin{figure}[ht]
  \centering
\begin{verbatim}
@inproceedings{bernardy2013efficient,
  title={Efficient divide-and-conquer parsing
         of practical context-free languages},
  author={Bernardy, Jean-Philippe and Claessen, Koen},
  booktitle={ACM SIGPLAN Notices},
  volume={48},
  number={9},
  pages={111--122},
  year={2013},
  organization={ACM}
}

@inproceedings{danielsson2010total,
  title={Total parser combinators},
  author={Danielsson, Nils Anders},
  booktitle={ACM Sigplan Notices},
  volume={45},
  number={9},
  pages={285--296},
  year={2010},
  organization={ACM}
}
\end{verbatim}
  \caption{Scholar lookup}
\label{fig:missing_org_scholar_lookup}
\end{figure}

As can be seen $ACM$ is the name of the organization and is probably
missing in the original Bib{\TeX} that produced the bbl file inspected
above.  The Denielsson has an additional tag with the content ``ICFP
'10'', which got added \remark{Olivier: do you happen to know what tag
  might have produced this?}

% ref 3

Following this example from arXiv.org we have entries with the same
conference where it can be seen that the conference name is different
in one of the entries in figure-\ref{fig:entry_journal_name_authors}
\remark{additional meta information perhaps}.  Also the list of author
names are in the second entry so heavily abbreviated that you cannot
realistically distinguish who the authors are, even if you know the
affiliation.

\begin{figure}[ht]
  \centering
\begin{small}
\begin{verbatim}
\bibitem{stanifordchen96grids}
S.~S.-C. \emph{et al}.
\newblock {GrIDS} -- {A} graph-based intrusion detection system 
          for large networks.
\newblock In {\em Proceedings of the 19th
          National Information Systems Security Conference},
          1996.

[...]

\bibitem{porras97emerald}
P.~A. Porras and P.~G. Neumann.
\newblock {EMERALD}: Event monitoring enabling responses 
          to anomalous live disturbances.
\newblock In {\em Proc. 20th {NIST}-{NCSC}
          National Information Systems Security Conference},
          pages 353--365, 1997.

\end{verbatim}
\end{small}
  \caption{Inconsistent reference to the conference and heavily abbreviated author names}
\label{fig:entry_journal_name_authors}
\end{figure}

% http://csrc.nist.gov/nissc/
% Different ways of writing it (Proc vs Proceedings, {NIST}--{NCSC} in one, pages omitted)
% even better, consider this entry:
Furthermore, consider the case where we use these proceedings to infer
some consistent way of writing the entries, based on the journal
names.  It turns out that ``National Information Systems Security
Conference'' used to be named ``National Computer Security
Conference'', which is probably the reason for the
$\{NIST\}--\{NCSC\}$ part of the first entry.  This results is that to
correctly identify potentiel areas of inconsistencies it should also
recognize entries such as the one in figure-\ref{fig:conference_name}.

\begin{figure}[ht]
  \centering
\begin{small}
\begin{verbatim}
\bibitem{snapp91dids}
S.~R.~S. \emph{et al}.
\newblock {DIDS} (distributed intrusion detection system) -
          motivation, architecture, and an early prototype.
\newblock In {\em Proceedings of the 14th
          National Computer Security Conference},
          pages 167--176, Washington, DC, 1991.
\end{verbatim}
\end{small}
  \caption{Name change of a conference}
\label{fig:conference_name}
\end{figure}


\remark{There should be some information about Bib{\TeX} creating a bbl
file, since me references come from there}
