\rfquote{Anything that can go wrong, will go wrong.}{Murphy's Law}

\section{Introduction}
The goal of this chapter is to introduce the challenges of {\bibtex}
by showing concrete examples, the terms structural and conjunctural
will be defined \chapref{sec:problems_structural_conjunctural}, the
use of de facto standards and why not all violations with {\bibtex}
are bad \chapref{sec:problems_de_facto}, the problems with: journal
abbreviations \chapref{sec:problems_abbreviations}, inconsistent tags
\chapref{sec:problems_inconsistent_tags}, duplicate entries
\chapref{sec:problems_duplicates}, online look ups
\chapref{sec:problems_look_ups}, inconsistent entry keys
\chapref{sec:problems_inconsistent_keys}, name changes in for instance
conference names \chapref{sec:problems_name_changes} and covering the
issues of initials \chapref{sec:problems_initials}.

\remark{need to add spelling to this chapter \\
  strings that end up in text but should've been a string, see
  \figref{fig:mendeley_output} under month}


\section{Structural vs conjunctural issues}
\label{sec:problems_structural_conjunctural}

Unfortunately, even though {\bibtex} has made life a lot better for
scientific authors, it is far from perfect and there are a lot of
issues that can be found.  Inspired by economics, the challenges in
{\bibtex} can be divided into structural and conjunctural issues. 

\begin{itemize}
\item The structural issues are the ones intrinsic to {\bibtex}, thus
  it is caused by the design, the standard tools for {\bibtex} and
  information that can not be found.

\item The conjunctural issues are the combination of circumstances,
  for instance if the source used does not the complete information
  (\eg, extracting a references from an article where the author only
  have initials even though the full name is known) or the users
  having other priorities than bibliographies.
\end{itemize}

Whether an issue is seen as conjunctural or structural is in part a
matter of opinion, as the definitions can be stretched in either
direction - an issue also faced in economics.  Most of the issues are
arguably a combination the two as they could be fixed by careful labor
or by having the right tools available, \eg, most bibliography
managers have tools to switch between abbreviated and full journal
names.  The ultimate goal of {\Orangutan} is essentially to move the
issues to the point where they are purely conjunctural as it would be
a matter of using the tool to ensure a well structured {\bibtex}
document.


\section{De facto standards}
\label{sec:problems_de_facto}

An interesting point is that not all of the structural issues are
necessarily bad.  There are a lot of ways to make use of the relaxed
properties of {\bibtex}, for instance that by design {\bibtex} ignores
unknown tags is useful in de facto standards such as commenting
entries out by prefixing with $OPT$ or adding information that are not
a part of the {\bibtex} specification, such as ISSN, DOI or the
crossref, which is technically not specified by {\bibtex} but still is
part of the tool.  In \figref{fig:mendeley_output}, an example is
provided by a PhD student from the Chemistry Department at Aarhus
University, this example is created from Mendeley (see
Section~\ref{sec:related_mendeley}) and shows a lot of additional
information about the article.

\begin{figure}
  \centering
\begin{small}
\begin{verbatim}
@article{Acatrinei2003,
author = {Acatrinei, Alice I and Browne, D and Losovyj, Y B
          and Young, D P and Moldovan, M and Chan, Julia Y
          and Sprunger, P T and Kurtz, Richard L},
doi = {10.1088/0953-8984/15/33/101},
file = {:C$\backslash$:/Users/[...]pdf},
issn = {0953-8984},
journal = {Journal of Physics: Condensed Matter},
month = {aug},
number = {33},
pages = {L511--L517},
title = {{Angle-resolved photoemission study
          and first-principles calculation
          of the electronic structure of LaSb 2}},
url = {http://iopscience.iop.org/[...]},
volume = {15},
year = {2003}
}
\end{verbatim}
\end{small}
  \caption{Output from Mendeley containing additional information}
\label{fig:mendeley_output}
\end{figure}

This design choice is an issue, if strict conformity to the
specification is desired, however as it is a very practical and widely
used feature a strict validation would be counterproductive.  Also
some formatting styles actually make use of some of the unspecified
tags as can be seen in \figref{fig:entry_with_issn}. It is still
interesting to find tags that do not conform to the specification and
the de-facto standards in use, thus treating de-facto standards as a
feature and only treating further deviations as actual issues seems to
make most sense.

% Some issues such as spelling errors are by nature conjunctural, these
% errors could be prevented by having a watchful eye and doing a spell
% check of a bib file.

%%
%% Internal ref 4
%%
\section{Journal abbreviations}
\label{sec:problems_abbreviations}

Most, if not all, journals require that journal names should be
abbreviated when publishing.  However internally in the {\bibtex} file
the owner's personal priorities are consistent and correct naming.  As
{\bibtex} can be seen as a database of references, it would make most
sense to consider full names as the most correct and the abbreviations
to be a matter of formatting.  Unfortunately {\bibtex} does not handle
abbreviations at all, which for instance is apparent in articles from
arXiv.org, as can be seen in the bbl output in
\figref{fig:inconsistent_naming}.

\begin{figure}
  \centering
\begin{small}
\begin{verbatim}
\bibitem[\protect\citename{Baroni \bgroup et al.\egroup }2014b]
          {baroni2014don}
          Marco Baroni, Georgiana Dinu, and Germ{\'a}n Kruszewski.
\newblock 2014b.
\newblock Don't count, predict!
          a systematic comparison of context-counting vs.
          context-predicting semantic vectors.
\newblock In {\em Proceedings of the 52nd Annual Meeting of
          the Association for Computational Linguistics},
          volume~1, pages 238--247.

\bibitem[\protect\citename{Bruni \bgroup et al.\egroup}2014]
          {bruni2014multimodal}
          Elia Bruni, Nam-Khanh Tran, and Marco Baroni.
\newblock 2014.
\newblock Multimodal distributional semantics.
\newblock {\em J. Artif. Intell. Res. (JAIR)}, 49:1--47.

[...]

\bibitem[\protect\citename{Collobert \bgroup et al.\egroup}2011]
          {collobert2011natural}
          Ronan Collobert, Jason Weston, L{\'e}on Bottou,
          Michael Karlen, Koray Kavukcuoglu, and Pavel Kuksa.
\newblock 2011.
\newblock Natural language processing (almost) from scratch.
\newblock {\em The Journal of Machine Learning Research},
          12:2493--2537.

[...]

\bibitem[\protect\citename{Kalchbrenner \bgroup et al.\egroup}2014]
          {kalchbrenner2014convolutional}
          Nal Kalchbrenner, Edward Grefenstette, and Phil Blunsom.
\newblock 2014.
\newblock A convolutional neural network for modelling sentences.
\newblock In {\em Proceedings of EMNLP}.
\end{verbatim}
\end{small}
  \caption{Inconsistent naming of journal and conference names}
\label{fig:inconsistent_naming}
\end{figure}

From the point of view that the style of {\bibtex} should format
abbreviations properly, the issue seen here is structural, as the long
names should have been formatted by the tool.  In cases where the
abbreviation is wrong (\eg, due to a typo), the issue would move
towards being conjunctural again, unless some kind of abbreviation
specific spell checker is being used.  Using full names and then
formatting them accordingly seems like the most sensible idea, since
the style of abbreviation could be interchanged, should the need
arise, it is more readable and it would create better conditions for
output tools to provide consistent formatting.

Currently there are multiple strategies for ensuring consistency in
abbreviations, some do a search and replace on the bib-file depending
on the formatting they need, an approach that is a bit more structured
is the use of strings in {\bibtex} to ensure a consistent naming of a
journal which can further be combined with the usage of crossref.
Another approach is the use of Bib{\LaTeX} and biber, which provide
the solution in the formatting
options\cite{koppensteiner2011abbreviate}, provided that the
abbreviation handing of the style is correct.  This causes it to
become a more conjunctural issue, as the issue now is if the right
names are written, however how conjunctural depends on the correctness
of the tools.  Bibliography managers (see
Section~\ref{sec:bibliography_managers}) such as JabRef, Mendeley
etc. tend to go with the strategy that it stores the references using
full names and then exports it to a {\bibtex} file (or whatever format
one decide to export to) having the desired abbreviation style applied
to the export, this strategy moves the issue towards being
conjunctural, for the same reasons as the Bib{\LaTeX} and biber
solution.

As the purpose is to work on the {\bibtex} files, the formatting in
the end is technically not the primary concern.  The concern would
optimally just be to ensure a consistent document, so the style can do
its work.  Since {\bibtex} styles currently does not take care of
abbreviations directly, there might be the need for considerations on
how to easily and consistently ensuring abbreviations according to a
desired format.  Having a solution that ensures a consistent
structure, which is easy to modify to a desired style of
abbreviations, would move the issue towards the conjunctural.  Having
the styles actually handling the abbreviations (such as the
Bib{\LaTeX} and biber combination does) would make the issue entirely
conjunctural if there is a an easy way to ensure that all journal
names, conference names etc. are always in correct full names.

%%
%% Internal ref 1
%% journal unknown oO
%%
\section{Inconsistent tags}
\label{sec:problems_inconsistent_tags}

Take the inconsistency in \figref{fig:entry_with_issn}, found in a
article on arXiv.org: two references from the same conference, but
different years.  The inconsistency is easy to identify due to the
consistent naming, which will make the issue even more visible to the
reader.  Correct and consistent naming will make it possible for tools
to detect inconsistencies when some fields are missing.  This exposes
a structural part of the issue, as no such tools exist (to the
authors knowledge).

\begin{figure}
  \centering
  \begin{small}
\begin{verbatim}
\bibitem[Bernardy and Claessen(2015)]{bernardy_efficient_2015}
J.-P. Bernardy and K.~Claessen.
\newblock Efficient parallel and incremental parsing
          of practical context-free languages.
\newblock \emph{J. of Funct. Prog.}, 25, 2015.
\newblock ISSN 1469-7653.
\newblock \doi{10.1017/S0956796815000131}.

[...]

\bibitem[Mu et~al.(2009)Mu, Ko, and Jansson]{MuKoJansson2009AoPA}
S.-C. Mu, H.-S. Ko, and P.~Jansson.
\newblock Algebra of programming in {Agda}:
          dependent types for relational program derivation.
\newblock \emph{J. Funct. Program.}, 19:\penalty0 545--579, 2009.
\newblock \doi{10.1017/S0956796809007345}.
\end{verbatim}
  \end{small}
  \caption{Additional tag ISSN is provided in one of the entries}
\label{fig:entry_with_issn}
% consider the pages
\end{figure}

The ISSN might not exist for the $MuKoJansson2009AoPA$-entry, in this
case a structural detection system might be the cause of new
structural issue, either the removal of relevant data or forcing
entries for cases where it is not relevant.  In this specific case the
search result in \figref{fig:entry_issn_found} reveals that the
missing ISSN does exist and thus a tool pointing out the inconsistency
would in this case make the issue conjunctural.

% Source http://journals.cambridge.org/action/displayAbstract?fromPage=online&aid=6171388&fileId=S0956796809007345#
\begin{figure}
  \centering
\begin{verbatim}
@article{Mu:2009:APA:1630623.1630627,
 author = {Mu, Shin-cheng and Ko, Hsiang-shang 
           and Jansson, Patrik},
 title = {Algebra of Programming in Agda: 
          Dependent Types for Relational Program Derivation},
 journal = {J. Funct. Program.},
 issue_date = {September 2009},
 volume = {19},
 number = {5},
 month = sep,
 year = {2009},
 issn = {0956-7968},
 pages = {545--579},
 numpages = {35},
 url = {http://dx.doi.org/10.1017/S0956796809007345},
 doi = {10.1017/S0956796809007345},
 acmid = {1630627},
 publisher = {Cambridge University Press},
 address = {New York, NY, USA},
 }
\end{verbatim}
  \caption{Search revealing the ISSN}
\label{fig:entry_issn_found}
\end{figure}

%%
%% Internal ref 2
%% Source Electronic Proceedings in Theoretical Computer Science
%%

Provided a reliable way to look up correct entries a tool could move
the issues towards being conjunctural, since it would then just being a
matter of doing the lookup on all entries and then be done (apart from
the abbreviation issues above).  Take the inconsistency in
\figref{fig:inconsistent_proceedings} where two entries from the
same conference has different information.  One has an additional
``ICFP '10'' and ``ACM'' in there, the other one does not.

\begin{figure}
  \centering
  \begin{small}
\begin{verbatim}
\bibitem[Bernardy and Claessen(2013)]{bernardy_efficient_2013}
J.-P. Bernardy and K.~Claessen.
\newblock Efficient divide-and-conquer parsing 
          of practical context-free languages.
\newblock In \emph{Proc. of ICFP 2013}, pages 111--122, 2013.

[...]

\bibitem[Danielsson(2010)]{danielsson_total_2010}
N.~A. Danielsson.
\newblock Total parser combinators.
\newblock In \emph{Proc. of ICFP 2010}, ICFP '10,
          pages 285--296. ACM, 2010.
\end{verbatim}
  \end{small}
  \caption{Capt}
\label{fig:inconsistent_proceedings}
\end{figure}

An online search for {\bibtex} information give the entries in
\figref{fig:missing_org_scholar_lookup} for the two articles,
which provides one possible option for a set of consistent entries.
As can be seen $ACM$ is the name of the organization and is probably
missing in the original {\bibtex} that produced the bbl file inspected
above.  The Danielsson has an additional tag with the content ``ICFP
'10'', which is not apparent in the search result.

\begin{figure}
  \centering
\begin{verbatim}
@inproceedings{bernardy2013efficient,
  title={Efficient divide-and-conquer parsing
         of practical context-free languages},
  author={Bernardy, Jean-Philippe and Claessen, Koen},
  booktitle={ACM SIGPLAN Notices},
  volume={48},
  number={9},
  pages={111--122},
  year={2013},
  organization={ACM}
}

@inproceedings{danielsson2010total,
  title={Total parser combinators},
  author={Danielsson, Nils Anders},
  booktitle={ACM Sigplan Notices},
  volume={45},
  number={9},
  pages={285--296},
  year={2010},
  organization={ACM}
}
\end{verbatim}
  \caption{Scholar lookup}
\label{fig:missing_org_scholar_lookup}
\end{figure}

\section{Duplicate entries}
\label{sec:problems_duplicates}

When a group of authors are writing on the same document work is often
divided and each author will have their own bib-file that they use for
their references.  All might be nice and happy until the group
combines their document and it turns out that multiple authors has
used the same reference but with different keys, ending up having a
duplicate entry in the bibliography.

Again having it arguably it can be seen as a structural or a
conjunctural problem.  The similarity of such entries will make them
easier to spot with the naked eye, arguably making a structural
solution less relevant.  However the due to the similarity (at least
if abbreviated the same way) a structural solution is arguably also
easy to make.  Making this problem purely conjunctural would require a
way to detect and merge these entries, furthermore should they use
different keys the corresponding document should also be corrected.

A similar issue, which actually happened in one of the drafts for this
document, is when an author copy pastes a template for en entry with
the intend to adjust it give it a key and then forget to adjust the
entry.  This will also result in a duplicate, but unlike the situation
above the duplicate should not be merged but contain different
content.

Having a tool that just merges duplicate entries may provide a
structural solution to the first case but will create a structural
issue in the second as it will automatically introduce new issues.
Thus the solution should not merge automatically.


\section{Online look ups}
\label{sec:problems_look_ups}

In the utopic case all entries could be looked up at all times, but
this care is not likely as unpublished works would need a reliable way
to be registered.  Also even though the databases out there are really
good, most of the time, erronous results can be found.  A lookup on
Google Scholar in the beginning of February 2016 for: ``Results and
Analysis of SyGuS-Comp’15'' can be seen in
\figref{fig:scholar_bad_result}, which contains erronous output.

\begin{figure}
  \centering
\begin{verbatim}
@article{alurresults,
  title={Results and Analysis of SyGuS-Comp’15},
  author={Alur, Rajeev and Fisman, Dana and Singh, Rishabh
          and Solar-Lezama, Armando}
}
\end{verbatim}
  \caption{Bad result from Google Scholar}
\label{fig:scholar_bad_result}
\end{figure}

Having found the article originally on arXiv.org the source of the
article is known to be EPTCS - Electronic Proceedings in Theoretical
Computer Science.  So not only does the Google Scholar result actually
not conform to the requirements of an article, the resource are in
fact not an article at all, but in the proceedings to a conference.
Finding the correct entry details at the EPTCS page reveals the entry
in \figref{fig:eptcs_lookup}.  Relying blindly on these being
correct would cause structural issues as the tools could automatically
introduce new errors.  Using it for suggestions can move the issue
towards being conjunctural for most cases.

\begin{figure}
  \centering
\begin{small}
\begin{verbatim}
@Inproceedings{EPTCS202.3,
  author    = "Alur, Rajeev and Fisman, Dana  and Singh, Rishabh 
               and Solar-Lezama, Armando",
  year      = "2016",
  title     = "Results and Analysis of SyGuS-Comp'15",
  editor    = "\v{C}ern\'y, Pavol and Kuncak, Viktor 
               and Parthasarathy, Madhusudan"b,
  booktitle = "{\rm Proceedings Fourth Workshop on}
               Synthesis,
               {\rm San Francisco, CA, USA, 18th July 2015}",
  series    = "Electronic Proceedings in 
               Theoretical Computer Science",
  volume    = "202",
  publisher = "Open Publishing Association",
  pages     = "3-26",
  doi       = "10.4204/EPTCS.202.3",
}
\end{verbatim}
\end{small}
  \caption{Correct lookup on EPTCS, after failed lookup on Google Scholar}
\label{fig:eptcs_lookup}
\end{figure}


\section{Inconsistent entry keys}
\label{sec:problems_inconsistent_keys}

The key for the entries may vary through a document in the reference,
for instance one of the users collaborating earlier might just grab
entries from various online databases getting keys corresponding to
\figref{fig:missing_org_scholar_lookup} and \figref{fig:eptcs_lookup}
in one big mess.  Had the user been working alone this inconsistency
would arguably be one's own headache, after all one might not even
remember the entries by their given key. But as the user is
collaborating with others they might influenced by it.  Another
potential challenge could be avoiding duplicate key names which with a
consistent structure would be easier and a duplicate key could give an
indicator of a duplicate entry.  One of the users collaborating might
also be new to using {\bibtex} (could be a student learning) and need
to find a nice an consistent way of writing the keys.

Inconsistencies in keys might range from not a problem at all to fully
conjectural or structural, highly depending on the users point of
view.  The user who does not use they keys to search for entries may
simply not care and apart from the potential to detect duplicate
entries may not have a reason to.  However in the collaboration
scenario others might be affected by it, and an additional way of
detecting duplicates (see Section~\ref{sec:problems_duplicates}).  In
this case it arguably becomes conjunctural as the authors should have
agreed on a style - which might also have helped a newcomer to have a
good practice, but it can yet again be argued that it is structural as
{\bibtex} does not provide naming guidelines nor tools for ensuring
those.

To make this issue structural a naming convention would have to be
agreed upon and some way of enforcing it provided.  This could result
in two entries with the same key, causing a key name clash, provided
a that a solution for the duplicate entry issue is already in place a
need for way to handle key name clashes.


% ref 3
\section{Name changes in conferences, journals etc.}
\label{sec:problems_name_changes}

In \figref{fig:missing_org_scholar_lookup} spotting the
consistency issues were relatively simple, and a tool that can find it
is possible, since the conference have the same name.  When looking at
\figref{fig:entry_journal_name_authors} it can be seen that the
conference name is slightly different in one of the entries, but so
close that they are probably the same conference.

\begin{figure}
  \centering
\begin{small}
\begin{verbatim}
\bibitem{stanifordchen96grids}
S.~S.-C. \emph{et al}.
\newblock {GrIDS} -- {A} graph-based intrusion detection system 
          for large networks.
\newblock In {\em Proceedings of the 19th
          National Information Systems Security Conference},
          1996.

[...]

\bibitem{porras97emerald}
P.~A. Porras and P.~G. Neumann.
\newblock {EMERALD}: Event monitoring enabling responses 
          to anomalous live disturbances.
\newblock In {\em Proc. 20th {NIST}-{NCSC}
          National Information Systems Security Conference},
          pages 353--365, 1997.

\end{verbatim}
\end{small}
  \caption{Inconsistent reference to the conference and heavily abbreviated author names}
\label{fig:entry_journal_name_authors}
\end{figure}

Diving into that additional information a visit to the homepage of the
conference reveals that ``National Information Systems Security
Conference'' used to be named ``National Computer Security
Conference'', which is probably the reason for the
$\{NIST\}--\{NCSC\}$ part of the first entry\cite{nist2014_nissc}.  In
the same source it turns out that there are also references to the
conference from before the name change, as seen in
\figref{fig:conference_name}, so to correctly identify potential areas
of inconsistencies it should also recognize name changes and
variations.

\begin{figure}
  \centering
\begin{small}
\begin{verbatim}
\bibitem{snapp91dids}
S.~R.~S. \emph{et al}.
\newblock {DIDS} (distributed intrusion detection system) -
          motivation, architecture, and an early prototype.
\newblock In {\em Proceedings of the 14th
          National Computer Security Conference},
          pages 167--176, Washington, DC, 1991.
\end{verbatim}
\end{small}
  \caption{Name change of a conference}
\label{fig:conference_name}
\end{figure}

In {\bibtex} there is no way of specifying that the same conference
has different names so there is a big structural part as there is no
support for identifying this issue, furthermore the owner of the
{\bibtex} file might not even be aware which arguably could be
conjunctural as the user could do his research or structural because
the user should have a tool that can assist.  If a tool could reliably
detect inconsistencies from the same conference with different names
the name changes would become conjunctural.


\section{Initials}
\label{sec:problems_initials}

Another issue in \figref{fig:conference_name} is the list of author
names that are so heavily abbreviated to initials that one cannot
realistically distinguish who the authors are.  This might originate
from the used citation style or from a resource where they are already
abbreviated.

If the initials comes from the citation style the issue is of a
structural nature.  However if it is copied again be conjunctural, as
the user did not check the names.  A structural version of the copy
issue is if the reference is copied by a tool it can be argued that
such a tool should try to detect initials.  Having a way to detect the
initials may again move the issue towards conjunctural however to make
it entirely conjunctural we would need a reliable way to know when the
initials are deliberate or not, as some people never use or reveal
that part of their name.


\section{Summary and conclusions}
\label{sec:problems_conclusion}

To summarize: {\bibtex} has a lot of issues which can be analyzed from
the perspective of if they are structural and conjunctural with the
goal of how the issues can become more conjunctural.  No all
structural issues are bad, since some of them allow for useful de
facto standards.  The problems covered are: journal abbreviations
where they can be cumbersome for analyzing tools and that {\bibtex}
does not handle abbreviations fully, inconsistent tags resulting in
non-uniform bibliographies and some times indicating missing
information, duplicate entries, online look ups which may some times
contain bad data, inconsistent entry keys which can be an issue in
collaboration and may make it harder to detect duplicates, name
changes such as conferences that change name which can affect
detection of inconsistent tag usage and how initials can hide peoples
names and challenge detection tools.

With all these problems at hand, {\bibtex} may not seem like the
optimal tool after all.  Being the widely used tool that it is, doing
something about it quite desirable.  But what can be done with such a
range of issues?
